\documentclass{article}
\usepackage[utf8]{inputenc}
\usepackage{amsmath}

\title{Laboratorio di Elettronica}
\author{Cesare Sabattini Manginella}
\date{\today}

\begin{document}

\maketitle

\section{Introduzione}
ATTENZIONE: formulario ancora in fase di sviluppo. Alcune informazioni potrebbero essere incomplete o errate.

\section{Teoria delle bande energetiche nei cristalli}
Alla distanza reticolare del cristallo, nell'ambito di questo modello, sono identificbili tre bande:
\begin{itemize}
    \item la banda di conduzione, in cui gli elettroni possono muoversi liberamente;
    \item la banda di valenza, in cui gli elettroni sono legati al nucleo.
    \item la banda proibita, in cui non possono esistere stati elettronici.
\end{itemize}

In termini delle loro proprietà di conducibilità, i materiali possono essere classificati in:
\begin{itemize}
    \item conduttori, in cui la banda di valenza e la banda di conduzione si sovrappongono;
    \item semiconduttori, in cui la banda proibita è facilmente superabile;
    \item isolanti, in cui la banda proibita è molto ampia.
\end{itemize}

Il gap energetico corrispondente alla banda proibita è superabile dagli elettroni per mezzo di apporto energetico termico o di un campo elettrico esterno.
La variabilità della conducibiltà dei materiali è variabile, ma particolarmente alta per i semiconduttori puri, o intrinseci.

L'energia minima per la transizione alla banda di conduzione è ovviamente pari a:

\begin{equation}
    E_g = E_c - E_v
\end{equation}

\subsection*{Mobilità e conducibilità}
\subsubsection{Conduttori}
Il reticolo cristallino di un conduttore è approssimabile ad una successione di ioni positivi e un gas elettroni liberi,
in moto, in assenza di campo elettrico esterno, con velocità termica, con media nulla.
Il modulo della velocità termica è molto maggiore di quella di deriva, in presenza di campo elettrico esterno.
L'applicazione di un campo elettrico si traduce dunque in una rottura dell'isotropia della distribuzione delle velocità, con conseguente deriva degli elettroni
lungo la direzione del campo elettrico.
Il modulo della velocità di deriva è dato da:

\begin{equation}
    v_d = \mu \epsilon
\end{equation}

dove $\mu$ è la mobilità elettronica, e $\epsilon$ è la differenza di potenziale applicata.
La densità di corrente è data, nel caso di isotropia, da:

\begin{equation}
    J = n e v_d= \rho v_d=n q \mu \epsilon= \sigma \epsilon
\end{equation}

dove $n$ è la densità di elettroni, $e$ è la carica elementare, $\rho$ è la densità di carica, $q$ è la carica elettrica, $\sigma$ è la conducibilità elettrica.
Si è così ricavata la legge di Ohm, a livello microscopico:

\begin{equation}
    J = \sigma \epsilon
\end{equation}
\subsubsection{Semiconduttori puri}
Il legame chimico nella shell di valenza è di tipo covalente.
A basse temperature, tale configurazione risulta anloga a quella di un isolante.
Aumentando la temperatura, si ha un aumento della conducibilità, dovuto alla ionizzazione degli atomi,
per via del superamento del gap energetico.
In corrispondenza dell'elettrone liberato, si genera una lacuna, formalizzabile come una carica positiva.

Il processo di creazione di una coppia elettrone-lacuna determina due possibili processi:

\begin{itemize}
    \item L'elettrone di valenza liberato transisce alla banda di conduzione, mentre la lacuna è immediatamente saturata.
          Ciò coincide nella composizione di due moti opposti elettroni-lacune.
    \item L'elettrone ricade nella lacuna, con conseguente annichilazione delle due cariche in termini di conducibilità.
\end{itemize}

In un semiconduttore intrinseco, il numero di lacune è sempre uguale a quello di elettroni liberi.
Analogamente per la densità:

\begin{equation}
    n = p = n_i
\end{equation}

dove $n_i$ è la densità intrinseca, che aumenta con la temperatura:

\begin{equation}
    n_i^2= A_0T^3 e^{-\frac{E_g}{kT}}
\end{equation}

dove $A_0$ è una costante, $E_g$ è il gap energetico a 0K, $k$ è la costante di Boltzmann, $T$ è la temperatura.

In presenza di un campo elettrico esterno, la densità di corrente totale J è pari a:

\begin{equation}
    J = J_n + J_p = q(n \mu_n + p \mu_p) \epsilon= \sigma \epsilon
\end{equation}

SI ricava immediatamente la relazione tra la conducibilità e la temperatura:

\begin{equation}
    \sigma= q(n \mu_n + p \mu_p)= qn_i(\mu_n + \mu_p)= q(\mu_n + \mu_p)A_0T^{3/2} e^{-\frac{E_g}{2kT}}
\end{equation}

Anche la mobilità è funzione della T, a la sua variazione è trascurabile, in termini di conduzione, rispetto all'aumento della densità intrinseca.

\subsection{Distribuzione di energia dei portatori di carica}
\subsubsection{Conduttori}
\begin{equation}
    d n_e= \rho dE
\end{equation}

$\rho$ è la densità di elettroni nell'intervallo dE, ed è data da:

\begin{equation}
    \rho = f(E) N(E)
\end{equation}

con f probabilità di occupazione di uno stato e N densità degli stati nella banda di conduzione.
N(E) è data da:

\begin{equation}
    N(E) = \gamma E^{1/2}
\end{equation}

con $\gamma$ costante.
La densità di probabilità di occupazione è data dalla funzione di probabilità di Fermi-Dirac:

\begin{equation}
    f(E) = \frac{1}{1+e^{\frac{E-E_F}{kT}}}
\end{equation}

Tale funzione indica la distribuzione in energia degli elettroni internamente ad un conduttore, ove con $E_F$ si indica l'energia di Fermi, che rappresenta l'energia massima di un elettrone a T=0K, e per T>0K rappresenta l'energia a cui la probabilità di occupazione è 1/2.
Dato che il numero di elettroni liberi in un conduttore è invariante per cambio di Temperatura, in un ampio range,
possiamo stimare la densità elettronica a T=0K:

\begin{equation}
    n = \int_{0}^{E_F} \rho dE =\int_{0}^{E_F} \gamma E^{\frac{1}{2}}=\frac{2}{3}\gamma E_F^{3/2}
\end{equation}

\subsubsection{Semiconduttori puri}
Analogamente ai conduttori, la densità di elettroni è data da:

\begin{equation}
    d n_e= \rho dE= f(E) N(E)
\end{equation}

Data E$_c$ l'energia minima della banda di conduzione, si può scrivere la densità degli stati come:

\begin{equation}
    N(E) = \gamma (E-E_c)^{1/2} \quad E > E_c
\end{equation}

la probablità di occupazione, invece, analogamente ai conduttori, è data dalla distribuzione di Fermi-Dirac:

\begin{equation}
    f(E) = \frac{1}{1+e^{\frac{E-E_F}{kT}}}
\end{equation}

La probabilità di occupazione tende a 0 per (E-$E_F$) >> kT, e a 1 per (E-$E_F$) << kT.
La densità di elettroni è data da:

\begin{equation}
    n = \int_{E_c}^{\infty} \rho dE = \int_{E_c}^{\infty} f(E) N(E) dE \circ \gamma \int_{E_c}^{\infty} (E-E_c)^{1/2} \frac{1}{e^{\frac{E-E_F}{kT}}} dE= N_c e^{\frac{E_F-E_c}{kT}}
\end{equation}

Dove N$_c$ coincide con la concentrazione intrinseca degli elettroni nel caso in cui l'Energia di Fermi coincida con l'energia minima della banda di conduzione.
Considerando che l'Energia di Fermi è sempre al centro della banda proibita, ciò equivale al caso limite di un conduttore.
Dunque la densità degli elettroni aumenta con la temperatura.

Nel caso delle lacune, invece, si ripete un ragionamento analogo, ottenendo:

\begin{equation}
    p = N_v e^{\frac{E_v-E_F}{kT}}
\end{equation}

\subsubsection{Livello di Fermi}

Per un semiconduttore puro, n=p, ossia:

\begin{equation}
    n_i = N_c e^{\frac{E_F-E_c}{kT}} = N_v e^{\frac{E_v-E_F}{kT}}
\end{equation}

da cui si ricava, considerando le masse delle particelle fittizie rappresentanti le lacune e quelle degli elettroni dello stesso ordine di grandezza, si ottiene
una stima dell'energia di Fermi:

\begin{equation}
    E_F = \frac{E_c + E_v}{2} + \frac{kT}{2} ln(\frac{N_v}{N_c})\circ \frac{E_c+E_v}{2}
\end{equation}

a supporto dell'affermazione precedente.

\subsubsection{Concentrazione intrinseca}

Il prodotto n p è dato da:

\begin{equation}
    n p = n_i^2 = N_c N_v e^{\frac{E_c-E_v}{kT}}= N_c N_v e^{\frac{E_g}{kT}}
\end{equation}

In un semiconduttore intrinseco, n=p, da cui si ricava la legge dell'azione di massa:

\begin{equation}
    np = n_i^2
\end{equation}

che implica che il prodotto delle concentrazioni è costante, data una temperatura.
Si trova in particolare la dipendenza di n$_i$ dalla temperatura:

\begin{equation}
    n_i = A_0 T^{3/2} e^{\frac{-E_g}{2kT}}
\end{equation}

Si è così riottenuta l'equazione (6).

\subsection{Semiconduttori drogati}
Inserendo impurezze chimiche in un semiconduttore, si può ottenere una condizione tale per cui risulti n$\neq$p.
La variabilià introdotta dipende principalmente dal livello di valenza dell'atomo drogante.
Per esempio, un atomo del quinto gruppo della tavola periodica, come il fosforo, introduce un elettrone in più rispetto al silicio, e dunque si parla di drogaggio di tipo n.
Ciò è realizzabile a patto di fornire la necessaria energia di attivazione dei donatori.
Al contrario, un atomo del terzo gruppo, come il boro, introduce una lacuna in più rispetto al silicio, e dunque si parla di drogaggio di tipo p.
n tal caso è necessario fornire l'energia di attivazione degli accettori.
A temperatura ambiente tutti gli atomi droganti sono da considerarsi inonizzati.
Se si aggiungono droganti in concentrazioni maggiori dell'intrinseca, si possono ottenere 3 effetti:

\begin{itemize}
    \item Si ottiene un conduttore in cui i portatori sono di tipo p o n, a seconda del drogaggio;
\end{itemize}

Dato che un semiconduttore drogato è comunque elettricamente neutro, deve valere:

\begin{equation}
    n + N_d = p + N_a
\end{equation}

\subsubsection{Drogaggio di tipo n}

Trascurando la concentrazione intrinseca, per un semiconduttore drogato di tipo n si ha:

\begin{equation}
    n \simeq N_d
\end{equation}

la concentrazione di lacune è pertanto:

\begin{equation}
    p = \frac{n_i^2}{n} = \frac{n_i^2}{N_d}
\end{equation}

\subsubsection{Drogaggio di tipo p}

Analogamente, per un semiconduttore drogato di tipo p si ha:

\begin{equation}
    p \simeq N_a
\end{equation}

la concentrazione di elettroni è pertanto:

\begin{equation}
    n = \frac{n_i^2}{p} = \frac{n_i^2}{N_a}
\end{equation}

\subsubsection{Conducibilità}

Al fine di ottenere una variazione apprezzabile della conducibilità, è necessario che la concentrazione di droganti sia molto piccola.
La conducibilità è data da:

\begin{equation}
    \sigma = q(n \mu_n + p \mu_p)
\end{equation}

La resistività di un semiconduttore drogato è indipendente dalla temperatura, per un ampio range di valori.

\subsubsection{Livello di Fermi nei semiconduttori drogati}

In assenza di impurità, il livello di Fermi coincide con l'energia di Fermi, al centro della banda proibita,
per via della medesima concentrazione di elettroni e lacune.

Se l'impurità è di tipo n, il livello di Fermi si trasla verso la banda di conduzione, in quanto la concentrazione di elettroni è maggiore di quella di lacune.
Ciò, probabilisticamente, implica che la probabilità di occupazione degli stati nella banda di conduzione è maggiore di quella nella banda di valenza,
essendo $E_f$ l'energia alla quale corrisponde una probabilità di occupazione di 1/2.

Trascurando la concentrazione intrinseca degli elettroni, si ha:

\begin{equation}
    N_D= N_c e^{\frac{E_c-E_F}{kT}} \rightarrow E_F = E_c + kT ln(\frac{N_D}{N_c})
\end{equation}

Simmetricamente, per un drogaggio di tipo p, si ha:

\begin{equation}
    N_A= N_v e^{\frac{E_F-E_v}{kT}} \rightarrow E_F = E_v - kT ln(\frac{N_A}{N_v})
\end{equation}

Da notare che all'aumentare della temperatura, il livello di fermi si sposta nuovamente al centro della banda proibita, in accordo con le considerazioni precedenti.

\subsection{Corrente di diffusione}

L'operazione di drogaggio può comportare un gradiente p o n all'interno del semiconduttore.
Ciò è il risultato di un puro processo di diffusione, non elettrostatico. Pertanto, la corrente di
diffusione è esprimibile mediante la legge di Fick:

\begin{equation}
    J_n = q D_n \frac{dn}{dx}
\end{equation}

simmetricamente, per le lacune:

\begin{equation}
    J_p = -q D_p \frac{dp}{dx}
\end{equation}

La relazione tra diffusione e mobilità è data dalla relazione di Einstein:

\begin{equation}
    \frac{D_p}{\mu_p} = \frac{D_n}{\mu_n} = V_T= \frac{\tilde{k}T}{q}
\end{equation}

Segue che la densità di corrente totale per gli elettroni è data da:

\begin{equation}
    J_n= q\mu_n n \epsilon+ qD_n \frac{dn}{dx}
\end{equation}

e simmetricamente per le lacune:

\begin{equation}
    J_p= q\mu_p p \epsilon- qD_p \frac{dp}{dx}
\end{equation}

\subsection{Generazione e ricombinazione di cariche}
Considero un semiconduttore di tipo n, con concentrazioni $p_0$ e $n_0$ all'equilibrio termico, con $n_0 >> p_0$.
Se lo si illumina uniformemente, con fotoni ad energia superiore a quella di gap, si ha la formazione di coppia p-n.
Dopo un transiente, si ha:

\begin{equation}
    \left\{
    \begin{aligned}
         & \Delta n= \overline{n} - n_0 \\
         & \Delta p= \overline{p} - p_0
         & \Delta n = \Delta p
    \end{aligned}
    \right.
\end{equation}

Percentualmente, è maggiore la variazione di lacune rispetto a quella di elettroni.

Analizziamo l'andamento della concentrazione delle lacune dopo aver spento la luce, assumendo $\tau_p$ indipendente dalla concentrazione di lacune.
Detto $\frac{p}{\tau_p}$ è la diminuzione di lacune per ricombinazione, $\frac{p_0}{\tau_p}$ è l'aumento delle lacune per generazione termica.
La variazione del numero di lacune nel tempo è dunque, esprimendo il numero di lacune aggiuntive rispetto a $p_0$ come p':
\begin{equation}
    \frac{dp}{dt} = \frac{p_0-p}{\tau_p} \rightarrow \frac{dp'}{dt}= -\frac{p'}{\tau_p} \rightarrow p'= p_0 e^{-\frac{t}{\tau_p}}
\end{equation}

\subsection{Equazione di continuità}


Considerando un volume infinitesimo di semiconduttore e supponendo la corrente funzione della sola coordinata x, si ha:

\begin{equation}
    \frac{1}{qA}\frac{dI_p}{dx}=\frac{1}{q}\frac{dJ_p}{dx}
\end{equation}

Per quanto ricavato nel paragrfo precedente, la variazione della concentrazione di lacune per via della ricombinazione dovrà eguagliare l'espressione appena ottenuta della densità di corrente di diffusione
per le lacune stesse:

\begin{equation}
    \frac{\partial}{\partial t} p= \frac{p_0-p}{\tau_p}- \frac{1}{q}\frac{\partial J_p}{\partial x}
\end{equation}

\subsection{Low Level Injection}
Considerando l'irraggiamento di una faccia di un semiconduttore drogato n con fotoni ad energia sufficiente, studio l'andamento dei portatori minoritari in condizioni stazionarie, con il fascio acceso, in funzione di x.
Considero l'intensità del fascio debole, in modo tale che il numero di minoritari rimanga inferiore a quello dei maggioritari (???).
La corrente di deriva, in assenza di campo elettrico, è nulla, dunque:

\begin{equation}
    \begin{aligned}
         & J_p=-qD_p\frac{dp}{dx}                                                                            \\
         & \frac{\partial}{\partial t} p= \frac{p_0-p}{\tau_p}- \frac{1}{q}\frac{\partial J_p}{\partial x}=0 \\
         & \frac{d^2}{dx^2} p= \frac{p-p_0}{D_p \tau_p}\equiv \frac{p'}{L_p^2}
    \end{aligned}
\end{equation}

Dove $L_p$ è la lunghezza di diffusione delle lacune.
Questa equazione prende il nome di equazione di Helmholtz.
La sua soluzione, considerando che non deve divergere in x, è:

\begin{equation}
    p(x)= p'(0)e^{-\frac{x}{L_p}}
\end{equation}

\subsubsection{Correnti dei portatori di carica}
La corrente di diffusione dei minoritari (lacune) è pari a, in assenza di campo elettrico:

\begin{equation}
    \left\{
    \begin{aligned}
         & I^{diff}_p= -qAD_p \frac{dp}{dx}= qAD_p \frac{p'}{L_p}e^{-\frac{x}{L_p}} \\
         & p(x)= p'(0)e^{-\frac{x}{L_p}}
    \end{aligned}
    \right.
\end{equation}

per i maggioritari:

\begin{equation}
    \left\{
    \begin{aligned}
         & I^{diff}_n= qAD_n \frac{dn}{dx}=qAD_n\frac{dp}{dx}=-\frac{D_n}{D_p}I^{diff}_p \equiv -RI^{diff}_p
         & n-n_0=p-p_0
    \end{aligned}
    \right.
\end{equation}

Dunque la corrente di diffusione totale è pari a:

\begin{equation}
    I^{diff}_n+I^{diff}_p=-\frac{D_n}{D_p}I^{diff}_p+I^{diff}_p= (1-\frac{D_n}{D_p})I^{diff}_p<0
\end{equation}

Però siamo in assenza di campo elettrico esterno, dunque la corrente totale deve essere nulla.

Motivo per cui ci aspettiamo che esista una componente di deriva, dovuta unicamente ai maggioritari, pari a:

\begin{equation}
    I^{de}_n=I^{diff}_p\left(\frac{D_n}{D_p}-1\right)
\end{equation}

Dimostro che la corrente di deriva dei minoritari sia trascurabile:

\begin{equation}
    \begin{aligned}
         & J^{de}_n=nq\mu_n\epsilon_I \rightarrow \epsilon_I=\frac{1}{nq\mu_nA}\left(\frac{D_n}{D_p}-1\right)I^{diff}_p     \\
         & I^{de}_p=pq\mu_p\epsilon_IA=\frac{p}{n}\frac{\mu_p}{\mu_n}\left(\frac{D_n}{D_p}-1\right)I^{diff}_p << I^{diff}_p
    \end{aligned}
\end{equation}

Dove la minorazione è dovuta al fattore di proporzionalità $\frac{p}{n}$, che è molto minore di 1 in un semiconduttore drogato n.

\section{Effetto Hall}

L'effetto Hall costituisce la verifica sperimentale dell'assunzione delle lacune come portatrici di carica positive.
Consideriamo un semiconduttore percorso da una corrente I lungo l'asse x, in un campo magnetico lungo z positivo.
Segue che verrà indotto un campo elettrico lungo la direzione in y.
In base alla differenza di potenziale indotta, è possibile determinare il segno dei portatori di carica.

Inoltre, la tensione di Hall permette di determinare la densità di portatori di carica, e, nota la conducibilità, la mobilità.

\begin{equation}
    \begin{aligned}
         & q\epsilon_H=qvB \rightarrow \epsilon_H=vB \rightarrow V_H=vBd                                      \\
         & V_H=\frac{I}{wdnq}Bd=\frac{1}{nq}\frac{IB}{w} \rightarrow n=\frac{1}{V_Hq}\frac{IB}{w} \rightarrow \\
         & \begin{aligned}
                & \mu_n=\frac{1}{nq}\sigma \quad \text{In un semiconduttore drogato n} \\
                & \mu_p=\frac{1}{pq}\sigma \quad \text{In un semiconduttore drogato p}
           \end{aligned}
    \end{aligned}
\end{equation}

In un semiconduttore puro, essendo la mobilità degli elettroni  maggiore di quella delle laune, si ha una tensione totale negativa.

\section{Metodi di drogaggio}
*Da terminare*

\section{Giunzione p-n}
Una giunzione p-n è una sezione di semiconduttore che separa una regione drogata p da una drogata n.
Trascuriamo i minoritari. Le differenze di concentrazione determinano la formazione di una corrente di diffusione.
Attraversata la giunzione, iniziano i processi di ricombinazione. Gli ioni che si vengono a formare prendono il nome di "cariche scoperte".
Tali ioni generano un campo elettrico, che si oppone alla diffusione dei portatori di carica, raggiunto il "potenziale di contatto".

A questo punto, interrotta la transizione delle cariche attraverso la giunzione per via del campo elettrico generato dagli ioni, si vengono a formare 4 zone:

due, corrispondenti alle regioni drogate, due elettricamente cariche, corrispondenti alle regioni di cariche scoperte (regione di svuotamento).

L'espressione del campo elettrico all'equilibrio è data da:

\begin{equation}
    \begin{aligned}
         & J_p=q\mu_pp\epsilon-qD_p\frac{dp}{dx}=q\mu_pp\epsilon-q\mu_pV_T\frac{dp}{dx}=0 \\
         & \epsilon=\frac{V_T}{p}\frac{dp}{dx}
    \end{aligned}
\end{equation}

Integrando il potenziale:

\begin{equation}
    \begin{aligned}
         & dV=-V_T\frac{dp}{p} \rightarrow V=V_t\ln{\frac{p_1}{p_2}}=\frac{kT}{q}\ln{\frac{p_1}{p_2}}=\frac{kT}{q}\ln{\frac{n_2}{n_1}} \rightarrow \\
         & n_1p_1=n_2p_2
    \end{aligned}
\end{equation}

Ciò significa che all'equilibrio, il prodotto np è costante, indipendentemente dal drogaggio.
A questo punto risulta possibile calcolare la differenzia di potenziale di contatto:

\begin{equation}
    V_0=V_T \ln{\frac{p_1n_2}{n_i^2}} \simeq V_T\ln{\frac{N_AN_D}{n_i^2}}
\end{equation}

dove si sono considerate le zone corrispondenti alle regioni drogate all'equilibrio.
Per l'equazione di Poisson:

\begin{equation}
    \begin{aligned}
         & \frac{d^2}{dx^2} V=-\frac{\rho}{\tilde{\epsilon}}\rightarrow \\
         & \frac{d\epsilon}{dx}=\frac{\rho}{\tilde{\epsilon}}
    \end{aligned}
\end{equation}

\subsection{Energia di Fermi}
L'energia di Fermi è costante in un semiconduttore puro, motivo per cui saranno le bande di conduzione e di valenza a variare in funzione del drogaggio.

Lo shift energetico tra le bande è pari a:

\begin{equation}
    E_0=E_{C_p}-E_{C_n}= E_{V_p}-E_{V_n}=E_1+E_2= |\frac{E_{C_p}-E_{V_p}}{2}-E_f|+|\frac{E_{C_n}-E_{V_n}}{2}-E_f|
\end{equation}

si deduce che:

\begin{equation}
    \begin{aligned}
         & E_0=E_1+E_2=E_g-(E_{C_n}-E_{C_F})-(E_F-E_{V_p }) \rightarrow                  \\
         & \left\{\begin{aligned}
                       & E_{C_n}-E_F=kT\ln{\frac{N_c}{N_d}} \quad \text{drogaggio n} \\
                       & E_F-E_{V_p}=kT\ln{\frac{N_v}{N_a}} \quad \text{drogaggio p}
                  \end{aligned}
        \right.
    \end{aligned}
\end{equation}

da cui si può trovare E$_0$:

\begin{equation}
    E_0=kT\ln{\frac{N_dN_a}{n_i^2}}=V_0
\end{equation}

E' possibile ricavare la relazione di Einstein:

\begin{equation}
    \begin{aligned}
         & J_p=q\mu_pp\epsilon-qD_p\frac{dp}{dx}=0                                    \\
         & \frac{dp}{p}=\frac{\mu_p}{D_p}\epsilon dx=-\frac{\mu_p}{D_p}dV \rightarrow \\
         & \frac{D_p}{\mu_p}=V_T
    \end{aligned}
\end{equation}

\subsection{Giunzione polarizzata (diodo)}

La polarizzazione si dice:

\begin{itemize}
    \item diretta, se la regione p è collegata al terminale positivo; segue un'estensione della depletion zone.
    \item inversa, se la regione p è collegata al terminale negativo; segue una riduzione della depletion zone.
\end{itemize}
In una polarizzazione inversa, i maggioritari sono attratti verso i terminali elettrici, mentre i minoritari sono respinti, e tendono ad attraversare la giunzione.

Dunque, se la corrente dei maggioritari dura solo per un transiente, quella dei minoritari, a temperatura fissata, continua indefinitamente e dipende solamente dalla concentrazione intrinseca.

Tale corrente prende il nome di "corrente inversa", ed è indipendente, in un ampio intervallo, dall'intensità della tensione applicata.

Questa corrente, bassa, ma non nulla, si comporta come un resistore ad elevata resistenza.

Tramite polarizzazione diretta, invece, i maggioritari vengono attratti dal terminale della regione opposta, per cui attraversano la giunzione, determinando un'alta densità di corrente,
per via dell'abbassamento della barriera di potenziale di giunzione.
Tale fenomeno risulta in un restringimento della depletion zone.

Fornendo una tensione maggiore o uguale a $V_0$ ci si aspetterebbe così una corrente infinita, cosa non vera, in quanto è necessario tenere conto
della resistenza intrinseca del semiconduttore.

Si definisce diodo un dispositivo bipolo formato da una sola giunzione p-n

\subsubsection{Componenti della corrente in un diodo a giunzione}

Considero un diodo polarizzato direttamente sotto le seguenti approssimazioni:

\begin{itemize}
    \item $V_d \simeq V_0$ (low level injection condition)
    \item Le corrnti dei minoritari sono trascurabili.
    \item il potenziale è applicato direttamente alla depletion one.
    \item Si trascura lo spessore della depletion zone
\end{itemize}

In prossimità della giunzione, si è in una situazione di non equilibrio, data dal fatto che:

\begin{equation}
    \left\{
    \begin{aligned}
         & p_nn_n>>n_i^2 \quad \text{regione n} \\
         & p_pp_n<<n_i^2 \quad \text{regione p}
    \end{aligned}
    \right.
\end{equation}

\begin{equation}
    \begin{aligned}
         & J_p=q\mu_pp\epsilon-qD_p\frac{dp}{dx}=J_p^{Dif,LLI}\simeq 0 \rightarrow                                       \\
         & q\mu_pp\epsilon=qD_p\frac{dp}{dx} \rightarrow \epsilon= \frac{V_T}{p}\frac{dp}{dx}=-\frac{dV}{dx} \rightarrow \\
         & \int_{p_{p_0}}^{p_n(0)}\frac{dp}{p}=-\int_{0}^{V_j}\frac{dV}{V_T} \rightarrow                                 \\
         & p_n(0)=p_{n_0}e^{\frac{V_D}{V_T}}
    \end{aligned}
\end{equation}

si è così ottenuta la "Legge della giunzione".

Si possono ora ricavare le componenti della corrente in un diodo a giunzione.

In condizione di low level injection, la corrente dei portatori minoritari è solo di diffusione, e si può scrivere:

\begin{equation}
    I_p^{Diff}=\frac{qAD}{L_p}p'e^{-\frac{x}{L_p}}
\end{equation}

Nella giunzione polarizzata:

\begin{equation}
    I_{pn}(0)=\frac{qAD_p}{L_p}p_n'(0)=\frac{qAD_nn_{p_0}}{L_n}e^{\frac{V_D}{V_T}-1}
\end{equation}

Ne segue che la corrente totale attraverso la giunzione sia:

\begin{equation}
    I=(\frac{qAD_pp_{n_0}}{L_p}+\frac{qAD_nn_{n_0}}{L_n})(e^{\frac{V_D}{V_T}}-1)\equiv I_0(e^{\frac{V_D}{V_T}}-1)
\end{equation}

Si è così ottenuta la legge di Shockley, da cui si evince che il diodo non è un elemento ohmico.


\subsection{Corrente inversa di saturazione e temperatura}

\begin{equation}
    \begin{aligned}
         & I_0=\frac{qAD_pp_{n_0}}{L_p}+\frac{qAD_nn_{n_0}}{L_n} \rightarrow                    \\
         & I_0\simeq qA \left(\frac{D_p}{L_pN_D}+\frac{D_n}{L_nN_A}\right)n_i^2=                \\
         & =qA \left(\frac{D_p}{L_pN_D}+\frac{D_n}{L_nN_A}\right)A_0T^3e^{\frac{-V_{g_0}}{V_T}}
    \end{aligned}
\end{equation}

Consideriamo due casi:

\begin{itemize}
    \item Germanio: sperimentalmente si rova che le correnti di diffusione varino come 1/T, quindi:

          \begin{equation}
              I_0=K_1T^2e^{\frac{-V_{g_0}}{V_T}}
          \end{equation}


    \item Silicio: La corrente di diffusione è trascurabile rispetto a quella creata per effetto termico
          all'interno della depletion zone. Si trova pertanto che la corrente di saturazione inversa dipende da $_i$ e non da $n_i^2$:

          \begin{equation}
              I_0=K_2T^{1.5}e^{-\frac{V_{g_0}}{\eta V_t}}
          \end{equation}

          in cui $\eta$ è un parametro che indica la non costanza della corrente internamente alla depletion zone (???).
          Nel silicio quindi la corrente generata dalle cariche per effetto termico internamente alla depletion zone domina su quella di diffusiione, ed è pertanto
          data da:

          \begin{equation}
              I_0=I_{gt}=\frac{n_i}{2\tau_0}qAW_{DR}
          \end{equation}

\end{itemize}

In generale, $I_0$ aumenta con T.


\subsection{Caratteristica I-V di un diodo}
Come ricavato precedentemente:

\begin{equation}
    I=I_0(e^{\frac{V_D}{\eta V_T}}-1)
\end{equation}

Un valore poitivo indica una corrente da p ad n, pertanto un diodo polarizzato direttamente.

La corrente si annulla per $V_D=0$.

Chiaramente vi sono delle dipendenze implicite dalla temperatura in V$_T$ e $I_0$. Approssimativamente, la
corrente inversa raddoppia ogni 10 K.

\begin{equation}
    I_0=I_0(e^{\frac{V_D}{\eta V_T}}-1)=I_{01}2^{\frac{T_2-T_1}{10}}
\end{equation}

per alte tensioni:

\begin{equation}
    \begin{aligned}
         & V_D=\eta V_T (\ln{I}-\ln{I_0}) \rightarrow                                                                \\
         & \frac{d V_D}{dT}=\frac{V_D}{T}-\eta V_T(\frac{1}{I_0}\frac{d I_0}{dT})= \frac{V_D-(V_{g_0}+m\eta V_T)}{T}
    \end{aligned}
\end{equation}

L'equazione ottenuta ci dice che avviene un cambiamento di verso della corrente per $V_D=V_{g_0}+m\eta V_T$.
Tuttaia, normalmente, nel range operativo, ciò non avviene.

A temperatura ambiente, si considera per Silicio e Germanio:

\begin{equation}
    \frac{V_D}{dT}\simeq -2.5 mV/K
\end{equation}

Graficamente, sulla caratteristica, un aumento della temperatura risulta in una sua traslazione verso sinistra.

Esiste un valore di tensione di soglia in polarizzazionee diretta a partire dal quale la corrente aumenta rapidamente:
per il Silicio si ha $V_{g_0}=0.6V$, per il Germanio $V_{g_0}=0.2V$.

Plottando la caratteristica in scala logaritmica, si riscontra una peculiarità ad alte tensioni: sebbene ci si aspetti un andamento lineare della corrente nella tensione,
si ha una diminuzione della slope, dovuta alla resistenza ohmica del diodo: la caduta di potenziale interna non è più trascurabile.


\subsection{Retta di carico}

Considerando un n-polo, ossia una componente da cui escano n terminali, quante misure indipendenti è possibile effettuare?

\begin{equation}
    \left\{
    \begin{aligned}
         & \sum_{j=1}^{n}I_j=0  \\
         & V_{jk}+V_{kl}=V_{jl}
    \end{aligned}
    \right.
\end{equation}

quindi si hanno n-1 misure indipendenti per corrente e n-1 per tensione.

Dunque il dispositiv è descrivibile fisicamente mediante n-1 caratteristiche.

Il "punto di lavoro" si ottiene inserendo le n-1 variabili, tramite la connessione del dipositico con l'esterno.

Considerando due bipoli connessi, si ottengono le seguenti condizioni ai terminali:

\begin{equation}
    I_{a_1}=-I_{a_2} \quad V_{a_1b_1}=V_{a_2b_2}
\end{equation}

Le soluzioni di tale sistema costituiscono i punti di lavoro. Graficamente, basta rappresentare sullo stesso grafico
le caratteristiche dei due bipoli, e trovare il punto di intersezione, tenendo conto dei versi discordi della corrente.

La scelta più ovvia per determinare caratteristica e punto di lavoro per un generico bipolo $E_1$ è
un generatore ideale in serie con una resistenza.
La caratteristica del dipolo "sonda" è data semplicemente dalla traslata della caratteristica della resistenza di un valore $V_0$,
ovvero una retta che intervetti l'asse delle tensioni in $V_0$. Tale caratteristica è pertanto utilizzabile come retta di carico per
determinare il punto di lavoro di un dispositivo, calcolandone le intersezioni con la caratteristica di quest'ultimo.

\subsection{Resistenza di un diodo}

Si definisce "resistenza statica R" di un diodo il rapporto tra tensione e corrente.
Essa è data dal reciproco della slope della congiungente tra il punto di lavoro e l'origine.

In regime di piccoli segnali, si definisce "Resistenza dinamica r" di un diodo la slope della caratteristica nel punto di lavoro.

\begin{equation}
    r=\frac{\eta V_T}{I+I_0}
\end{equation}

Essa può essere considerata costante nell'intorno del punto di lavoro.
In regime di grandi segnali, si usa la rappresentazione asintotica della caratteristica:

\begin{itemize}
    \item Per $V<V_\gamma$, la corrente è nulla.
    \item Per $V>V_\gamma$, la corrente è proporzionale alla tensione.
\end{itemize}


\subsection{Capacità di transizione di un diodo}

L'aumento dello spessore della depletion zone può essere considerato un effetto capacitivo del diodo:

\begin{equation}
    C_T \equiv |\frac{dQ}{dV}| \rightarrow i=\frac{dQ}{dt}=C_T\frac{dV}{dt}
\end{equation}

\subsubsection{Giunzione a lega o a fusione}
Nel caso in cui si presente una grande asimmetria di carica nella depletion zone, si parla di giuzione a lega o a fusione.
Spesso si vuole porre un'asimmetria tra il numero di accettori e di donatori.
Se $W_p<<W_n\simeq W$, si ha:

\begin{equation}
    \begin{aligned}
         & N_aW_p=N_dW_n \rightarrow \frac{d^2}{dx^2} V=-\frac{qN_D}{\epsilon} \rightarrow \\
         & \epsilon = -\frac{dV}{dx}=\frac{qN_d}{\epsilon}(x-W) \rightarrow                \\
         & V(x)=-\frac{qN_d}{2\epsilon}(x^2-2Wx)
    \end{aligned}
\end{equation}

Il potenziale in corrispondenza di x=W è pari a $V_j$. In polarizzazione diretta

\begin{equation}
    V_j\frac{qN_D}{2\epsilon}W^2=V_0-V_D \rightarrow W \alpha \sqrt{V_0-V_D}
\end{equation}

Si è verificato che la larghezza della depletion zone diminuisce all'aumentare della tensione.
La capacità è dunque:

\begin{equation}
    C_T=\frac{\epsilon A}{W}=(\frac{\epsilon q}{2})^{\frac{1}{2}}(\frac{N_D}{V_j})^{1/2}
\end{equation}

\subsection{Giunzione per accrescimento}
La variazione della densità di carica nella depletion zone è lineare:

\begin{equation}
    \rho(x)=kx \rightarrow V_j \alpha W^3 \rightarrow C_T \alpha \frac{1}{W}
\end{equation}


\subsubsection{Giunzione VARICAP}

Un diodo VARICAP consente di variare la capacità di transizione mediante applicazione di una tensione di controllo.
In particolare, per polarizzazione diretta, la capacità di transizione diminuisce all'aumentare della tensione applicata,
per via del restringimento della depletion zone. Viceversa per polarizzazione inversa.

\subsection{Capacità di transizione di un diodo}
Per semplicità di trattazione, si considera un diodo in polarizzazione diretta, fortemente drogato p, in modo tale che la corrente
di I attraverso la giunzione sia data unicamente dalle lacune.
Si ha che la carica dei minoritari sia presente solo nella regione n:

\begin{equation}
    Q_e=\int_{0}^{\infty}Aq[p_n(0)-p_{n_0}]e^{-\frac{x}{L_p}}dx=AqL_pp'_n(0)
\end{equation}
La corrente di diffusione delle lacune è data da:

\begin{equation}
    I_{pn}(x)=\frac{qAD_p}{L_p}p'_n(0)e^{-\frac{x}{L_p}}=I_0e^{-\frac{x}{L_p}}
\end{equation}

da cui si ottiene che la corrente del diodo sia proporzionale alla carica Q dei minoritari in eccesso:

\begin{equation}
    I=Q_e\frac{D_p}{L_p^2}=\frac{Q_e}{\tau_p}
\end{equation}

\subsection{Capacità di diffusione}
Si definisce capacità di diffusione la variazione della carica iniettata con la tensione:

\begin{equation}
    C_d=\frac{dQ}{dV}=\tau \frac{dI}{dV}\equiv \tau g\equiv \frac{\tau}{r}= \frac{\tau I}{\eta V_T}
\end{equation}

con g la conduttanza e r la resistenza dinamiche.
Questa trattazione è valida solo per drogaggi fortemente asimmetrici. In caso di drogaggio uniforme, due sono le capacità di diffusione.
Per un diodo polarizzato direttamente, la capacità di diffusione è molto maggiore di quella di transizione. All'opposto per una polarizzazione inversa.

Data una tensione varabile dinamicamente, la carica initettata è dQ' e la corrente è data da:

\begin{equation}
    i=\frac{dQ'}{dt}=C_d' \frac{dV}{dt}
\end{equation}
dfff
Per esempio, per una tensione sinusoidale, si ricavano i seguenti risultati:

\begin{equation}
    \begin{aligned}
         & V=V_D+V_0e^{i\omega t} \rightarrow                                   \\
         & C_D'= \frac{\tau}{2r} \text{A bassa frequenza}                       \\
         & C_D'=\frac{1}{r}\sqrt{\frac{\tau}{2\omega}} \text{Ad alta frequenza}
    \end{aligned}
\end{equation}

\end{document}