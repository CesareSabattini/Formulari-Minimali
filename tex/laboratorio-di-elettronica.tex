\documentclass{article}
\usepackage[utf8]{inputenc}
\usepackage{amsmath}

\title{Laboratorio di Elettronica}
\author{Cesare Sabattini Manginella}
\date{\today}

\begin{document}

\maketitle

\section{Introduzione}
ATTENZIONE: formulario ancora in fase di sviluppo. Alcune informazioni potrebbero essere incomplete o errate.

\section{Teoria delle bande energetiche nei cristalli}
Alla distanza reticolare del cristallo, nell'ambito di questo modello, sono identificbili tre bande:
\begin{itemize}
    \item la banda di conduzione, in cui gli elettroni possono muoversi liberamente;
    \item la banda di valenza, in cui gli elettroni sono legati al nucleo.
    \item la banda proibita, in cui non possono esistere stati elettronici.
\end{itemize}

In termini delle loro proprietà di conducibilità, i materiali possono essere classificati in:
\begin{itemize}
    \item conduttori, in cui la banda di valenza e la banda di conduzione si sovrappongono;
    \item semiconduttori, in cui la banda proibita è facilmente superabile;
    \item isolanti, in cui la banda proibita è molto ampia.
\end{itemize}

Il gap energetico corrispondente alla banda proibita è superabile dagli elettroni per mezzo di apporto energetico termico o di un campo elettrico esterno.
La variabilità della conducibiltà dei materiali è variabile, ma particolarmente alta per i semiconduttori puri, o intrinseci.

L'energia minima per la transizione alla banda di conduzione è ovviamente pari a:

\begin{equation}
    E_g = E_c - E_v
\end{equation}

\subsection*{Mobilità e conducibilità}
\subsubsection{Conduttori}
Il reticolo cristallino di un conduttore è approssimabile ad una successione di ioni positivi e un gas elettroni liberi,
in moto, in assenza di campo elettrico esterno, con velocità termica, con media nulla.
Il modulo della velocità termica è molto maggiore di quella di deriva, in presenza di campo elettrico esterno.
L'applicazione di un campo elettrico si traduce dunque in una rottura dell'isotropia della distribuzione delle velocità, con conseguente deriva degli elettroni
lungo la direzione del campo elettrico.
Il modulo della velocità di deriva è dato da:

\begin{equation}
    v_d = \mu \epsilon
\end{equation}

dove $\mu$ è la mobilità elettronica, e $\epsilon$ è la differenza di potenziale applicata.
La densità di corrente è data, nel caso di isotropia, da:

\begin{equation}
    J = n e v_d= \rho v_d=n q \mu \epsilon= \sigma \epsilon
\end{equation}

dove $n$ è la densità di elettroni, $e$ è la carica elementare, $\rho$ è la densità di carica, $q$ è la carica elettrica, $\sigma$ è la conducibilità elettrica.
Si è così ricavata la legge di Ohm, a livello microscopico:

\begin{equation}
    J = \sigma \epsilon
\end{equation}
\subsubsection{Semiconduttori puri}
Il legame chimico nella shell di valenza è di tipo covalente.
A basse temperature, tale configurazione risulta anloga a quella di un isolante.
Aumentando la temperatura, si ha un aumento della conducibilità, dovuto alla ionizzazione degli atomi,
per via del superamento del gap energetico.
In corrispondenza dell'elettrone liberato, si genera una lacuna, formalizzabile come una carica positiva.

Il processo di creazione di una coppia elettrone-lacuna determina due possibili processi:

\begin{itemize}
    \item L'elettrone di valenza liberato transisce alla banda di conduzione, mentre la lacuna è immediatamente saturata.
          Ciò coincide nella composizione di due moti opposti elettroni-lacune.
    \item L'elettrone ricade nella lacuna, con conseguente annichilazione delle due cariche in termini di conducibilità.
\end{itemize}

In un semiconduttore intrinseco, il numero di lacune è sempre uguale a quello di elettroni liberi.
Analogamente per la densità:

\begin{equation}
    n = p = n_i
\end{equation}

dove $n_i$ è la densità intrinseca, che aumenta con la temperatura:

\begin{equation}
    n_i^2= A_0T^3 e^{-\frac{E_g}{kT}}
\end{equation}

dove $A_0$ è una costante, $E_g$ è il gap energetico a 0K, $k$ è la costante di Boltzmann, $T$ è la temperatura.

In presenza di un campo elettrico esterno, la densità di corrente totale J è pari a:

\begin{equation}
    J = J_n + J_p = q(n \mu_n + p \mu_p) \epsilon= \sigma \epsilon
\end{equation}

SI ricava immediatamente la relazione tra la conducibilità e la temperatura:

\begin{equation}
    \sigma= q(n \mu_n + p \mu_p)= qn_i(\mu_n + \mu_p)= q(\mu_n + \mu_p)A_0T^{3/2} e^{-\frac{E_g}{2kT}}
\end{equation}

Anche la mobilità è funzione della T, a la sua variazione è trascurabile, in termini di conduzione, rispetto all'aumento della densità intrinseca.

\subsection{Distribuzione di energia dei portatori di carica}
\subsubsection{Conduttori}
\begin{equation}
    d n_e= \rho dE
\end{equation}

$\rho$ è la densità di elettroni nell'intervallo dE, ed è data da:

\begin{equation}
    \rho = f(E) N(E)
\end{equation}

con f probabilità di occupazione di uno stato e N densità degli stati nella banda di conduzione.
N(E) è data da:

\begin{equation}
    N(E) = \gamma E^{1/2}
\end{equation}

con $\gamma$ costante.
La densità di probabilità di occupazione è data dalla funzione di probabilità di Fermi-Dirac:

\begin{equation}
    f(E) = \frac{1}{1+e^{\frac{E-E_F}{kT}}}
\end{equation}

Tale funzione indica la distribuzione in energia degli elettroni internamente ad un conduttore, ove con $E_F$ si indica l'energia di Fermi, che rappresenta l'energia massima di un elettrone a T=0K, e per T>0K rappresenta l'energia a cui la probabilità di occupazione è 1/2.
Dato che il numero di elettroni liberi in un conduttore è invariante per cambio di Temperatura, in un ampio range,
possiamo stimare la densità elettronica a T=0K:

\begin{equation}
    n = \int_{0}^{E_F} \rho dE =\int_{0}^{E_F} \gamma E^{\frac{1}{2}}=\frac{2}{3}\gamma E_F^{3/2}
\end{equation}

\subsubsection{Semiconduttori puri}
Analogamente ai conduttori, la densità di elettroni è data da:

\begin{equation}
    d n_e= \rho dE= f(E) N(E)
\end{equation}

Data E$_c$ l'energia minima della banda di conduzione, si può scrivere la densità degli stati come:

\begin{equation}
    N(E) = \gamma (E-E_c)^{1/2} \quad E > E_c
\end{equation}

la probablità di occupazione, invece, analogamente ai conduttori, è data dalla distribuzione di Fermi-Dirac:

\begin{equation}
    f(E) = \frac{1}{1+e^{\frac{E-E_F}{kT}}}
\end{equation}

La probabilità di occupazione tende a 0 per (E-$E_F$) >> kT, e a 1 per (E-$E_F$) << kT.
La densità di elettroni è data da:

\begin{equation}
    n = \int_{E_c}^{\infty} \rho dE = \int_{E_c}^{\infty} f(E) N(E) dE \circ \gamma \int_{E_c}^{\infty} (E-E_c)^{1/2} \frac{1}{e^{\frac{E-E_F}{kT}}} dE= N_c e^{\frac{E_F-E_c}{kT}}
\end{equation}

Dove N$_c$ coincide con la concentrazione intrinseca degli elettroni nel caso in cui l'Energia di Fermi coincida con l'energia minima della banda di conduzione.
Considerando che l'Energia di Fermi è sempre al centro della banda proibita, ciò equivale al caso limite di un conduttore.
Dunque la densità degli elettroni aumenta con la temperatura.

Nel caso delle lacune, invece, si ripete un ragionamento analogo, ottenendo:

\begin{equation}
    p = N_v e^{\frac{E_v-E_F}{kT}}
\end{equation}

\subsubsection{Livello di Fermi}

Per un semiconduttore puro, n=p, ossia:

\begin{equation}
    n_i = N_c e^{\frac{E_F-E_c}{kT}} = N_v e^{\frac{E_v-E_F}{kT}}
\end{equation}

da cui si ricava, considerando le masse delle particelle fittizie rappresentanti le lacune e quelle degli elettroni dello stesso ordine di grandezza, si ottiene
una stima dell'energia di Fermi:

\begin{equation}
    E_F = \frac{E_c + E_v}{2} + \frac{kT}{2} ln(\frac{N_v}{N_c})\circ \frac{E_c+E_v}{2}
\end{equation}

a supporto dell'affermazione precedente.

\subsubsection{Concentrazione intrinseca}

Il prodotto n p è dato da:

\begin{equation}
    n p = n_i^2 = N_c N_v e^{\frac{E_c-E_v}{kT}}= N_c N_v e^{\frac{E_g}{kT}}
\end{equation}

In un semiconduttore intrinseco, n=p, da cui si ricava la legge dell'azione di massa:

\begin{equation}
    np = n_i^2
\end{equation}

che implica che il prodotto delle concentrazioni è costante, data una temperatura.
Si trova in particolare la dipendenza di n$_i$ dalla temperatura:

\begin{equation}
    n_i = A_0 T^{3/2} e^{\frac{-E_g}{2kT}}
\end{equation}

Si è così riottenuta l'equazione (6).

\subsection{Semiconduttori drogati}
Inserendo impurezze chimiche in un semiconduttore, si può ottenere una condizione tale per cui risulti n$\neq$p.
La variabilià introdotta dipende principalmente dal livello di valenza dell'atomo drogante.
Per esempio, un atomo del quinto gruppo della tavola periodica, come il fosforo, introduce un elettrone in più rispetto al silicio, e dunque si parla di drogaggio di tipo n.
Ciò è realizzabile a patto di fornire la necessaria energia di attivazione dei donatori.
Al contrario, un atomo del terzo gruppo, come il boro, introduce una lacuna in più rispetto al silicio, e dunque si parla di drogaggio di tipo p.
n tal caso è necessario fornire l'energia di attivazione degli accettori.
A temperatura ambiente tutti gli atomi droganti sono da considerarsi inonizzati.
Se si aggiungono droganti in concentrazioni maggiori dell'intrinseca, si possono ottenere 3 effetti:

\begin{itemize}
    \item Si ottiene un conduttore in cui i portatori sono di tipo p o n, a seconda del drogaggio;
\end{itemize}

Dato che un semiconduttore drogato è comunque elettricamente neutro, deve valere:

\begin{equation}
    n + N_d = p + N_a
\end{equation}

\subsubsection{Drogaggio di tipo n}

Trascurando la concentrazione intrinseca, per un semiconduttore drogato di tipo n si ha:

\begin{equation}
    n \simeq N_d
\end{equation}

la concentrazione di lacune è pertanto:

\begin{equation}
    p = \frac{n_i^2}{n} = \frac{n_i^2}{N_d}
\end{equation}

\subsubsection{Drogaggio di tipo p}

Analogamente, per un semiconduttore drogato di tipo p si ha:

\begin{equation}
    p \simeq N_a
\end{equation}

la concentrazione di elettroni è pertanto:

\begin{equation}
    n = \frac{n_i^2}{p} = \frac{n_i^2}{N_a}
\end{equation}

\subsubsection{Conducibilità}

Al fine di ottenere una variazione apprezzabile della conducibilità, è necessario che la concentrazione di droganti sia molto piccola.
La conducibilità è data da:

\begin{equation}
    \sigma = q(n \mu_n + p \mu_p)
\end{equation}

La resistività di un semiconduttore drogato è indipendente dalla temperatura, per un ampio range di valori.

\subsubsection{Livello di Fermi nei semiconduttori drogati}

In assenza di impurità, il livello di Fermi coincide con l'energia di Fermi, al centro della banda proibita,
per via della medesima concentrazione di elettroni e lacune.

Se l'impurità è di tipo n, il livello di Fermi si trasla verso la banda di conduzione, in quanto la concentrazione di elettroni è maggiore di quella di lacune.
Ciò, probabilisticamente, implica che la probabilità di occupazione degli stati nella banda di conduzione è maggiore di quella nella banda di valenza,
essendo $E_f$ l'energia alla quale corrisponde una probabilità di occupazione di 1/2.

Trascurando la concentrazione intrinseca degli elettroni, si ha:

\begin{equation}
    N_D= N_c e^{\frac{E_c-E_F}{kT}} \rightarrow E_F = E_c + kT ln(\frac{N_D}{N_c})
\end{equation}

Simmetricamente, per un drogaggio di tipo p, si ha:

\begin{equation}
    N_A= N_v e^{\frac{E_F-E_v}{kT}} \rightarrow E_F = E_v - kT ln(\frac{N_A}{N_v})
\end{equation}

Da notare che all'aumentare della temperatura, il livello di fermi si sposta nuovamente al centro della banda proibita, in accordo con le considerazioni precedenti.

\subsection{Corrente di diffusione}

L'operazione di drogaggio può comportare un gradiente p o n all'interno del semiconduttore.
Ciò è il risultato di un puro processo di diffusione, non elettrostatico. Pertanto, la corrente di
diffusione è esprimibile mediante la legge di Fick:

\begin{equation}
    J_n = q D_n \frac{dn}{dx}
\end{equation}

simmetricamente, per le lacune:

\begin{equation}
    J_p = -q D_p \frac{dp}{dx}
\end{equation}

La relazione tra diffusione e mobilità è data dalla relazione di Einstein:

\begin{equation}
    \frac{D_p}{\mu_p} = \frac{D_n}{\mu_n} = V_T= \frac{\tilde{k}T}{q}
\end{equation}

Segue che la densità di corrente totale per gli elettroni è data da:

\begin{equation}
    J_n= q\mu_n n \epsilon+ qD_n \frac{dn}{dx}
\end{equation}

e simmetricamente per le lacune:

\begin{equation}
    J_p= q\mu_p p \epsilon- qD_p \frac{dp}{dx}
\end{equation}

\subsection{Generazione e ricombinazione di cariche}
Considero un semiconduttore di tipo n, con concentrazioni $p_0$ e $n_0$ all'equilibrio termico, con $n_0 >> p_0$.
Se lo si illumina uniformemente, con fotoni ad energia superiore a quella di gap, si ha la formazione di coppia p-n.
Dopo un transiente, si ha:

\begin{equation}
    \left\{
    \begin{aligned}
         & \Delta n= \overline{n} - n_0 \\
         & \Delta p= \overline{p} - p_0
         & \Delta n = \Delta p
    \end{aligned}
    \right.
\end{equation}

Percentualmente, è maggiore la variazione di lacune rispetto a quella di elettroni.

Analizziamo l'andamento della concentrazione delle lacune dopo aver spento la luce, assumendo $\tau_p$ indipendente dalla concentrazione di lacune.
Detto $\frac{p}{\tau_p}$ è la diminuzione di lacune per ricombinazione, $\frac{p_0}{\tau_p}$ è l'aumento delle lacune per generazione termica.
La variazione del numero di lacune nel tempo è dunque, esprimendo il numero di lacune aggiuntive rispetto a $p_0$ come p':
\begin{equation}
    \frac{dp}{dt} = \frac{p_0-p}{\tau_p} \rightarrow \frac{dp'}{dt}= -\frac{p'}{\tau_p} \rightarrow p'= p_0 e^{-\frac{t}{\tau_p}}
\end{equation}

\subsection{Equazione di continuità}


Considerando un volume infinitesimo di semiconduttore e supponendo la corrente funzione della sola coordinata x, si ha:

\begin{equation}
    \frac{1}{qA}\frac{dI_p}{dx}=\frac{1}{q}\frac{dJ_p}{dx}
\end{equation}

Per quanto ricavato nel paragrfo precedente, la variazione della concentrazione di lacune per via della ricombinazione dovrà eguagliare l'espressione appena ottenuta della densità di corrente di diffusione
per le lacune stesse:

\begin{equation}
    \frac{\partial}{\partial t} p= \frac{p_0-p}{\tau_p}- \frac{1}{q}\frac{\partial J_p}{\partial x}
\end{equation}

\subsection{Low Level Injection}
Considerando l'irraggiamento di una faccia di un semiconduttore drogato n con fotoni ad energia sufficiente, studio l'andamento dei portatori minoritari in condizioni stazionarie, con il fascio acceso, in funzione di x.
Considero l'intensità del fascio debole, in modo tale che il numero di minoritari rimanga inferiore a quello dei maggioritari (???).
La corrente di deriva, in assenza di campo elettrico, è nulla, dunque:

\begin{equation}
    \begin{aligned}
         & J_p=-qD_p\frac{dp}{dx}                                                                            \\
         & \frac{\partial}{\partial t} p= \frac{p_0-p}{\tau_p}- \frac{1}{q}\frac{\partial J_p}{\partial x}=0 \\
         & \frac{d^2}{dx^2} p= \frac{p-p_0}{D_p \tau_p}\equiv \frac{p'}{L_p^2}
    \end{aligned}
\end{equation}

Dove $L_p$ è la lunghezza di diffusione delle lacune.
Questa equazione prende il nome di equazione di Helmholtz.
La sua soluzione, considerando che non deve divergere in x, è:

\begin{equation}
    p(x)= p'(0)e^{-\frac{x}{L_p}}
\end{equation}



\end{document}