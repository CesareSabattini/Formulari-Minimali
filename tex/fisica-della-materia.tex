\documentclass{article}

\usepackage[utf8]{inputenc}
\usepackage{amsmath}
\usepackage{graphicx}

\title{Fisica della Materia}
\author{Sabattini Manginella Cesare}
\date{\today}

\begin{document}

\maketitle

\section{Introduzione}
ATTENZIONE: formulario ancora in fase di sviluppo. Alcune informazioni potrebbero essere incomplete o errate.

\section{Microstati e Macrostati}
Considero uno spazio delle configurazioni equiprobabili in un processo di N casistiche binarie, con probabilità:

\begin{equation}
    P(\alpha) = \frac{1}{2^N}
\end{equation}

Ognuna di tali configurazioni è detta "microstato".
Per arrivare al "macrostato" è necessario considerare una proprietà emergente.
Differenti microstati possono definire lo stesso macrostato (ad esempio l'insieme dei microstati aventi la medesima differenza tra i campioni in configurazioni binarie).
Un macrostato è una caratterizzazione dei microstati.
La probabilità di un macrostato è dato dalla somma delle probabilità dei microstati che lo definiscono, normalizzata.
Dunque, ovviamente, microstati con egual probabilità non determinano egual probabilità dei macrostati.

Considerando lo spazio delle fasi delle particelle di gas, di 6N dimensionale, ogni suo punto rappresenta un microstato, mentre un macrostato è definito da un insieme di microstati con le stesse proprietà macroscopiche.


\end{document}
