\documentclass{article}

\usepackage[utf8]{inputenc}
\usepackage{amsmath}
\usepackage{graphicx}

\title{Astrofisica}
\author{Cesare Sabattini Manginella}
\date{\today}

\begin{document}

\maketitle

\begin{abstract}

    
\end{abstract}

\section{Introduzione}
ATTENZIONE: formulario ancora in fase di sviluppo. Alcune informazioni potrebbero essere incomplete o errate.

\section{Unità di misura}
\subsubsection{Lunghezza}
\begin{itemize}
    \item Raggio solare: $R_{\odot} = 6.96 \times 10^8 m$
    \item Unità astronomica (UA): $1 UA = 1.496 \times 10^{11} m$
    \item Anno luce (AL): $1 AL = 9.461 \times 10^{15} m$
    \item Parsec (pc): $1 pc = 3.086 \times 10^{16} m$. E' la distanza cui 1AU sottende 1''.
    \item Redshift (lo vedremo più avanti).
\end{itemize}
\subsubsection{Massa}
\begin{itemize}
    \item Massa solare: $M_{\odot} = 1.989 \times 10^{30} kg$
    
\end{itemize}
\subsubsection{Tempo}
\begin{itemize}
    \item Anno: $1 yr = 3.156 \times 10^7 s$
\end{itemize}
\subsubsection{Velocità}
\begin{itemize}
    \item $\frac{km}{s}$
\end{itemize}
\subsubsection{Temperatura}
\begin{itemize}
    \item Kelvin
\end{itemize}

\section{Luminosità}
\subsection{Luminosità Bolometrica L}
Quantità di energia emessa da una sorgente luminosa in un intervallo di tempo.
Unità di misura:
\begin{itemize}
    \item Watt
    \item Luminosità solare: $L_{\odot} = 3.83 \times 10^{26} W$
    \item Erg/s
\end{itemize}

\subsection{Luminosità Monocromatica L$_\nu$}
Luminosità bolometrica per unità di frequenza.
Unità di misura:
\begin{itemize}
    \item Watt/Hz
    \item erg/(s Hz)
\end{itemize}
Altresì esprimibile in funzione della lunghezza d'onda $\lambda$ (L$_\lambda$).
Unità di misura:
\begin{itemize}
    \item Watt/$\mu m$
    \item erg/(s $\mu $)
\end{itemize}

La relazione tra le due luminosità è data da:

\begin{equation}
    L_{\nu}d\nu = -L_{\lambda}d\lambda \rightarrow L_{\lambda}= \frac{c}{\lambda^2}L_{\nu}
\end{equation}

inoltre L e L$_\nu$ sono legate da:

\begin{equation}
    L = \int_0^{\inf} L_{\nu}d\nu
\end{equation}


\subsection{Flusso F}
Si definisce Flusso il rapporto tra luminosità e l'area sottesa dalla superficie di emissione isotropa di una sorgente.

\begin{equation}
    F = \frac{L}{4\pi d^2}
\end{equation}

\subsubsection{Flusso monocromatico F$_{\nu}$}

Si definisce Flusso monocromatico il rapporto tra la luminosità monocromatica e l'area sottesa dalla superficie di emissione isotropa di una sorgente.

\begin{equation}
    F_{\nu} = \frac{L_{\nu}}{4\pi d^2}
\end{equation}

Ne segue immediatamente:

\begin{equation}
    F= \int_0^{\inf} F_{\nu}d\nu
\end{equation}

\subsection{Magnitudine}
\subsubsection{Magnitudine apparente m}
La magnitudine è una scala logaritmica per misurare la luminosità di un oggetto celeste.
La magnitudine apparente m è definita come:

\begin{equation}
    m = -2.5 \log_{10}(\frac{F}{F_0})
\end{equation}

dove F$_0$ rappresenta il punto di riferimento per definire un flusso, sovente la stella Vega.

\subsubsection{Magnitudine assoluta M}

La magnitudine assoluta M è definita come la magnitudine apparente che avrebbe un oggetto celeste se fosse posto alla distanza di 10 pc.

\begin{equation}
    M = m + 5 -\log_{10}(d_{pc})
\end{equation}


\subsection{Colore}
Si definisce colore la differenza in termini di megnitudine tra due bande fotometriche.

\begin{equation}
    C = m_1 - m_2
\end{equation}

\subsection{Estinzione}
L'estinzione è la diminuzione del flusso dovuta alla presenza di polvere interstellare.



\section{Astrofisica Osservativa}
Studio dei corpi celesti mediante il loro spettro di emissione.

\subsection{Potere risolutivo}
Si definisce potere risolutivo teorico:

\begin{equation}
    \theta = \frac{\lambda}{D}
\end{equation}
dove $\lambda$ è la lunghezza d'onda e D il diametro della lente o dello specchio.
Dunque si vuole che $\theta$ sia il più piccolo possibile, aumentando la dimensione dello strumento di osservazione.

\subsection{Atmospheric seeing}
Il seeing atmosferico è la distorsione dell'immagine di un oggetto celeste dovuta alla turbolenza dell'aria.
Si ovvia a tale problema mediante l'uso di telescopi adattivi.




\end{document}