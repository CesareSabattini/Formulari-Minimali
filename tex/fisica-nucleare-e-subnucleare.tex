\documentclass{article}

\title{Fisica Nucleare e Subnucleare}
\author{Cesare Sabattini Manginella}
\date{\today}

\begin{document}

\maketitle

\begin{abstract}

\end{abstract}


\section{Introduzione}
ATTENZIONE: formulario ancora in fase di sviluppo. Alcune informazioni potrebbero essere incomplete o errate.

\section{Modelli nucleari}
\subsection{Modello a gas di fermioni}
Per spiegare i termini 4 e 5 del modello a goccia, viene introdotto il modello di nucleo come gas fermionico,
fondato sui seguenti enunciati:
\begin{itemize}
    \item L'interazione di legame è a corto raggio ($\simeq R$).
    \item L'interazione reciproca è determinata unicamente dalle particelle a contatto. Ne segue una densità volumetrica uniforme.
    \item La media delle forze di interazione è zero se non sulla superficie, ove è centrale.
\end{itemize}

riassumibili nel sunto:

\emph{Il nucleo è un insieme di fermioni soggetti ad un potenziale uniforme e tridimensionalmente sferico.}

Ne segue che, quantisticamente,

\begin{itemize}
    \item $\psi(r, t)=0$ per $r\geq R$.
    \item Ciò comporta la quantizzazione di Energia e lunghezza d'onda.
    \item Si forma una distribuzione di livei energetici in cui è possibile inserire solo un protone ed un neutrone ciascuno. Essi sono
    occupati ordinatamente fino al livello di Fermi.
\end{itemize}

Considerando per semplicità una buca di potenziale cubica, la funzione d'onda deve essere del tipo:

\begin{equation}
    \psi(\vec{r},t)= A\sin(\frac{p_x x}{\hbar})\sin(\frac{p_y y}{\hbar})\sin(\frac{p_z z}{\hbar})e^{-iEt/\hbar}
\end{equation}
da cui, imponendo la condizione di annullamento in corrispondenza della superficie, si ricava la quantizzazione della quantità di moto:

\begin{equation}
    p_i=\frac{n_i\pi\hbar}{L}
\end{equation}

Dunque l'energia per nucleone deve valere:

\begin{equation}
    E=\frac{p^2}{2m}=\frac{\hbar^2}{2m}(\frac{n_x^2}{L^2}+\frac{n_y^2}{L^2}+\frac{n_z^2}{L^2})
\end{equation}

Segue l'espressione della funzione d'onda per nucleone:

\begin{equation}
    \psi(\vec{r},t)=\sqrt{\frac{8}{L^3}}\sin(\frac{n_x\pi x}{L})\sin(\frac{n_y\pi y}{L})\sin(\frac{n_z\pi z}{L})e^{-iEt/\hbar}
\end{equation}

Ci si chiede ora quanti stati quantici siano disponibili al di sotto di un certo livello energetico, considerando che sia $n_i$ che E sono funzioni delle quantità di moto $p_i$.
Data la discretizzazione dello spazio degli impulsi, con volume elementare pari a $\frac{\hbar \pi}{L}$, si ha che il numero di stati quantici disponibili è pari al rapporto
tra il volume dell'ottente positivo della sfera di raggio P prefissato e il volume elementare:

\begin{equation}
    N=\frac{1}{8}\frac{4}{3}\pi P^3(\frac{L}{\pi\hbar})^3=\frac{V}{6\pi^2 \hbar^3}P^3
\end{equation}

a cui vanno sottratti gli stati con quantità di moto giacente sui piani coordinati:

\begin{equation}
 n_0= 3\frac{\frac{1}{4}\pi P^2}{(\frac{\pi \hbar}{L})^2}= \frac{S}{8\pi\hbar^2}P^2
\end{equation}
con S superficie del cubo di potenziale.
Infine il numero di stati quantici disponibili è pari a:

\begin{equation}
    N= \frac{V}{6\pi^2 \hbar^3}P^3 - \frac{S}{8\pi\hbar^2}P^2
\end{equation}

contenenti complessivamente 2$n_s$ nucleoni.

\subsection{Modello a Shell}
Il modello a gas di fermioni semplifica eccessivamente le interazioni risultanti.
La ragione dell'irreaizzabilità di tale modello è essenzialmente data dal cosiddetto "Pauli Blocking", che determina
l'impossibilità di isotropia volumetrica nel volume del nucleo.
Infatti, la variazione dello stato energetico di un nucleone, determinata da un urto con un altro,
comporterà una variazione del suo stato quantico solamente se l'energia apportata sia sufficiente.
Ma dal momento che ciò non avviene nella stragrande maggioranza dei casi, l'interazione efficace non è di tipo
stocastico, bensì posizionale, dunque descrivibile da un potenziale.
La forma del potenziale è ricercabile mediante lo studio dei cosiddetti "numeri magici" (A= 8, 20, 28, 50, 82, 126,...), ossia la serie numerica determinata dalle variazioni dell'energia differenzalegame in funzione del numero atomico A.
Essi sono semplicemente determinati dal fatto che un elemento avente A incrementato unitariamente varierà la propria energia di legame in modo sostanziale solamente se il nucleone aggiunto si collochi in un livello energetico superiore.
Il percorso di ricavazione è il seguente: il modello di potenziale centrale coulombiano è inadatto, per cui si ipotizza un potenziale a buca sferica infinita, per poi trovare un "compromesso di rapidità", ossia il potenziale di Saxon-Woods.
Tuttavia, tale potenziale non giustifica l'andamento dei numeri magici sopra il 20, quindi è necessaria una correzione di tipo spin-orbitale, analoga a quella dell'elettrone, ma più intensa.

\subsubsection{Potenziale a buca sferica infinita}


\end{document}