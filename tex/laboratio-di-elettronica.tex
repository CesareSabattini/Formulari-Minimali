\documentclass{article}

\title{Laboratorio di Elettronica}
\author{Cesar}
\date{\today}

\begin{document}

\maketitle

\section{Introduzione}

\section{Teoria delle bande energetiche nei cristalli}
Alla distanza reticolare del cristallo, nell'ambito di questo modello, sono identificbili tre bande:
\begin{itemize}
    \item la banda di conduzione, in cui gli elettroni possono muoversi liberamente;
    \item la banda di valenza, in cui gli elettroni sono legati al nucleo.
    \item la banda proibita, in cui non possono esistere stati elettronici.
\end{itemize}

In termini delle loro proprietà di conducibilità, i materiali possono essere classificati in:
\begin{itemize}
    \item conduttori, in cui la banda di valenza e la banda di conduzione si sovrappongono;
    \item semiconduttori, in cui la banda proibita è facilmente superabile;
    \item isolanti, in cui la banda proibita è molto ampia.
\end{itemize}

Il gap energetico corrispondente alla banda proibita è superabile dagli elettroni per mezzo di apporto energetico termico o di un campo elettrico esterno.
La variabilità della conducibiltà dei materiali è variabile, ma particolarmente alta per i semiconduttori puri, o intrinseci.

L'energia minima per la transizione alla banda di conduzione è ovviamente pari a:

\begin{equation}
    E_g = E_c - E_v
\end{equation}

\subsection*{Mobilità e conducibilità}
\subsubsection{Conduttori}
Il reticolo cristallino di un conduttore è approssimabile ad una successione di ioni positivi e un gas elettroni liberi,
in moto, in assenza di campo elettrico esterno, con velocità termica, con media nulla.
Il modulo della velocità termica è molto maggiore di quella di deriva, in presenza di campo elettrico esterno.
L'applicazione di un campo elettrico si traduce dunque in una rottura dell'isotropia della distribuzione delle velocità, con conseguente deriva degli elettroni
lungo la direzione del campo elettrico.
Il modulo della velocità di deriva è dato da:

\begin{equation}
    v_d = \mu \epsilon
\end{equation}

dove $\mu$ è la mobilità elettronica, e $\epsilon$ è la differenza di potenziale applicata.
La densità di corrente è data, nel caso di isotropia, da:

\begin{equation}
    J = n e v_d= \rho v_d=n q \mu \epsilon= \sigma \epsilon
\end{equation}

dove $n$ è la densità di elettroni, $e$ è la carica elementare, $\rho$ è la densità di carica, $q$ è la carica elettrica, $\sigma$ è la conducibilità elettrica.
Si è così ricavata la legge di Ohm, a livello microscopico:

\begin{equation}
    J = \sigma \epsilon
\end{equation}
\subsubsection{Semiconduttori puri}
Il legame chimico nella shell di valenza è di tipo covalente.
A basse temperature, tale configurazione risulta anloga a quella di un isolante.
Aumentando la temperatura, si ha un aumento della conducibilità, dovuto alla ionizzazione degli atomi,
per via del superamento del gap energetico.
In corrispondenza dell'elettrone liberato, si genera una lacuna, formalizzabile come una carica positiva.

Il processo di creazione di una coppia elettrone-lacuna determina due possibili processi:

\begin{itemize}
    \item L'elettrone di valenza liberato transisce alla banda di conduzione, mentre la lacuna è immediatamente saturata.
    Ciò coincide nella composizione di due moti opposti elettroni-lacune.
    \item L'elettrone ricade nella lacuna, con conseguente annichilazione delle due cariche in termini di conducibilità.
\end{itemize}

In un semiconduttore intrinseco, il numero di lacune è sempre uguale a quello di elettroni liberi.
Analogamente per la densità:

\begin{equation}
    n = p = n_i
\end{equation}

dove $n_i$ è la densità intrinseca, che aumenta con la temperatura:

\begin{equation}
   n_i^2= A_0T^3 e^{-\frac{E_g}{kT}}
\end{equation}

dove $A_0$ è una costante, $E_g$ è il gap energetico a 0K, $k$ è la costante di Boltzmann, $T$ è la temperatura.

In presenza di un campo elettrico esterno, la densità di corrente totale J è pari a:

\begin{equation}
    J = J_n + J_p = q(n \mu_n + p \mu_p) \epsilon= \sigma \epsilon
\end{equation}

SI ricava immediatamente la relazione tra la conducibilità e la temperatura:

\begin{equation}
    \sigma= q(n \mu_n + p \mu_p)= qn_i(\mu_n + \mu_p)= q(\mu_n + \mu_p)A_0T^{3/2} e^{-\frac{E_g}{2kT}}
\end{equation}

Anche la mobilità è funzione della T, a la sua variazione è trascurabile, in termini di conduzione, rispetto all'aumento della densità intrinseca.

\subsection{Distribuzione di energia dei portatori di carica}
\subsubsection{Conduttori}
\begin{equation}
    d n_e= \rho dE
\end{equation}

$\rho$ è la densità di elettroni nell'intervallo dE, ed è data da:

\begin{equation}
    \rho = f(E) N(E)
\end{equation}

con f probabilità di occupazione di uno stato e N densità degli stati nella banda di conduzione.
N(E) è data da:

\begin{equation}
    N(E) = \gamma E^{1/2}
\end{equation}

con $\gamma$ costante.
La densità di probabilità di occupazione è data dalla funzione di probabilità di Fermi-Dirac:

\begin{equation}
    f(E) = \frac{1}{1+e^{\frac{E-E_F}{kT}}}
\end{equation}

con $E_F$ energia di Fermi, che rappreseenta l'energia massima di un elettrone a T=0K, e per T>0K rappresenta l'energia a cui la probabilità di occupazione è 1/2.
Dato che il numero di elettroni liberi in un conduttore è invariante per cambio di Temperatura, in un ampio range,
possiamo stimare la densità elettronica a T=0K:

\begin{equation}
    n = \int_{0}^{E_F} \rho dE \frac{2}{3}\gamma E_F^{3/2}
\end{equation}

\subsubsection{Semiconduttori puri}


\end{document}