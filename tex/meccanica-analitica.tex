\documentclass{article}
\usepackage{graphicx} % Required for inserting images
\usepackage{amsmath}

\title{Formulario-Meccanica-Analitica}
\author{Cesare Sabattini}
\date{June 2024}

\begin{document}

\maketitle

\section{Meccanica di base}

Si definisce entro di massa di un sistema:

\begin{equation}
    M= \sum_i^N \frac{m_i\cdot \vec{r_i}}{
        M_{tot}
    }
\end{equation}

Le forze agenti sul sistema di punti materiali sono:

\begin{equation}
    \vec{F_i}=\vec{F_i}^{(e)}+\sum\vec{F_i}^{(i)}
\end{equation}

Il terzo principio della dinamica, applicato ad un modello di forza interna di tipo centrale, è:

\begin{equation}
    \vec{F_i}^{(i)}=-\vec{F_j}^{(i)}=f_{ij}(\vec{r_i}-\vec{r_j}), \quad f_{ij}=f_{ji}
\end{equation}

Ciò determina dunque una sommatoria delle forze interne nulla, da cui segue la legge di conservazione della quantità di moto.

\begin{equation}
    \sum \vec{F_e}^{(i)}= \frac{d\vec{P}}{dt}
\end{equation}

Le forze esterne determinano dunque l'accelezione del centro di massa del sistema.
Inoltre, data l'invarianza del momento angolare rispetto al punto di applicazione della forza, si ha:

\begin{equation}
    \sum \vec{M_e}^{(i)}= \frac{d\vec{L}}{dt}
\end{equation}
ovvero, le forze interne non determinano variazioni del momento angolare ( il raggio vettore è parallelo alla forza).



\section{Sistemi ad un grado di libertà}
Per un sistema dinamico ad un grado di libertà, soggetto ad una forza posizionale, risulta possibile esprimere la forza come gradiente di un potenziale.
Dall'equazione di Newton, segue:

\begin{equation}
    \Ddot{x}=f(x)=-\frac{dU}{dx}
\end{equation}

In un sistema naturale ad un grado di libertà, valendo la legge di conservazione dell'energia meccanica,
è possibile ricondurre l'equazione differenziale del secondo ordine di Newton, ad un sistema di equazioni differenziali del primo ordine nel piano delle fasi.

Ne risulta la possibilità di integrare per quadratura per risolvere analiticamente l'equazione del moto:

\begin{equation}
    \int dt = \int \frac{1}{\sqrt{2E-2U}}
\end{equation}
Risolvere un'equazione del moto equivale dunque a trovare le equazioni parametriche della traiettoria nello spazio delle fasi, ovvero i luoghi dei punti tali che l'energia meccanica sia costante.
Le curve risultano simmetriche rispetto all'asse delle ascisse, in quanto risultano invarianti per inversione della velocità.



\subsection{Punti di equilibrio}
Un punto di equilibrio è tale che la forza posizionale sia nulla, ovvero:

\begin{equation}
    \nabla U(x_0)=\vec{0}
\end{equation}
ed è stabile se:

\begin{equation}
    \mathcal{H}''_U(x_0) \quad \text{è definita o semidefinita positiva}
\end{equation}

In caso contrario, il punto di equilibrio è instabile.

Rigorosamente, andrebbe studiata la varietà delle configurazioni, in quanto un'hessiana semidefinita potrebbe indicare un punto di sella.


E' dunque logico che un punto di equilibrio nel piano delle fasi sia del tipo:
\begin{equation}
    x_{eq}=(x_0,0)
\end{equation}




Globalità delle equazioni del moto:

\begin{equation}
    \exists \inf(U) \rightarrow |\dot x|=\sqrt{2E} \rightarrow |x-x_0|=\sqrt{2E}t<\infty \forall t \in R
\end{equation}

\subsubsection*{Esercizio teorico:}

Trovare il periodo delle traiettorie sulle separatrici di un sistema conservativo.

\subsubsection*{Esercizio teorico:}

Dimostrare che, data S=area racchiusa da una traiettoria nello spazio delle fasi, vale:

\begin{equation}
    T=\frac{dS}{dE}
\end{equation}

\subsubsection*{Esercizio teorico:}

Dimostrare che:

\begin{equation}
    \lim_{E \rightarrow U(x_{eq})} T= \frac{2\pi}{ \sqrt{U''(x_{eq})}}
\end{equation}

\subsection{Moti in campi centrali}
Un campo centrale è identificato da una forza posizionale che dipende unicamente dalla distanza dal centro del campo.

\begin{equation}
    \vec{F}=-\nabla U(r)= |\vec{F(r)}|\hat{r}
\end{equation}

Introducendo un sistema di coordinate polari, sul piano di orbita, si ha:

\begin{equation}
    \begin{aligned}
         & e_r= \cos{\theta}\hat{x}+\sin{\theta}\hat{y}                \\
         & e_\theta= -\sin{\theta}\hat{x}+\cos{\theta}\hat{y}
         & \vec{r}= r\hat{e_r}                                         \\
         & \vec{\dot{r}}= \dot{r}\hat{e_r}+r\dot{\theta}\hat{e_\theta}
    \end{aligned}
\end{equation}

Si ha inoltre la conservazione di L, momento angolare, e di E, energia meccanica:

\begin{equation}
    |\vec{L}|= |m\vec{r}\times \vec{\dot{r}}|= m r^2 \dot{\theta}= cost
\end{equation}


E' possibile ricondursi ad un grado di libertà.
Potenziale efficace:
\begin{equation}
    V_{eff}=U+\frac{L^2}{2\rho^2}
\end{equation}
La dimostrazione è semplice: basta partire dalla conservazione di L in polari.


\subsubsection*{Esercizio teorico:}
Calcolare la differenza angolare tra perielio e afelio di un'orbita ellittica.

\subsection*{Leggi di Keplero}
\begin{itemize}
    \item Le orbite dei pianeti sono ellittiche, con il sole in uno dei fuochi.

          \textit{Dimostrazione:} Si considera l'espressione del potenziale gravitazionale e si integra l'espressione  dell'angolo $\theta(r)$.
          Ci si riconduce dunque all'equazione di una conica parametrizzata, del tipo:

          \begin{equation}
              r(\theta)= \frac{p}{1+e\cos{\theta}}
          \end{equation}

          Le relazioni che legano p ed e ai semiassi sono:

          \begin{equation}
              a= \frac{p}{1-e^2}, \quad b= \sqrt{a^2(1-e^2)}
          \end{equation}
    \item Il raggio vettore che congiunge il pianeta al sole descrive aree uguali in tempi uguali.

          \textit{Dimostrazione:} Si approssima l'area al primo ordine in theta,

          considerando r costante, e si ottiene:

          \begin{equation}
              \frac{dA}{dt}= \frac{L}{2}
          \end{equation}

    \item Il quadrato del periodo di rivoluzione di un pianeta è proporzionale al cubo del semiasse maggiore dell'orbita.

          \textit{Dimostrazione:} Si tratta sostanzialmente di combinare i risultati ottenuti dalle prime due leggi: si esprime l'area in funzione del tempo e i semiassi in funzione dei parametri orbitali.
          Si giunge a:

          \begin{equation}
              T= \frac{2\pi k}{\sqrt{\frac{k}{a}^3}}
          \end{equation}
\end{itemize}



\section{Meccanica Lagrangiana}

\textit{Definizione: Funzionale}

$\Phi$ è un funzionale $\leftrightarrow$ $\psi$ $\in$ $\mathcal{L^*}$, cioè al duale dello spazio delle curve.


\textit{Definizione: Estremale di $\Phi$}

\begin{equation}
    \gamma \in \mathcal{L*} \text{ è estremale per } \Phi(x,\dot{x},t) \leftrightarrow \frac{d}{dt} \frac{\partial \Phi }{\partial \dot{x}} - \frac{\partial \Phi }{\partial x} = 0
\end{equation}
con $\gamma$ parametrizzazione del luogo dei punti $(x,\dot{x})$ nello spazio delle fasi.

\subsection{Principio di Minima Azione di Hamilton}

Le leggi orarie dei sistemi dinamici coincidono con gli estremali del funzionale:

\begin{equation}
    \Phi = \int_{t_0}^{t} L dt , L=T-U
\end{equation}

Ciò vale per ogni sistema di coordinate ($q, \dot q$) scelto, in quanto la curva esiste indipendentemente dalla sua descrizione analitica.

Segue il sistema di equazioni di Eulero-Lagrange nelle coordinate generalizzate:
\begin{equation}
    \frac{d}{dt}\nabla_{\dot{q}}L= \nabla_{q}L
\end{equation}

Si definisce impulso generalizzato:

\begin{equation}
    p_i= \frac{\partial L}{\partial \dot{q_i}}
\end{equation}

Il tutto vale anche per un potenziale generalizzato.

\section{Elementi di Geometria Differenziale}

\subsection{Carta}
Date U, V varietà,  si definisce carta un'applicazione $\phi$: $\phi$: U $\xrightarrow[su]{1-1}$ V , $\phi^{-1}$: V $\xrightarrow[su]{1-1}$ U, con $\phi$, $\phi^{-1}$ $\in$$ \mathcal{C}^0$.

            \subsection{Atlante}
            Una collezione di carte $V_\alpha$ si dice atlante per la varietà M se:
            \begin{itemize}
                \item $U V_\alpha$= M.
                \item se $V_i \cap V_j != \emptyset \rightarrow$ $V_i$ è compatibile con $V_j$.
            \end{itemize}
            Al fine della parametrizzazione di una varietà, per non perdere la biettività della carta, è necessario un atlante.

            \section{Sistemi vincolati}
            Lo spazio delle configurazioni di un sistema con k vincoli, descritto da n coordinate definisce una varietà n-k dimensionale. Si può dunque ragionare intuitivamente per gradi di libertà, ovvero di coordinate necessarie a descrivere la configurazione del sistema.
            Se un vincolo è bilatero e dipende unicamente da q e t, si dice $olonomo$.

            Dunque, ai fini della risoluzione di un sistema soggetto unicamente a vincoli olonomi:

            \begin{itemize}
                \item Individuare lo spazio delle configurazioni M.
                \item Scegliere un sistema di coordinate, funzionale all'esplicitazione del vincolo ed individuazione delle coordinate cicliche.
                \item Esprimere l'energia cinetica come forma quadratia nelle coordinate generalizzate:
                      \begin{equation}
                          T= \sum \mathcal{A}_{ij} \dot q_i \dot q_j
                      \end{equation}
                \item Ricavare U(q) su M.
                \item Costruire L= T- U e trovare le equazioni di Eulero-Lagrange.
            \end{itemize}

            \subsubsection{Esercizio esemplificativo:}
            Ricavare il teorema di Clairaut dallo studio del moto libero di un punto materiale su una superficie di rotazione in $\mathcal{R}^3$.

            \subsubsection{Esercizio esemplificativo:}
            Ricavare equazioni del moto di un punto materiale su una guida circolare in moto rotatorio uniforme intorno ad un asse fisso.

            \subsection{Principio di d'Alembert}
            Nello studio delle equazioni del moto, si vuole ridurre il problema sostituendo all'inclusione delle reazioni vincolari, la presenza di vincoli. Il principio di d'Alembert formalizza tale possibilità.

            \textit{Il lavoro delle reazioni vincolari su spostamenti virtuali è nullo, ovvero:}

            \begin{equation}
                \sum_i R \delta \vec{r}=0
            \end{equation}

            \textit{Dato un vincolo scleronomo, che individua una varietà differenziale M, una legge oraria su M è un estremale condizionato su M del funzionale $\Phi$ se e solo se per esso vale il principio di d'Alembert.}




            Dunque il principio di d'Alembert implica l'estremalità confdizionata alla varietà M definita dal vincolo stesso.

            \subsection{Teorema di Noether}
            Dato un vincolo olonomo, che induce una varietà n-1 dimesionale in $\mathcal{E}^n$, con Lagrangiana L ivi definita, se esiste una famiglia di diffeomorfismi $h^s: M\rightarrow M$, che conservino L, allora il corrispondente sistema ammette una costante del moto, che in coordinate locali vale:

            \begin{equation}
                p_q= \frac{\partial L}{\partial \dot q}\cdot \frac{d h^s}{ds}
            \end{equation}

            \subsection{Trasformata di Legendre}

            \[
                f: A \rightarrow \mathcal{R}, \quad A \subset \mathcal{R}^n, \quad \text{con} \quad \frac{\partial ^2 f}{\partial x_i^2}>0 \quad \text{continua},
            \]
            \[
                \mathcal{L}f = g: B \rightarrow \mathcal{R},
            \]
            \[
                g(\vec{p}) \equiv \max_{\vec{x}} (\vec{p} \cdot \vec{x} - f(\vec{x}))
            \]

            f è l'inviluppo di g.
        $\mathcal{L}$ è involutiva.


            \section{Meccanica Hamiltoniana}

            \subsection{Funzione Hamiltoniana}

            \begin{equation}
                \mathcal{H}(q,p, t)\equiv \mathcal{L}L(q,\dot q, t)= p\dot q- L(q,\dot q, t)
            \end{equation}

            \begin{equation}
                L=T-U \rightarrow H= T+U
            \end{equation}

            \subsection{Equazioni di Hamilton}
            \begin{equation}
                \left\{
                \begin{aligned}
                     & \dot p=-\frac{\partial H}{\partial q} \\
                     & \dot q= \frac{\partial H}{\partial p}
                \end{aligned}
                \right.
            \end{equation}

            Il sistema di 2n equazioni di Hamilton è equivalente al sistema di n equazioni del secondo ordine di Eulero-Lagrange.

            \subsection{Conservazione di H}

            \begin{equation}
                \frac{d H}{dt}= \frac{\partial H}{\partial t}
            \end{equation}


            \section{Oscillazioni}
            \subsection{Criterio di linearizzazione del sistema lagrangiano}
            Dalla linearizzazione di $U$, si ottiene l'equazione di Lagrange linearizzata, che intorno ad un punto di equilibrio è data dalla somma di due forme quadratiche (che qui i ipotizzano definite positive).

            \begin{equation}
                L_l=T-U= \frac{1}{2}QAQ- \frac{1}{2}Q\mathcal{H}_UQ \quad Q=q-q_{eq}
            \end{equation}

            (La linearizzazione è tale in quanto $\nabla$$U$=0 in q=q\_{eq}, in quanto q\_e è estremo del potenziale).

        da cui si ricavano le equazioni di Eulero Lagrange del sistema linearizzato:

        \begin{equation}
            A\ddot{Q} + BQ = 0, \quad B = \mathcal{H}_U
        \end{equation}

        I moti descritti dall'equazione sono oscillatori, espressi da (se A, B definite positive):

        \begin{equation}
            Q(t)=\sum_{i=1}^n \vec{m_i}[c_1 \cos{\sqrt{\lambda}t}+c_2 \sin{\sqrt{\lambda}t}]
        \end{equation}

        dove $\lambda_i$ sono gli autovalori relativi a $\Vec{m_i}$, autovettori di B rispetto ad A, cioè

        \begin{equation}
            B \vec{m_i}=\lambda_iA\vec{m_i}
        \end{equation}

        I $\lambda_i$ sono pertanto ottenibili dall'equazione:

        \begin{equation}
            det(B-A\lambda)=0
        \end{equation}

        Segue che gli autovettori sono ortogonali secondo la metrica indotta da A.

        \section{Moto relativo}
        \subsection{Cinematica relativa}
        Dato K s.d.r. mobile, k s.d.r. fisso, q $\in$ k e Q $\in$ K, r $\in$ k coordinata di O\_K vale

\begin{equation}
    q(t)=r(t)+Q(t)= r(t)+ BQ(t)= DQ(t)
\end{equation}
con B matrice di rotazione che "allinea" K con k. D è la composizione di una traslazione con una rotazione, tale che

\begin{equation}
    D: K \rightarrow k
\end{equation}

Ne segue che la velocità del punto in k sia

\begin{equation}
    \dot{q}= \dot(r)+\dot{B}Q+ B\dot{Q}
\end{equation}

Si dimostra che:

\begin{equation}
    \exists \omega \in k : w \times q= \dot{B}Q
\end{equation}

dunque

\begin{equation}
    \dot{q}= \dot{r}+\omega \times q+ B\dot{Q}
\end{equation}

dunque il moto è elicoidale intorno all'asse di istantanea rotazione.

\subsection{Dinamica relativa}







\end{document}
