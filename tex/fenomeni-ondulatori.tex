\documentclass{article}
\usepackage{graphicx}
\usepackage{amsmath}

\title{Formulario Fenomeni Ondulatori}
\author{Cesare Sabattini}
\date{June 2024}

\begin{document}

\maketitle

\section{Introduction}


\section{Oscillazioni}
\subsection{Oscillazioni forzate}
Equazioni lineare non omogenea:
\begin{equation}
    m\Ddot{x}+\beta\dot{x}+kx=F_0 e^{i\Omega t}
\end{equation}

Soluzione:
\begin{equation}
    \begin{aligned}
    x = x_{omo} + x_{part} &= A_0 e^{-\frac{t}{2\tau}} \cos\left( \omega t + \phi \right) \\
    &\quad + \frac{\frac{F_0}{m}}{\sqrt{(\omega_0^2 - \Omega^2)^2 + (\gamma \Omega)^2}} 
    \cdot \cos\left( \Omega t - \arctan{\frac{\gamma \Omega}{\omega_0^2 - \Omega^2}} \right) \\
    &\simeq x_{part}
    \end{aligned}
\end{equation}

La soluzione particolare è detta stazionaria.

\subsection{Impedenza}
Misura la tendenza di un mezzo di opporsi ad una variazione di tipo ondulatorio. Intercorre dunque in relazioni di tipo causa-effetto.
\begin{equation}
    Z\equiv\frac{f(t)}{\dot{z}} \rightarrow Z\in \mathcal{C}
\end{equation}

\subsection{Funzione di risposta R}

\begin{equation}
    R(\Omega)=\frac{\gamma^2\Omega^2}{(\omega_0^2-\Omega^2)^2+(\gamma\Omega)^2}
\end{equation}

\subsection{Fattore di qualità Q}
Costante caratteristica di un oscillatore armonico, che misura la rapidità di smorzamento dell'oscillazione. Si definisce come il rapporto tra l'energia dissipata e quella immagazzinata:

\begin{equation}
    Q\equiv \frac{\omega_0}{\gamma}
\end{equation}

\subsection{Ampiezza elastica e ampiezza assorbitiva}

\begin{equation}
\left\{
\begin{aligned}
    &A_{el}=|A|\cos(\sigma)= \frac{F_0(\omega_0^2-\Omega^2)}{m((\omega_0-\Omega^2)^2+(\gamma\Omega)^2)} \\
    &A_{ass}=|A|\sin(\sigma)=\frac{F_0(\gamma\Omega)}{m((\omega_0-\Omega^2)^2+(\gamma\Omega)^2)}
\end{aligned}
\right.    
\end{equation}

\section{Fourier}

\subsection{Serie di Fourier}
Data f(t) periodica,
\begin{equation}
    f(x)= a_0+\sum_{n=1}^\infty a_n\cos(n\omega t)+ b_n\sin(n\omega t)= \sum_{n=-\infty}^{+\infty} c_n e^{i\omega t}
\end{equation}

con coefficienti:

\begin{equation}
\left\{
    \begin{aligned}
        & a_0=\frac{1}{T} \int_0^T f(t) dt \\
        & a_n=\frac{2}{T} \int_0^T f(t)\cos(n\omega t) dt \\
        & b_n=\frac{2}{T} \int_0^T f(t)\sin(n\omega t) dt \\
        & c_n = \frac{1}{T}\int_0^T f(t)e^{-i\omega t} dt
    \end{aligned}
    \right.
\end{equation}

\section{Onde su corda}

Si trova l'equazione di d'Alembert per l'ordinata y=$\xi$:
\begin{equation}
    \frac{\partial^2\xi}{\partial x^2}= \frac{\mu}{T}\frac{\partial^2 \xi}{\partial t^2}
\end{equation}
con velocità di fase
\begin{equation}
    v_f=\sqrt{\frac{T}{\mu}}
\end{equation}

Per onde armoniche:
\begin{equation}
    \xi=A\cos(kx-\omega t)
\end{equation}
con numero d'onda
\begin{equation}
    k=\frac{\omega}{v}
\end{equation}

Potenza trasportata (onde progressive):

\begin{equation}
\left\{
\begin{aligned}
   & P=-T\frac{\partial \xi}{\partial x}\frac{\partial \xi}{\partial t}= Z\frac{\partial^2 \xi}{\partial t^2} \\
   & Z=\frac{T}{v}
    \end{aligned}
    \right.
\end{equation}

Densità di energia:

\begin{equation}
    u=u_k+u_p= \frac{1}{2}\mu(\frac{\partial \xi}{\partial t})^2 + \frac{1}{2}T(\frac{\partial \xi}{\partial x})^2= 2u_k=2u_p
\end{equation}

Vale:
\begin{equation}
    P=vu
\end{equation}

Si definisce intensità di un'onda su corda la potenza mediata sul periodo:
\begin{equation}
    I\equiv <P>
\end{equation}



\subsection{Relazione di dispersione}
Si definiscono velocità di fase e velocità di gruppo:

\begin{equation}
\left\{
\begin{aligned}
& v_f=\frac{\omega}{k} \\
& v_g=\frac{d \omega}{d k}
\end{aligned}
\right.
\end{equation}

Se le due velocità differiscono, il mezzo è detto dispersivo.
La relazione di dispersione è:
\begin{equation}
    \omega(k)=k v_f
\end{equation}


\subsection{Riflessione e trasmissione}

\begin{equation}
\left\{
\begin{aligned}
    & A_t=\frac{2Z_1}{Z_1+Z_2}A_i \\
    & A_r=\frac{Z_1-Z_2}{Z_1+Z_2}A_i
\end{aligned}
\right.
\end{equation}

da cui sono direttamente ricavabili le intensità e i coefficienti di trasmissione e riflessione, definiti come:

\begin{equation}
\left\{
\begin{aligned}
    & T=\frac{I_t}{I_i}
    & R=\frac{I_r}{I_i}
\end{aligned}
\right.
\end{equation}

\section{Onde Sonore}
Onde longitudinali di pressione, densità e elongazione.
Si ricava:
\begin{equation}
    v_f=\sqrt{\frac{\gamma p_0}{\rho_0}}
\end{equation}

Si definisce l'intensità come

\begin{equation}
\left\{
\begin{aligned}
    & I=<\frac{P}{\Sigma}>= <-\gamma p_0 \frac{\partial \xi}{\partial x} \frac{\partial \xi}{\partial t}>=<Z\frac{\partial^2 \xi}{\partial t^2}> \\
    & Z=\frac{\gamma p_0}{v}
\end{aligned}
\right.
\end{equation}

\subsection{Interferenza}
Si verifica quando 2 onde hanno la medesima pulsazione, con fase $\phi$:
\begin{equation}
    A_{tot}= \sqrt{A_1^2+A_2^2+2A_1A_2\cos{\phi}}
\end{equation}

\subsection{Battimenti}
Si verificano quando $\omega_1\simeq\omega_2$. Se $A_1=A_2$ si trova
\begin{equation}
    \left\{
    \begin{aligned}
       & \xi(x,t)=2A\sin(\Delta kx-\Delta\omega t)sin(k_0x-\omega_0 t) \\
       & \Delta k= \frac{k_2-k_1}{2} \\
       & \Delta \omega= \frac{\omega_2-\omega_1}{2} \\
       & k_0= \frac{k_1+k_2}{2} \\
       & \omega_0= \frac{\omega_1+\omega_2}{2}
    \end{aligned}
\right.
\end{equation}

\subsection{Effetto Doppler}
\begin{itemize}
    \item S e R fermi:
    \begin{equation}
        \nu_R=\nu_S
    \end{equation}
    \item S in moto, R fermo:
    \begin{equation}
        \nu_R=\frac{\nu_s}{1-\frac{v_S}{v_M}}
    \end{equation}
    \item S ferma, R in moto:
        \begin{equation}
        \nu_R=\nu_s(1+\frac{v_R}{v_M})
    \end{equation}
    \item S ed R in moto:
        \begin{equation}
        \nu_R=\frac{\nu_s(1+\frac{v_R}{v_M})}{1+\frac{v_S}{v_M}}
    \end{equation}
\end{itemize}

\section{Onde nello spazio}
\subsection{Onde scalari nello spazio}
Vale l'equazione di d'Alembert in $\mathcal{R}^3$:
\begin{equation}
    \nabla^2 \xi=\frac{1}{v^2}\frac{\partial^2 \xi}{\partial t^2}
\end{equation}

Onde progressive e regressive sono rispettivamente esprimibili come:
\begin{equation}
    \xi(x,t)=f(x \mp vt)
\end{equation}

Per le onde armoniche vale:
\begin{equation}
    \xi(x,t)=A\cos(k\hat{u}\cdot\Vec{x}-vt)
\end{equation}

In coordinate sferiche vale:
\begin{equation}
    \nabla^2= \frac{1}{r^2}\frac{\partial}{\partial r}(r^2\frac{\partial}{\partial r})+\frac{1}{r^2}\left(\frac{1}{\sin{\theta}}\frac{\partial}{\partial \theta}(\sin{\theta}\frac{\partial}{\partial \theta})+\frac{1}{\sin{\theta}^2}\frac{\partial^2}{\partial\phi^2}\right)
\end{equation}

Una generica onda progressiva in coordinate sferiche è del tipo:

\begin{equation}
\left\{
\begin{aligned}
     & \xi(\Vec{x}, t)= \frac{f(\Vec{k}\cdot \Vec{x}-vt)}{r}Y^m_l(\theta, \phi) \\
     & \nabla^2Y^m_l(\theta,\phi)=-l(l-1)Y^m_l
\end{aligned}
\right.
\end{equation}

\subsection{Onde vettoriali}
Perturbazioni tali per cui:
\begin{equation}
    \nabla^2 \Vec{\xi}=\frac{1}{v^2}\frac{\partial^2 \Vec{\xi}}{\partial t^2}
\end{equation}

\subsubsection{Polarizzazione}
Dato $\hat{u}= \hat{k}$, si parla di polarizzazione:
\begin{itemize}
    \item longitudinale se
\begin{equation}
    \xi_x=\xi_y=0
\end{equation}

 \item lineare se
\begin{equation}
  \frac{\xi_x}{\xi_y}=\frac{A_x}{A_y}
\end{equation}

\item ellittica se
\begin{equation}
    (\frac{\xi_x}{A_x})^2+(\frac{\xi_y}{A_y})^2-2cos{\phi}\frac{\xi_x}{A_x}\frac{\xi_y}{A_y}= sin{\phi}^2
\end{equation}
\end{itemize}

\section{Onde elettromagnetiche}
\subsection{Equazioni di Maxwell}
Nei mezzi:
\begin{equation}
    \left\{\begin{aligned}
        & \nabla \cdot \Vec{E}=\frac{\rho}{\epsilon_0} \\
        & \nabla \cdot \Vec{B}=0 \\
        & \nabla \wedge  \Vec{E}=-\frac{\partial B}{\partial t}\\
        & \nabla \wedge \Vec{B}= \mu_0\Vec{J}+ \mu\epsilon\frac{\partial E}{\partial t}\\
    \end{aligned}
    \right.
\end{equation}
Nel vuoto:
\begin{equation}
    \left\{\begin{aligned}
        & \nabla \cdot \Vec{E}=0 \\
        & \nabla \cdot \Vec{B}=0 \\
        & \nabla \wedge  \Vec{E}=-\frac{\partial B}{\partial t}\\
        & \nabla \wedge \Vec{B}=\mu\epsilon\frac{\partial E}{\partial t}\\
    \end{aligned}
    \right.
\end{equation}

Nel vuoto si trova che i campi soddisfano l'equazione di d'Alembert:
\begin{equation}
\left\{
    \begin{aligned}
        & \nabla^2 \Vec{E}= \mu\epsilon\frac{\partial^2 \Vec{E}}{\partial t^2}\\
         & \nabla^2 \Vec{B}= \mu\epsilon\frac{\partial^2 \Vec{B}}{\partial t^2}
    \end{aligned}
    \right.
\end{equation}

Supponendo $\Vec{E}$ e $\Vec{B}$ onde armoniche,

\begin{equation}
    \left\{\begin{aligned}
        & i \Vec{k} \cdot \Vec{E}=0 \\
        & i \Vec{k} \cdot \Vec{B}=0 \\
        & i \Vec{k} \wedge  \Vec{E}=i\omega \Vec{B} \\
        &  i \Vec{k} \wedge \Vec{B}= -\mu\epsilon i \omega \Vec{E}
    \end{aligned}
    \right.
\end{equation}

Da cui si evince che:
\begin{itemize}
    \item \begin{equation}
        \Vec{E}\perp \Vec{B}, \Vec{E}\perp \hat{u}, \Vec{B} \perp \hat{u}
    \end{equation}

    \item Le relazioni tra i moduli dei campi. Da
    \begin{equation}
        \Vec{E}=\mu v \Vec{H}
    \end{equation}
    si ha
    \begin{equation}
        Z= \mu v
    \end{equation}
    
\end{itemize}

\subsection{Energia, vettore di Poynting e pressione di radiazione}
Conservazione dell'energia delle onde elettromagnetiche:
\begin{equation}
    \frac{\partial}{\partial t}\left( \frac{1}{2}\epsilon E^2+ \frac{1}{2\mu}B^2 \right)+\nabla\cdot\left( \frac{\Vec{E}\wedge \Vec{B}}{\mu}\right)=0
\end{equation}
Si definisce l'intensità come l'energia media che attraversa una sezione S ortogonale alla direzione di propagazione, nell'unità di tempo, nell'area:
\begin{equation}
I=<\Vec{S}>
\end{equation}
Si trova che l'onda esercita una pressione pari a:

\begin{equation}
    p=\frac{|\Vec{S}|}{c}
\end{equation}

\subsection{Propagazione in un dielettrico}
Si trova che l'indice di rifrazione di un mezzo è pari a:

\begin{equation}
    \left\{
    \begin{aligned}
    & n=n_r-in_i \\
        & n_r= 1+ \frac{n_a e^2}{2 m \epsilon_0}\frac{\omega_0^2-\Omega^2}{
        (\omega_0^2-\Omega^2)^2+(\gamma\omega)^2
        } \\
        &  n_i= 1+ \frac{n_a e^2}{2 m \epsilon_0}\frac{\gamma\omega}{
        (\omega_0^2-\Omega^2)^2+(\gamma\omega)^2
        } \\
    \end{aligned}
    \right.
\end{equation}

Segue che il numero d'onda k sarà:
\begin{equation}
    k= k_r-ik_i
\end{equation}

da cui si ottiene la Legge di Lambert

\begin{equation}
    I(z)= I_0e^{-2k_iz}
\end{equation}

\subsection{Propagazione nei metalli}
\begin{equation}
    k=\frac{\omega}{v}\sqrt{1+i\frac{\mu\sigma v^2}{\omega}}=k_r+ik_i
\end{equation}

Anche qui vale pertanto la Legge di Lambert.

\subsection{Polarizzatori}
Dato $\phi$ angolo tra la direzione di polarizzazione dell'onda e l'asse del polarizzatore, vale la legge di Malus:

\begin{equation}
    I(\theta)=I_0\cos{\theta}^2
\end{equation}

Dato un materiale anisotropo, si può ottenere una condizione di polarizzazione circolare se:
\begin{equation}
    \frac{\omega d n_x}{c} -  \frac{\omega d n_y}{c}= \frac{\pi}{2}
\end{equation}


\subsection{Trasmissione e riflessione}
\subsubsection{E parallelo al piano di incidenza}
\begin{equation}
    \left\{
    \begin{aligned}
        & E_t= \frac{2Z_2}{Z_1+Z_2}E_i \\
        & E_r= \frac{Z_2-Z_1}{Z_1+Z_2}E_i\\
    \end{aligned}
    \right.
\end{equation}
da cuoi sono ricavabili i coefficienti di trasmissione e riflessione $I_t$ e $I_r$.

\subsubsection{$\Vec{E}\perp\Vec{\Sigma}$ }
\begin{equation}
    \left\{
    \begin{aligned}
        & E_t= \frac{2n_1\cos{\theta_i}}{n_1\cos{\theta_r}+n_2\cos{\theta_i}}E_i \\
        & E_r= \frac{n_1\cos{\theta_r}-n_2\cos{\theta_i}}{n_1\cos{\theta_r}+n_2\cos{\theta_i}} E_i\\
    \end{aligned}
    \right.
\end{equation}

\subsubsection{E ortogonale al piano di incidenza}
\begin{equation}
    \left\{
    \begin{aligned}
        & E_t= \frac{2n_1\cos{\theta_i}}{n_1\cos{\theta_i}+n_2\cos{\theta_r}}E_i \\
        & E_r= \frac{n_1\cos{\theta_i}-n_2\cos{\theta_r}}{n_1\cos{\theta_i}+n_2\cos{\theta_r}} E_i\\
    \end{aligned}
    \right.
\end{equation}

Si definisce l'angolo di Brewster come l'angolo in corrispondenza del quale la luce risulta completamente polarizzata:
\begin{equation}
    \theta_b=\arctan(\frac{n_2}{n_1})
\end{equation}

\subsection{Interferenza}
\subsubsection{Interferenza da dispositivo di Young}
\begin{equation}
    I(\theta)=4I_0\cos^2\left(\frac{\pi d}{\lambda}\sin{\theta} \right)
\end{equation}

\subsubsection{Interferenza da N fenditure}
\begin{equation}
    I(\theta)=I_0\frac{\sin^2\left(\frac{N\pi d}{\lambda}\right) \sin{\theta}}{
    \sin^2\left(\frac{\pi d}{\lambda}\right) \sin{\theta}
    }
\end{equation}

\begin{equation}
    I=N^2I_0
\end{equation}

\subsubsection{Reticoli}
Dispersione D:
\begin{equation}
\mathcal{D}\equiv \frac{d\theta}{d \lambda}= \frac{n}{dcos(\theta_{max})}
\end{equation}
Potere risolutivo R:

\begin{equation}
\mathcal{R}\equiv \frac{\lambda}{d \lambda}= \frac{nN}{cos(\theta)}\simeq nN
\end{equation}


\subsection{Diffrazione}
\begin{equation}
    I(\theta)=I_0\frac{\sin^2\left(\frac{\pi a}{\lambda} \right)\sin{\theta}}{\left(\frac{\pi a}{\lambda} \sin{\theta}\right)^2}
\end{equation}

Si definisce il "Potere separatore" come:
\begin{equation}
    \gamma_R=1,22\frac{\lambda}{D}
\end{equation}


\section{Ottica}
\subsection{Specchi}
\begin{equation}
\left\{
\begin{aligned}
      & \frac{1}{p}-\frac{1}{q}=-\frac{2}{R}=-\frac{1}{f} \\
      & m=\frac{q}{p}
\end{aligned}
\right.
\end{equation}

\subsection{Diottri}
\begin{equation}
\left\{
\begin{aligned}
      & \frac{n_1}{p}+\frac{n_2}{q}=\frac{n_2-n_1}{R} \\
      & \frac{f_1}{p}+\frac{f_2}{q}=1\\
      & m=\frac{n_1q}{n_2p}
\end{aligned}
\right.
\end{equation}

\subsection{Lenti sottili}
\begin{equation}
\left\{
\begin{aligned}
      & \frac{1}{p}+\frac{1}{q}= (\frac{n_2}{n_1}-n_1)(\frac{1}{R_1}-\frac{1}{R_2})=\frac{1}{f} \\
      & m=\frac{q}{p}
\end{aligned}
\right.
\end{equation}

\end{document}