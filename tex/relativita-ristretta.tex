\documentclass{article}
\usepackage{graphicx}
\usepackage{amsmath}

\title{
    Relatività Ristretta
}
\author{Cesare Sabattini}
\date{July 2024}

\begin{document}

\maketitle

\section{
  Relatività Galileiana
 }

\subsection{Principi della Relatività Galileiana}

\begin{enumerate}
    \item Le leggi della meccanica sono le stesse in tutti i sistemi di riferimento inerziali.
    \item Il tempo è assoluto.
    \item Lo spazio è assoluto.
\end{enumerate}

Ne seguono geometricamente le trasformazioni di Galileo.

\subsection{Trasformazioni di Galileo}
Per coppie di sistemi di riferimento inerziali O e O' valgono le seguenti leggi di trasformazione:

\subsubsection{
    Trasformazioni delle coordinate
}
\begin{equation}
    \left\{
    \begin{aligned}
        x' & = x - vt \\
        y' & = y      \\
        z' & = z      \\
        t' & = t
    \end{aligned}
    \right.
\end{equation}

\subsubsection{
    Trasformazione della velocità
}
\begin{equation}
    \vec{w'} = \vec{w} - \vec{v}
\end{equation}

\subsubsection{
    Trasformazione delle accelerazioni
}
\begin{equation}
    \vec{a'} = \vec{a}
\end{equation}

\subsubsection{Trasformazioni delle forze}
Dalla rappresentazione della forza come un segmento orientato, dato lo spazio assoluto, si ha:
\begin{equation}
    \vec{F'} = \vec{F}
\end{equation}

\subsubsection{Trasformazione della massa inerziale}
Dal secondo principio della dinamica, invariante per cambio di sistema di riferimento inerziale, si ha:

\begin{equation}
    m' = m
\end{equation}


\subsection{Repporto con l'elettromagnetismo}
Si dimostra di seguito l'incompatibilità tra le trasformazioni di Galileo e le equazioni di Maxwell.

\subsubsection{Forza di Lorentz}
\begin{equation}
    \vec{F} = q(\vec{E} + \vec{v} \times \vec{B})
\end{equation}

da cui emerge una dipendenza della forza dalla velocità, in contrasto con i principi di relatività galileiani.

\subsection*{Equazioni di Maxwell}
Date le equazioni di Maxwell in O
\begin{equation}
    \left\{
    \begin{aligned}
        \nabla \cdot \vec{E}  & = \frac{\rho}{\varepsilon_0}                                              \\
        \nabla \cdot \vec{B}  & = 0                                                                       \\
        \nabla \times \vec{E} & = -\frac{\partial \vec{B}}{\partial t}                                    \\
        \nabla \times \vec{B} & = \mu_0 \vec{J} + \mu_0 \varepsilon_0 \frac{\partial \vec{E}}{\partial t}
    \end{aligned}
    \right.
\end{equation}

le leggi di trasformazione degli operatori differenziali:
\begin{equation}
    \left\{
    \begin{aligned}
        \nabla'                      & = \nabla                                             \\
        \frac{\partial}{\partial t'} & = \frac{\partial}{\partial t} - \vec{v} \cdot \nabla
    \end{aligned}
    \right.
\end{equation}

e le leggi di trasformazione dei campi elettrici e magnetici:

\begin{equation}
    \left\{
    \begin{aligned}
        \vec{E'} & = \vec{E} + \vec{v} \times \vec{B} \\
        \vec{B'} & = \vec{B}
    \end{aligned}
    \right.
\end{equation}


si verifica, supponendo la validità delle equazioni in O', che:
\begin{itemize}
    \item La prima equazione non è né invariante né covariante.
    \item La seconda equazione è invariante.
    \item La terza equazione è covariante.
    \item La quarta equazione non è né invariante né covariante.
\end{itemize}

\subsubsection{Invarianza di c}
L'elettromagnetismo sancisce che:
\begin{equation}
    c = \frac{1}{\sqrt{\varepsilon_0 \mu_0}}
\end{equation}

in completo disaccordo con il principio di relatività galileiana.

Si ipotizzò dunque che le leggi dell'elettromagnetismo dovessero valere unicamente nell'etere e che la velocità della luce fosse relativa ad esso.

\subsection{Esperimento di Michelson-Morley}

\begin{itemize}
    \item Scopo: misurazione della velocità della luce rispetto all'etere,

          confrontando le figure di interferenza originate da due raggi luminosi, dipendentemente dall'orientamento dell'interferometro rispetto al moto della Terra nel mezzo.

    \item Aspettazione: variazione della differenza di fase tra i due raggi luminosi, in base all'orientamento dell'interferometro rispetto al moto della Terra, in base a:
          \begin{equation}
              \Delta \phi = \frac{2L}{\lambda} \frac{v^2}{c^2}
          \end{equation}

          Il valore atteso di $\Delta \phi$ è di circa 0.4 frange.

    \item Esito: nessuna variazione della figura di interferenza, in quanto $\Delta \phi$ è circa nullo.
\end{itemize}


\section{Relatività Ristretta}
L'apparente conflitto tra meccanica e elettromagnetismo si risolse con la teoria della relatività ristretta.

\subsection{Postulati della Relatività Ristretta}

\begin{enumerate}
    \item Le leggi della fisica sono le stesse in tutti i sistemi di riferimento inerziali.
    \item La velocità della luce è c in tutti i sistemi di riferimento inerziali.
\end{enumerate}

\subsection{Trasformazioni di Lorentz}
Si ricercano delle leggi di trasformazione delle coordinate tali che:
\begin{itemize}
    \item x'=0 $\rightarrow$ x=vt, che si verifica se x'=$\alpha$(x-vt).
    \item Per il primo postulato, deve valere x=$\alpha$(x'+vt).
    \item x'=ct' $\rightarrow$ x=ct.
\end{itemize}

Si ricavano così le trasformazioni di Lorentz (da O a O') delle coordinate:
\begin{equation}
    \left\{
    \begin{aligned}
         & x'= \frac{x-vt}{\sqrt{1-\frac{v^2}{c^2}}}             \\
         & t'= \frac{t-\frac{v}{c^2}x}{\sqrt{1-\frac{v^2}{c^2}}} \\
         & y'=y                                                  \\
         & z'=z                                                  \\
    \end{aligned}
    \right.
\end{equation}

e delle velocità:

\begin{equation}
    \left\{
    \begin{aligned}
         & v_x' = \frac{v_x-v}{1-\frac{vv_x}{c^2}}                       \\
         & v_y' = \frac{\sqrt{1-\frac{v^2}{c^2}}v_y}{1-\frac{vv_x}{c^2}} \\
         & v_z' = \frac{\sqrt{1-\frac{v^2}{c^2}}v_z}{1-\frac{vv_x}{c^2}} \\
    \end{aligned}
    \right.
\end{equation}

\subsection{Effetti relativistici}

\subsubsection{Dilatazione dei tempi}

\begin{equation}
    \Delta t = \frac{\Delta t'}{\sqrt{1-\frac{v^2}{c^2}}}
\end{equation}

\subsubsection{Relatività della simultaneità}

\begin{equation}
    \Delta t = \frac{\frac{v}{c^2}\Delta x'}{\sqrt{1-\frac{v^2}{c^2}}}
\end{equation}

\subsubsection{Contrazione delle lunghezze}

\begin{equation}
    L = L' \sqrt{1-\frac{v^2}{c^2}}
\end{equation}

\subsection{Onde elettromagnetiche}
\subsubsection{Trasformazioni del vettore d'onda e frequenza}
\begin{equation}
    \left\{
    \begin{aligned}
         & k_x'= \frac{k_x'-\frac{v}{c^2}\omega'}{\sqrt{1-\frac{v^2}{c^2}}} \\
         & k_y'= k_y'                                                       \\
         & k_z'= k_z'                                                       \\
         & \omega'= \frac{\omega'-vk_x'}{\sqrt{1-\frac{v^2}{c^2}}}
    \end{aligned}
    \right.
\end{equation}

\subsubsection{Aberrazione relativistica della luce}
\begin{equation}
    \tan \theta' = \sqrt{1-\frac{v^2}{c^2}}\frac{\sin \theta}{\cos \theta - \frac{v}{c}}
\end{equation}

\subsubsection{Effetto Doppler relativistico}

\begin{equation}
    \omega' = \omega\frac{1-\frac{v}{c}\cos{\theta}}{\sqrt{1-\frac{v^2}{c^2}}}
\end{equation}


\subsection{Causalità}
Due eventi possono risultare in mutua connessione causale se e solo se

\begin{equation}
    \frac{|x_2-x_1|}{|ct'_1-ct'_2|} \leq 1
\end{equation}

Se ne evince che:

\begin{itemize}
    \item L'ordine causale degli eventi è invariante.
    \item L'ordine degli eventi non causali è relativo.
    \item La causalità è sempre locale.
\end{itemize}

\section{Formalismo covariante}
Alla base del formalismo covariante vi è il

\subsection*{Postulato di Covarianza}
\textit{
    Le leggi della fisica sono covarianti per trasformazioni di Lorentz.
}
\subsection{Quadrivettori controvarianti e covarianti}
In notazione indiciale, si definiscono i quadrivettori come:

\begin{equation}
    x^\mu = (ct, \vec{x})
\end{equation}

ovvero quartetti ordinati che si trasformano secondo le trasformazioni di Lorentz.

Le coordinate di un quadrivettore controvariante sono ottenute per proiezione parallela rispetto agli assi coordinati.

Le componenti di un quadrivettore covariante sono ottenute per proiezione ortogonale rispetto agli assi coordinati.

\subsection{Prodotto scalare}
Dall'invariante spaziotemporale che emerge dalle trasformazioni di Lorentz, si induce una metrica nello spazio di Minkowski:

\begin{equation}
    <x,y> = x^\mu y_\mu = x^0y^0 - \vec{x} \cdot \vec{y}
\end{equation}

che equivale ad introdurre il tensore metrico:

\begin{equation}
    g_{\mu\nu} =
    \begin{pmatrix}
        1 & 0  & 0  & 0  \\
        0 & -1 & 0  & 0  \\
        0 & 0  & -1 & 0  \\
        0 & 0  & 0  & -1
    \end{pmatrix}
\end{equation}

tale che:

\begin{equation}
    <x,y> = x^\mu y_\mu = g_{\mu\nu}x^\mu y^\nu
\end{equation}

\subsection{Trasformazioni di Lorentz per quadrivettori}

I quadrivettori controvarianti si trasformano con L, mentre i covarianti si trasformano con $L^{-1}$.

\begin{equation}
    \left\{
    \begin{aligned}
         & x'^j = L^j_{\ k}x^k                              \\
         & x'_j  = L_j^{\ k}x_k= L^{-1 j}_{\ \  \  \ k} x_k
    \end{aligned}
    \right.
\end{equation}

dove $L^j_{\ k}$ è la matrice di Lorentz:

\begin{equation}
    L^j_{\ k} =
    \begin{pmatrix}
        \frac{1}{\sqrt{1-\frac{v^2}{c^2}}}      & -\frac{\beta}{\sqrt{1-\frac{v^2}{c^2}}} & 0 & 0 \\
        -\frac{\beta}{\sqrt{1-\frac{v^2}{c^2}}} & \frac{1}{\sqrt{1-\frac{v^2}{c^2}}}      & 0 & 0 \\
        0                                       & 0                                       & 1 & 0 \\
        0                                       & 0                                       & 0 & 1
    \end{pmatrix}
\end{equation}

Dato che i tensori di rango superiore risultano da prodotto dialico tra

quadrivettori, è immediato dedurne le regole di trasformazione.



\section{Meccanica relativistica}

\subsection{Relazioni principali}

\subsubsection{Secondo principio della dinamica relativistica}
Incorporando i vettori tridimensionali nei quadrivettori nello spazio di Minkowski,
si definiscono la quadriposizione, la quadrivelocità, la quadriaccelerazione e la quadriforza:

\begin{equation}
    \left\{
    \begin{aligned}
         & x^\mu = (ct, \vec{x})                          \\
         & u^\mu = \frac{dx^\mu}{d\tau}                   \\
         & a^\mu = \frac{du^\mu}{d\tau}                   \\
         & F^\mu = m_0\frac{d}{d\tau}\frac{d}{d\tau}x^\mu
    \end{aligned}
    \right.
\end{equation}

in cui $\tau$ è il tempo proprio, ovvero relativo al sistema comovente, istantaneamente solidale con il corpo, dunque non accelerato.
Si trova dunque che:

\begin{equation}
    F_j=m_0\frac{d}{d\tau}\frac{d}{d\tau}x_j=m_0\frac{d}{\sqrt{1-\frac{v^2}{c^2}}dt}(\frac{c}{\sqrt{1-\frac{v^2}{c^2}}},-\frac{\vec{v}}{\sqrt{1-\frac{v^2}{c^2}}})
\end{equation}

\subsubsection{Equazioni cardine della dinamica relativistica}

Si definiscono quadrimpulso ed energia totale come:

\begin{equation}
    \left\{
    \begin{aligned}
         & P_j=m_0V_j                                    \\
         & E_t^= \frac{m_0c^2}{\sqrt{1-\frac{v^2}{c^2}}}
    \end{aligned}
    \right.
\end{equation}

Dalla 2.8 seguono le equazioni cardine della dinamica relativistica:

\begin{equation}
    \left\{
    \begin{aligned}
         & \vec{f}\cdot\vec{v}=\frac{d}{dt}\frac{m_0c^2}{\sqrt{1-\frac{v^2}{c^2}}}= \frac{d E_T}{dt}                                  \\
         & \vec{f}=\frac{d}{dt}\frac{m_0\vec{v}}{\sqrt{1-\frac{v^2}{c^2}}}=\frac{d \vec{p}}{dt}=\frac{d (\frac{E_t\vec{v}}{c^2})}{dt}
    \end{aligned}
    \right.
\end{equation}

in cui

\begin{equation}
    \vec{f}=\sqrt{1-\frac{v^2}{c^2}}\vec{F}
\end{equation}

\subsubsection{Relazione Energia-Impulso}

\begin{equation}
    E_t^2=c^2p^2+m_0^2c^4
\end{equation}

\subsubsection*{Centro di massa}
Nel sistema di riferimento del centro di massa:
\begin{equation}
    \begin{aligned}
         & \vec{v}_{cm}=\frac{\sum E_i\vec{v}_i}{\sum E_i} \\
         & M= \sum(m_j+\frac{T_j}{c^2}+\frac{V_j^2}{c^2})
    \end{aligned}
\end{equation}

\subsection{Fotoni}
Particelle di massa nulla, con una quantità di moto.
La relazione energia-impulso per i fotoni è:

\begin{equation}
    E_T=cp
\end{equation}

Vale la relazione di Einstein-De Broglie:

\begin{equation}
    P_j=\hbar K_j=\hbar (\frac{\omega}{c},-\vec{k})
\end{equation}

\subsubsection{Effetto Compton}
Urto di un fotone con una particella ferma, con variazione di lunghezza d'onda:

\begin{equation}
    \Delta \lambda = \lambda' - \lambda = \frac{h}{m_0c}(1-\cos \theta)
\end{equation}

\subsubsection{Decadimento di una particella}

Basta utilizzare la conservazione del quadrimpulso

e la relazione energia-impulso.

\section{Elettromagnetismo}

\subsection{Conservazione della carica}
Definita la quadricorrente:
\begin{equation}
    J_j=(\rho c, \vec{j})
\end{equation}

Si passa alla formulazione covariante della conservazione della carica:

\begin{equation}
    \partial _k J^k=0
\end{equation}

Si definisce inoltre la quadricorrente della carica puntiforme come:

\begin{equation}
    J^k=qV_j
\end{equation}
con $V_j$ quadrivelocità.

\subsection*{Quadriforza di Lorentz}

Si definisce la densità di quadriforza di Lorentz come:

\begin{equation}
    \vec{f}=\frac{d \vec{F}}{dV}\sqrt{1-\frac{v^2}{c^2}}
\end{equation}

Esprimendo il tensore elettromagnetico misto come:

\begin{equation}
    F^j_k=
    \begin{pmatrix}
        0     & E_1/c & E_2/c & E_3/c \\
        E_1/c & 0     & B_3   & -B_2  \\
        E_2/c & -B_3  & 0     & B_1   \\
        E_3/c & B_2   & -B_1  & 0
    \end{pmatrix}
\end{equation}

si può scrivere la quadriforza di Lorentz come:

\begin{equation}
    f^j=F^j_k J^k
\end{equation}

\subsection{Equazioni di Maxwell}

La prima e la quarta equazione di Maxwell secondo il formalismo covariante sono:
\begin{equation}
    \begin{aligned}
        \partial_j F^{jk}=\mu_0 J^k
    \end{aligned}
\end{equation}

La seconda e la terza equazione di Maxwell secondo il formalismo covariante sono:
\begin{equation}
    \begin{aligned}
        \epsilon^{jklm}\partial_k F_{lm}=0 \leftrightarrow \partial_k \tilde F^{jk}=0
    \end{aligned}
\end{equation}

con

\begin{equation}
    \tilde F^{jk}=\frac{1}{2}\epsilon^{jklm}F_{lm}
\end{equation}

\subsection{Trasformazioni dei campi elettromagnetici}

Tenendo conto che $F^j_k$ è un tensore misto di rango 2, si ha che:

\begin{equation}
    F'^{jk}=L^j_{\ i}L^k_{\ l}F^{il}
\end{equation}

si trovano le trasformazioni dei campi elettrici e magnetici:

\begin{equation}
    \begin{aligned}
         & E_1'=E_1                                       \\
         & E_2'=\frac{E_2-vB_3}{\sqrt{1-\frac{v^2}{c^2}}} \\
         & E_3'=\frac{E_3+vB_2}{\sqrt{1-\frac{v^2}{c^2}}} \\
         & B_1'=B_1                                       \\
         & B_2'=\frac{B_2+vE_3}{\sqrt{1-\frac{v^2}{c^2}}} \\
         & B_3'=\frac{B_3-vE_2}{\sqrt{1-\frac{v^2}{c^2}}}
    \end{aligned}
\end{equation}

\subsection{Tensore Energia-Impulso}
Per giungere alle equazioni di conservazione per le onde elettromagnetiche, si definisce il tensore
energia-impulso $T^{lk}$ come:

\begin{equation}
    T^{lk}=-\frac{1}{\mu_0}F^{lm}F^k_{\ m}+\frac{1}{4\mu_0}g^{lk}F_{km}F^{km}
\end{equation}

cosicché:

\begin{equation}
    f^k=\partial_l T^{lk}
\end{equation}

definita la densità di energia elettromagnetica come:

\begin{equation}
    \epsilon=\frac{1}{2\mu_0}(E^2/c^2+B^2)
\end{equation}

il vettore di Poynting come:

\begin{equation}
    \vec{S}=\frac{\vec{E}\wedge\vec{B}}{\mu_0}
\end{equation}

e il tensore degli sforzi di Maxwell come:

\begin{equation}
    \sigma_{jk}=\frac{1}{\mu_0}(E_jE_k/c^2+B_jB_k)-\delta_{jk}\epsilon
\end{equation}

si ottiene il tensore energia-impulso come:

\begin{equation}
    T^{lk}=
    \begin{pmatrix}
        \epsilon & S_1/c       & S_2/c       & S_3/c       \\
        S_1/c    & \sigma_{11} & \sigma_{12} & \sigma_{13} \\
        S_2/c    & \sigma_{21} & \sigma_{22} & \sigma_{23} \\
        S_3/c    & \sigma_{31} & \sigma_{32} & \sigma_{33}
    \end{pmatrix}
\end{equation}



\end{document}
