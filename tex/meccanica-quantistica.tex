\documentclass{report}

\usepackage[utf8]{inputenc}
\usepackage{amsmath}
\usepackage{amsfonts}
\usepackage{amssymb}
\usepackage{graphicx}
\usepackage{hyperref}
\usepackage{tcolorbox}

\hypersetup{
    colorlinks=true,      
    linkcolor=black,        
    citecolor=blue,        
    urlcolor=blue,         
    pdfborder={0 0 0},     
}

\title{Meccanica Quantistica}
\author{Cesare Sabattini Manginella}
\date{\today}

\begin{document}

\maketitle

\tableofcontents
\newpage

\chapter{Introduzione}
ATTENZIONE: formulario ancora in fase di sviluppo. Alcune informazioni potrebbero essere incomplete o errate.
Inoltre, è in corso la riscrittura in inglese.



\chapter{Development of the formalism}

\subsection{Bra an Ket}

Given an infinite dimensional curvature-less and tortion-less manifold, one defines a ket $\langle A|$ as an element of $V$, while a bra $|A\rangle $ is a linear functional on $V$.
The notation embodies the theorem:

\begin{tcolorbox}[colframe=gray!50, colback=gray!10, coltitle=black, title=Fisher Reisz Representation Theorem]
  Being  $H$ over $\mathcal{C}$,

  \begin{equation}
    \forall \phi \in H^* \exists x \in H : \forall y \in H \phi(x)=\langle x,y\rangle
  \end{equation}
\end{tcolorbox}

meaning that the biunivocal correspondence between a ket and a bra allows for the covariant-contravariant representation of the same vector.
In quantum mechanics, as it'll be shown in the following, the ket is associated to a state, while the bra is associated to the dual state, or the complex conjugate of the state:
this implies, due to the linearity of the Hilbert space, that the most general state can be expressed as a linear combination of the basis vectors of the Hilbert space, meaning
the ket that diagonalize a certain operator.

\subsection{Operatorial spaces}

Given a linear space, an operator is defined as an endomorphism on the space itself.
Extending a non-commutative associative algebra on sets of operators, operatorial spaces can be defined.

Operators in quantum mechanics represent dynamical variables: if they are self-adjoint, they represent observables.

Thinking in terms of covariant and contravariant indices, it appears logical that the product of a ket by a bra originates an observable, acting on bras and kets:
this is fundamental to understand the possibility to diagonalize an Hilbert space in terms of the eigenstates of a set of operators.

Since, in general, operators do not commute, physical measurements can permanently alter the state of the system, as the system collapses in the eigenstate of the measured operator.

\section{Observables}

\begin{tcolorbox}[colframe=gray!90, colback=blue!5, coltitle=white, title=\textbf{Definition: Observable}, fonttitle=\large\bfseries]
  An observable $\xi$ is a self-adjoint linear operator on a Hilbert space, whose eigenstates form a complete system, meaning that:

  \begin{equation}
    \text{$|P\rangle$ = $\sum_{0}^{v} \langle P|\xi_i\rangle |\xi_i\rangle  + \int_{0}^{v} \langle P|\xi'\rangle |\xi'\rangle d\xi'$  $\forall |P\rangle \in H$}
  \end{equation}
\end{tcolorbox}

The request of self-adjointness is necessary to ensure that the eigenvalues of the operator are real, meaning they do represent a unique physical quantity.
Indeed, their complexity would imply the possibility of multiple measurements, each potentially perturbing the following one.

Therefore, if the system is in an eigenstate of an observable, the outcome of a measurement of that observable will be the eigenvalue associated to that eigenstate.

Moreover, if an operator is an observable, the Hilbert space can be diagonalized in terms of the eigenstates of the operator, and a general state,
for the principle of superposition, can be expressed as a linear combination of the eigenstates of the operator.

The non commutative algebra of the operatorial space leads to the definition of the commutator:

\begin{tcolorbox}[colframe=gray!90, colback=blue!5, coltitle=white, sharp corners, title=\textbf{Definition: Commutator}, fonttitle=\large\bfseries]
  Given an operatorial space, the commutator of two observables $\xi$ and $\eta$ is defined as:

  \begin{equation}
    [\xi,\eta]=\xi\eta-\eta\xi
  \end{equation}
\end{tcolorbox}


Physically, the commutator gives the possibility to measure two dynamical variables simultaneously.
Matematically, it implies the possibility to diagonalize simoltaneously the two observables.

\subsection{Functions of observables}

\begin{tcolorbox}[colframe=gray!90, colback=blue!5, coltitle=white, title= \textbf{Definition}: Function of Observables, fonttitle=\large]
  A function of an observable $\xi$ is defined as:
  \begin{equation}
    \text{$(f(\xi): L(H) \rightarrow L(H))$ : $f(\xi)|\xi'\rangle = f(\xi')|\xi'\rangle $ , $\forall |\xi'\rangle $}
  \end{equation}
\end{tcolorbox}

\subsection{Physical interpretation of observables}

The average value of an observable $\xi$ on a system in a state $|x\rangle $ is expected to be equal to $\langle x|\xi|x\rangle $.

Since an observable form a complete system, and the state can be expressed as a superposition of the orthonormal eigenstates of the observable,
this assumption is perfectly embodied in the formalism, as shown in the following.


\section{Representations}

Representing a quantity is equivalent to the search of the coordinates on a base of the Hilbert space in which it exists.
The simplest base to operate on, in the context of an observable $\xi$, is given by its normalized eigenstates:

\begin{tcolorbox}[colframe=gray!90, colback=blue!5, coltitle=white, title=\textbf{Definition: Orthonormalized Eigenstates}, fonttitle=\large\bfseries]

  A spectrum of an observable $\xi$ is said to be orthonormalized if:

  \begin{equation}
    \langle \xi'|\xi''\rangle  = \delta(\xi'-\xi'') + \delta_{\xi',\xi''} \quad \forall \xi',\xi'' \text{ eigenstates of $\xi$}
  \end{equation}

  where $\delta(\xi'-\xi'')$ is the Dirac delta function, and $\delta_{\xi',\xi''}$ is the Kronecker delta function.

\end{tcolorbox}

\subsection{Ket and Bra representation}

For the superposition principle, the most general state of a quantum system can be expressed as a linear combination of the eigenstates of an observable $\xi$.

\begin{tcolorbox}[colframe=gray!50, colback=gray!10, coltitle=black, title=Ket representation]
  \begin{equation}
    |P\rangle = \sum_{0}^{v} \langle P|\xi_i\rangle |\xi_i\rangle + \int \langle P|\xi'\rangle |\xi'\rangle d\xi'
  \end{equation}
\end{tcolorbox}

Specularly, one can find the corresponding bra representation.

Therefore, the scalar product assumes the form:

\begin{tcolorbox}[colframe=gray!50, colback=gray!10, coltitle=black, title=Scalar product]
  \begin{equation}
    \langle Q|P\rangle = \sum_{0}^{v} \langle Q|\xi_i\rangle \langle \xi_i|P\rangle + \int \langle Q|\xi'\rangle \langle \xi'|P\rangle d\xi'
  \end{equation}
\end{tcolorbox}

thanks to the property of orthonormalization of the eigenstates of the observable $\xi$.

\subsection{Representation of an observable}

\begin{tcolorbox}[colframe=gray!50, colback=gray!10, coltitle=black, title=Observable Representation]
  Given an Hilbert space diagonalized with respect to an observable $\xi$, a generic observable $\chi$ can be represented as:

  \begin{equation}
    \begin{aligned}
       & \chi= \sum_{i,j} \langle \xi_i|\chi|\xi_j\rangle |\xi_i\rangle \langle \xi_j|+ \int\int \langle \xi'|\chi|\xi''\rangle |\xi'\rangle \langle \xi''| d\xi'd\xi'' + \\
       & + \sum_i\int \langle \xi_i|\chi|\xi'\rangle |\xi_i\rangle \langle \xi'|d\xi'
    \end{aligned}
  \end{equation}

\end{tcolorbox}

It follows that the observable in respect to which the space is diagonalized, can be represented as a generalized diagonal matrix:

\begin{equation}
  \xi = \sum_{0}^{v} \xi_i|\xi_i\rangle \langle \xi_i|+ \int \xi'|\xi'\rangle \langle \xi'|d\xi'
\end{equation}

\subsection{Probability of measurement}

To find the probabilty of a measurement of an observable, one can diagonalize the Hilbert space in respect to that observable,
and then square the module of the scalar product between the normalized state and the eigenstate of the observable referred to the specific eigenvalue.
Many examples will be shown in the following.

An immediate consequence is that the mean value of a measurement of an observable is given by:

\begin{tcolorbox}[colframe=gray!50, colback=gray!10, coltitle=black, title=Mean Value of an Observable]
  \begin{equation}
    \langle \xi\rangle  = \sum_i |\langle x|\xi_i\rangle |^2 \xi_i + \int |\langle x|\xi'\rangle |^2 \xi' d\xi'
  \end{equation}
\end{tcolorbox}


\section{Wave function as a representation of a ket}

A generic function of observables can represent the evolution of a quantum system in time.

One defines the wave function as the representation of a ket, that is, as a function of observables operating on the standard ket $\rangle $:

\begin{equation}
  \psi = |\psi\rangle
\end{equation}

So, assuming the presence of the standard ket $\rangle $, it follows that the difference between a generic function of observables $f$ and $\psi$ is that
the latter can be applied to an operator only on the left, while $f$ can be applied to both sides.

The properties it has to satisfy in order to represent a state of a quantum system are:

\begin{tcolorbox}[colframe=gray!90, colback=blue!5, coltitle=white, title=\textbf{Definition: Wave Function}, fonttitle=\large\bfseries]
  $\psi$ $\in$ $L^2(\mathbb{R}^3)$ is a wave function if:
  \begin{itemize}
    \item $\psi$ $\in$ $L^2(\mathbb{R}^3)$.
    \item $P$(x,t)=|$\psi(x,t)$|$^2$ is the probability density of finding the particle in the interval $[x,x+dx]$ at time t.
    \item $\int_{-\infty}^{\infty}$ |$\psi(x,t)$|$^2$dx=1.
  \end{itemize}
\end{tcolorbox}


\subsection{Example of wave function}
Sullo spazio di Hilbert degli stati, diagonalizzato rispetto all'osservabile impulso, le armoniche piane relative ad uno spettro continuo di impulsi, costituiscono una base ortonormale.
Pertanto, considerando un modello di pacchetto d'onda, ovverosia un'onda risultante dalla sovrapposizione di armoniche di base tali che l'impulso sia incluso in
un certo range, si può scrivere la funzione d'onda, secondo la rappresentazione di ket dell'equazione ... come:

\begin{equation}
  \psi(x,t)=\int_{p-\Delta p}^{p+\Delta p} C(p)e^{i(px-Et)/\hbar}dp
\end{equation}

Introducendo un'approssimazione tale per cui risulti:

\begin{itemize}
  \item C quasi costante nell'intervallo.
  \item $\epsilon \simeq \Delta k$, con k vettore d'onda, dipendente da p secondo la relazione di De Broglie.
\end{itemize}

si ottiene:

\begin{equation}
  \psi(x,t)= \frac{1}{\sqrt{\pi \Delta k}}\frac{\sin[(x-\omega_1 t)\Delta k]}{x-\omega_1 t}e^i(k_0x-\omega_0t)
\end{equation}

in cui gli indici 0 e 1 sono riferiti all'indice di espansione in serie di Taylor della pulsazione.

\section{Quantum conditions}

In general, a quantum condition is defined as a relation between observables.
In this section, the importance of the promotion to operators and the commutation relations between the cannical variables are shown.

\subsection{Quantum Poisson brackets}

One can define the 1-to-1 correspondence between the Poisson brackets in phase space and the commutator of operators on the Hilbert space:

\begin{tcolorbox}[colframe=gray!50, colback=gray!10, coltitle=black, title=Quantum Poisson Brackets]

  \begin{equation}
    \{a, b\} = \sum_{i=1}^{n} \left( \frac{\partial a}{\partial q_i} \frac{\partial b}{\partial p_i} - \frac{\partial a}{\partial p_i} \frac{\partial b}{\partial q_i} \right)
    \longleftrightarrow
    [\hat{a}, \hat{b}] = \frac{1}{i\hbar}(\hat{a}\hat{b} - \hat{b}\hat{a})
  \end{equation}

  where $\hat{a}$ and $\hat{b}$ are the dinamic variables promoted to operators, via the correspondence principle.

\end{tcolorbox}

\subsubsection{Commutators of canonical variables}

The commutation relations between the canonical variables fullfill a fundamental role for the development of the formalism:

\begin{equation}
  \left\{
  \begin{aligned}
     & [\hat{q_i}, \hat{q_j}]=0                  \\
     & [\hat{p_i}, \hat{p_j}]=0                  \\
     & [\hat{q_i}, \hat{p_j}]=i\hbar \delta_{ij}
  \end{aligned}
  \right.
\end{equation}

\subsection{Canonical representations}
Diagonalizing the Hilbert space in respect to a specific observable, peculiar representations can be obtained.
The fundamental ones derive from the diagonalization of the canonical observables.

\subsubsection{Coordinate representation}

A generic state can be rapresented as a wave function acting on the standard ket:

\begin{align*}
  |P\rangle = \psi\rangle \equiv \psi(q)
\end{align*}

Now, let's derive the form of the impulse observable in this representation:
in one dimension, one can introduce the derivation operator:

\begin{align*}
  \frac{d}{dq} : \frac{d}{dq} \psi = \frac{d \psi}{dq}
\end{align*}

Therefore:


\begin{align*}
  \frac{d}{dq} |q\psi\rangle = q\frac{d}{dq}\psi\rangle  + \psi\rangle  \longrightarrow q\frac{d}{dq}-\frac{d}{dq}q=1 \longrightarrow [q,\frac{d}{dq}]=-i\hbar
\end{align*}

Which is actually the quantum relation between the canonical variables (15),
which uniquely defines the impulse in the coordinate representation.
In 3 dimensions:

\begin{tcolorbox}[colframe=gray!50, colback=gray!10, coltitle=black, title=Coordinate representation of the impulse]
  \begin{equation}
    \hat{p}_r = -i\hbar \frac{\partial}{\partial q_r}
  \end{equation}
\end{tcolorbox}

\subsubsection{Impulse representation}

...da terminare...

\subsection{Statistics of observables}

Consistently with the physical interpretation of the formalism, one can extrapolate the
expression of the main statistical quantities of an observable:

\begin{tcolorbox}[colframe=gray!50, colback=gray!10, coltitle=black, title=Statistics of an observable]
  Mean value:
  \begin{equation}
    \begin{aligned}
       & \langle \xi\rangle = \langle \psi|\xi|\psi\rangle = \sum \xi_i |\psi_i|^2 \quad discrete       \\
       & \langle \xi\rangle = \langle \psi|\xi|\psi\rangle = \int \psi^*\xi'\psi d\xi' \quad continuous
    \end{aligned}
  \end{equation}

  Variance:
  \begin{equation}
    \begin{aligned}
       & (\Delta \xi)^2=\langle (\xi-\langle \xi\rangle )^2\rangle = \langle \psi|(\xi-\langle \xi\rangle )^2|\psi\rangle
    \end{aligned}
  \end{equation}
\end{tcolorbox}

Indeed, the error on the measurement of an observable is null if and only if the state is an eigenstate of the observable, as expected.

\subsection{Heisenberg Indetermination Principle}

\begin{tcolorbox}[colframe=gray!50, colback=gray!10, coltitle=black, title=Heisenberg Indetermination Principle]
  Given two observables $\xi$ and $\eta$:

  \begin{equation}
    \Delta \xi \Delta \eta \geq \frac{1}{2}|\frac{\langle [A,B]\rangle }{2i}|
  \end{equation}

\end{tcolorbox}

\textbf{Proof:}
Given two generic ket $|a\rangle $ and $|b\rangle $:
\begin{align*}
  \left\{
  \begin{aligned}
     & |a\rangle=(A-\langle A\rangle)|\psi\rangle \\
     & |b\rangle=(B-\langle B\rangle)|\psi\rangle
  \end{aligned}
  \right.
\end{align*}

One can write the following inequality:

\begin{align*}
   & (\Delta A)^2(\Delta B)^2=\langle a|a\rangle\langle b|b\rangle \geq |\langle a|b\rangle |^2 \geq \frac{(\langle f|g\rangle -\langle g|f\rangle )^2}{(2i)^2}=                        \\
   & =\frac{(\langle FG\rangle -\langle F\rangle \langle G\rangle -\langle GF\rangle +\langle G\rangle \langle F\rangle )^2}{(2i)^2}=\frac{\langle [A,B]\rangle ^2}{(2i)^2} \rightarrow \\
   & \rightarrow \Delta A \Delta B \geq |\frac{\langle [A,B]\rangle }{2i}|
\end{align*}

For the canonical operators, in particular, due to the fundamental commutation relations, one has:

\begin{equation}
  \Delta q_i \Delta p_j\geq \frac{\hbar}{2}\delta_{ij}
\end{equation}

\subsection{Time traslation operator and Shrodinger equation}


Translations operators are defined as operators that satisfy the following relation:

\begin{equation}
  |R\rangle =c_1|A\rangle +c_2|B\rangle  \rightarrow |Rd\rangle =c_1|Ad\rangle +c_2|Bd\rangle
\end{equation}

from which derives the unitarity of the traslation operator.

Given a translation operator D, it's easy to dderive how an observable transforms under a traslation:

\begin{align*}
  \left\{
  \begin{aligned}
     & \xi'= |a'><a'|          \\
     & |a'\rangle = D|a\rangle \\
     & <a'|= <a|D^{-1}
  \end{aligned}
  \right.
  \rightarrow \xi'= D\xi D^{-1}
\end{align*}

The form of the time translation operator T and the Shrodinger equation are logically
equivalente postulates of quantum mechanics. One of them is necessary.

\begin{tcolorbox}[colframe=gray!50, colback=gray!10, coltitle=black, title=Time translation operator and Shrodinger equation]
  \begin{equation}
    T=e^{-\frac{i}{\hbar}Ht} \longleftrightarrow i\hbar \frac{\partial \psi}{\partial t}=H\psi
  \end{equation}
\end{tcolorbox}

Equivalently, the Hamiltonian operator form a representation of the Lie algebra of the time traslation operator.

\vspace{1cm}
\textbf{Proof:}

First forward:

\begin{align*}
   & T=e^{-\frac{i}{\hbar}Ht}\rightarrow |\psi(t)\rangle =T|\psi\rangle =e^{-\frac{i}{\hbar}Ht}|\psi\rangle \rightarrow \\
   & \rightarrow i\hbar \frac{\partial}{\partial t}|\psi(t)\rangle= H\frac{\partial}{\partial t}|\psi\rangle
\end{align*}

Then backward:

\begin{align*}
   & i\hbar \frac{\partial}{\partial t} |\psi\rangle = H |\psi\rangle \rightarrow |\psi(t)\rangle=e^{-\frac{i}{\hbar}Ht}|\psi\rangle=T|\psi\rangle
\end{align*}


Therefore, the Lie group of the time traslation operator is represented by the Hamiltonian operator,
meaning that:

\begin{align*}
  T|\psi\rangle=e^{-\frac{i}{\hbar}Ht}|\psi\rangle = -\frac{i}{\hbar}Ht|\psi\rangle+o(t^2)
\end{align*}

\subsection{Heisenberg equations of motion}

Symmetrically to the Shrodinger representation, it's possible to build a formalism in which the states remain unchanged in time,
while the observables change.

\begin{tcolorbox}[colframe=gray!50, colback=gray!10, coltitle=black, title=Heisenberg equations of motion]
  \begin{equation}
    \frac{d\xi}{dt}=\frac{i}{\hbar}[\xi,H]
  \end{equation}
\end{tcolorbox}

\textbf{Proof:}
For a generic observable $\xi$:

\begin{align*}
   & \xi (t) = T\xi T^{-1} \rightarrow T\xi (t) = \xi T \rightarrow                     \\
   & \frac{d}{dt}T\xi T^{-1} = \frac{d}{dt}T\xi T^{-1} \rightarrow                      \\
   & \rightarrow HT\xi (t) + i\hbar T\frac{d\xi (t)}{dt} = T\frac{d\xi}{dt} \rightarrow \\
   & \rightarrow i\hbar \frac{d\xi}{dt} = T\xi (t)H - H\xi (t)T \rightarrow             \\
   & \rightarrow \frac{d\xi}{dt} = \frac{i}{\hbar}[\xi,H]
\end{align*}

Therefore, if an observable is constant in time, it must commute with the Hamiltonian.




Diagonalizing the Hilbert space in respect to the Hamiltonian, one finds that the matrix elements of the observables are periodic in time, with a frequency given by the difference of the eigenvalues of the Hamiltonian:

\begin{tcolorbox}[colframe=gray!50, colback=gray!10, coltitle=black, title=Matrix elements periodicity]
  \begin{equation}
    \nu= \frac{E'-E''}{h}
  \end{equation}
\end{tcolorbox}

\textbf{Proof:}

The only hypotesis of diagonalization in respect to H allows to convert the exponential of an operator in the exponential of a scalar:
\begin{align*}
  <H'|\xi(t)|H''>=<H'|T \xi T^{-1} |H''>=e^{\frac{i}{\hbar}(H'-H'')t}<H'|\xi|H''>
\end{align*}

From which one can extract the periodicity of the matrix elements.

\chapter{1D Systems}

\section{Introduction}
In the Shrodinger picture, a state evolves as:

\begin{equation}
  \psi(x,t)=T\psi(x,0)=e^{-\frac{i}{\hbar}H't}\psi(x,0)
\end{equation}

this implies that the Shrodinger equation can be conveniently reduced for stationary systems to the eigenvalue equation of the Hamiltonian for t=0:

\begin{align*}
   & i\hbar \frac{\partial}{\partial t}  |\psi(t)\rangle = H|\psi(t)\rangle= E|\psi(t)\rangle =                                                           \\
   & i\hbar e^{-\frac{i}{\hbar}H't}\frac{\partial}{\partial t} \psi(x,0)=H'e^{-\frac{i}{\hbar}H't}\psi(x,0)=Ee^{-\frac{i}{\hbar}H't}\psi(x,0) \rightarrow \\
   & \rightarrow H |\psi(x, 0)\rangle = \hat{p}^2\psi(x,0)+V(x)\psi(x,0)=E\psi(x,0) \rightarrow                                                           \\
   & -\hbar^2\frac{d^2}{dx^2}\psi(x,0)+V(x)\psi(x,0)=E\psi(x,0)
\end{align*}

To obtain the solution at a generic time, one can use the time traslation operator T.

One dimensional system are kinda of a paradox, since they would simply violate the Heisenberg principle for
the othogonal components.

However, their importance emerges whenever a problem can be broken down into
one dimensional problems, for independent degrees of freedom.

\subsection{Fundamental theorems}

\begin{tcolorbox}[colframe=gray!50, colback=gray!10, coltitle=black, title=Non degeneracy of the spectrum]
  The spectrum of the Hamiltonian of a one dimensional system is non degenerate.
\end{tcolorbox}

\textbf{Proof:}

Suppose that 2 states are eigenstates of the Hamiltonian with the same eigenvalue:

\begin{align*}
   & \left\{
  \begin{aligned}
     & H|\psi_1\rangle = -\hbar^2\frac{d^2}{dx^2}\psi_1(x)+V(x)\psi_1(x)=E\psi_1(x) \\
     & H|\psi_2\rangle = -\hbar^2\frac{d^2}{dx^2}\psi_2(x)+V(x)\psi_2(x)=E\psi_2(x)
  \end{aligned}
  \right. \rightarrow                                                                                                                                               \\
   & \psi_2 \frac{d^2 \psi_1}{dx^2} - \psi_1 \frac{d^2 \psi_2}{dx^2}=0                                                                                              \\                                                                                         \\
   & \frac{d}{dx}(\psi_2 \frac{d \psi_1}{dx} - \psi_1 \frac{d \psi_2}{dx})=0 \rightarrow \psi_2 \frac{d \psi_1}{dx} - \psi_1 \frac{d \psi_2}{dx}=cost=0 \rightarrow \\
   & \frac{d}{dx}\ln(\psi_1)=\frac{d}{dx}\ln(\psi_2) \rightarrow \ln(\psi_1)=\ln(\psi_2)+cost \rightarrow \psi_1=\psi_2e^{cost}=C\psi_2
\end{align*}

Where the constant is equalized to zero, since $\psi_{1,2}\in \mathcal{L}^2$.


\begin{tcolorbox}[colframe=gray!50, colback=gray!10, coltitle=black, title=Parity of the solutions]
  The solutions of the Shrodinger equation for a one dimensional system with even Hamiltonian are either even or odd.
\end{tcolorbox}

\begin{tcolorbox}[colframe=gray!50, colback=gray!10, coltitle=black, title=Minimum value]
  If the potential V(x) has an absolute minimum, any Hamiltonian eigenvalue is greater or equal to the minimum value of the potential.
\end{tcolorbox}

\textbf{Proof:}

\begin{align*}
   & -\frac{\hbar^2}{2m}\int_{-\infty}^\infty \psi*(x)\frac{d^2}{dx^2} \psi(x)dx + \int_{-\infty}^\infty V(x)\psi*(x)\psi(x)dx = E\int_{-\infty}^\infty \psi*(x)\psi(x)dx \leftrightarrow \\
   & \leftrightarrow <V>+<T>=E\geq V_{min}+<T>\geq V_{min}
\end{align*}


\section{Infinite potential well}

\begin{tcolorbox}[colframe=gray!50, colback=gray!10, coltitle=black, title=Infinite potential well]
  Given a potential V(x) such that:
  \begin{equation}
    \left\{
    \begin{aligned}
       & V(x)=0 \quad se \quad 0\leq x\leq L \\
       & V(x)=\infty \quad otherwise
    \end{aligned}
    \right.
  \end{equation}

  a general state is given by:

  \begin{equation}
    \Psi(x,t)=\left(\sum_0^\infty \psi_n(x) \right)e^{-\frac{i}{\hbar}H t}= \left(\sum_0^\infty \sqrt{\frac{2}{L}}\sin\left(\frac{n \pi}{L}\right) \right)e^{-\frac{i}{\hbar}H t}
  \end{equation}

  with energy

  \begin{equation}
    E= \sum_0^\infty E_n= \sum_0^\infty \frac{\hbar^2 \pi^2 n^2}{2mL^2}
  \end{equation}

\end{tcolorbox}



The particle function cannot exist for x<0 and x>L, so $\psi(0)=\psi(L)=0$.

In the segment, the spectrum is discrete as shown it the following:

\begin{align*}
   & H|psi>=-\hbar^2\frac{d^2}{dx^2} \psi(x)= E\psi(x) \rightarrow \frac{d^2}{dx^2}\psi(x)=-\frac{2mE}{\hbar^2}\psi(x) \rightarrow \\
   & \left\{
  \begin{aligned}
     & \psi(x)=A\sin(kx)+B\cos(kx) \\
     & \psi(0)=\psi(L)=0           \\
     & k= \frac{\sqrt{2mE}}{\hbar}
  \end{aligned}
  \right. \rightarrow                                                                                                              \\
   & \left\{
  \begin{aligned}
     & \psi=A\sin(kx)
     & k= \frac{\sqrt{2mE}}{\hbar}=\frac{n \pi}{L}
  \end{aligned}
  \right. \rightarrow
   & E_n= \frac{\hbar^2 n^2 \pi^2}{2mL^2}
\end{align*}

The condition of normalization leads to:

\begin{align*}
  \int_0^L|\psi(x)|^2 dx=A^2\int_0^L \sin^2(kx)dx=1 \rightarrow A=\sqrt{\frac{2}{L}}
\end{align*}


One can verify that the eigenstates form a complete othonormal system for $\mathcal{L}^2(0,L)$.

One can even calculate the mean value of the position on a stationary state.

...da inserire calcoli...



\section{Delta attractive potential}

\begin{tcolorbox}[colframe=gray!50, colback=gray!10, coltitle=black, title=Delta attractive potential]

  Given the potential V(x):
  \begin{align*}
    V(x)=-\alpha \delta(x)
  \end{align*}


\end{tcolorbox}


\textbf{Proof:}

The eigenvalues Shrodinger equation is
\begin{align*}
  -\frac{\hbar^2}{2m}\frac{d^2}{dx^2}\psi-\alpha \delta(x)\psi=E\psi
\end{align*}

\subsection[short]{Stati legati}
Mi chiedo se possano esistere stati legati, ossia stati tali per cui la probabilità all'infinito sia nulla.
Per x $\neq$ 0, l'equazione di Shroedinger è:

\begin{equation}
  \left\{
  \begin{aligned}
     & \psi(x)=Be^{k|x|}
     & k=\frac{\sqrt{2mE}}{\hbar}
    \\ B= \sqrt{k}
  \end{aligned}
  \right.
\end{equation}

Dunque esiste solo uno stato legato per E < 0.
Tale funzione risulta a quadrato sommabile, ma con derivata discontinua per x=0. Esaminiamone le conseguenze, integrando
l'equazione agli autovalori:

\begin{equation}
  \begin{aligned}
     & \int_{-\epsilon}^{\epsilon} \frac{d^2}{dx^2} \psi dx = \int_{-\epsilon}^{\epsilon}\frac{2m\alpha}{\hbar^2}\delta(x)\psi dx= \int_{-\epsilon}^{\epsilon} \frac{2mE}{\hbar^2}\psi dx \\
     & \psi'(\epsilon)-\psi'(-\epsilon)+\frac{2m\alpha}{\hbar^2}\psi(0)=k^2[\psi(\epsilon)- \psi(-\epsilon)] \rightarrow                                                                  \\
     & \rightarrow \Delta \psi'= -\frac{2m\alpha}{\hbar^2}\sqrt{k} \rightarrow                                                                                                            \\
     & k= \frac{m\alpha}{\hbar^2} \rightarrow E=-\frac{m\alpha^2}{2\hbar^2}
  \end{aligned}
\end{equation}

dunque il salto è finito, per via del fatto che il potenziale è infinito.

Si calcolano ora i valori medi delle osservabili canoniche: * da terminare *

\subsection{Stati del continuo}
Considero E > 0, per cui l'equazione agli autovalori assume la forma:

\begin{equation}
  \frac{d^2}{dx^2} \psi+k^2\psi=0
\end{equation}

la cui soluzione più generale sull'intero asse reale è:
\begin{equation}
  \left\{
  \begin{aligned}
     & \psi(x)=Ae^{ikx}+Be^{-ikx} \quad , x\langle 0 \\
     & \psi(x)=Fe^{ikx}+Ge^{-ikx} \quad , x\rangle 0 \\
     & k=\frac{\sqrt{2mE}}{\hbar}\rangle 0
  \end{aligned}
  \right.
\end{equation}

per le condizioni di riflessione e trasmissione si ha:

\begin{equation}
  \left\{
  \begin{aligned}
     & A+B=F+G                                                                        \\
     & ik(A-B)=ik(F-G) \rightarrow \Delta= ik(F-G-A+B)=-\frac{2m\alpha}{\hbar^2}(A+B)
  \end{aligned}
  \right.
\end{equation}

ne segue, ponendo $\beta=\frac{m\alpha}{\hbar^2k}$:

\begin{equation}
  F-G=A(1+2i\beta)-B(1-2i\beta)
\end{equation}

se G=0 (onda proveniente da $-infty$), si ha:

\begin{equation}
  \left\{
  \begin{aligned}
     & F=A+B
     & F= A(1+2i\beta)-B(1-2i\beta)
  \end{aligned}
  \right.
\end{equation}

da cui:

\begin{equation}
  \left\{
  \begin{aligned}
     & B= \frac{i\beta}{1-i\beta}A \\
     & F= \frac{1}{1-i\beta}A
  \end{aligned}
  \right.
\end{equation}

considerando che il modulo quadro dei coefficienti rappresenta la densità di probabilità dello stato nella rispettiva regione,
è possibile definire le correnti di probabilità come:

\begin{equation}
  \left\{
  \begin{aligned}
     & J_{inc}= \frac{\hbar k}{m}|A|^2 \\
     & J_{tr}= \frac{\hbar k}{m}|F|^2  \\
     & J_{rif}= \frac{\hbar k}{m}|B|^2
  \end{aligned}
  \right.
\end{equation}

da cui si ricavano i coefficienti di riflessione e trasmissione:

\begin{equation}
  \left\{
  \begin{aligned}
     & R= \frac{|B|^2}{|A|^2}= \frac{\beta^2}{1+\beta^2} \\
     & T= \frac{|F|^2}{|A|^2}= \frac{1}{1+\beta^2}       \\
     & \beta= \frac{m\alpha}{\hbar^2k}
  \end{aligned}
  \right.
\end{equation}

\subsection{Buca di potenziale rettangolare simmetrica}

Il potenziale è dato da:

\begin{equation}
  V(x)=
  \left\{
  \begin{aligned}
     & 0 \quad se \quad |x|\langle L \\
     & V_0 \quad altrimenti
  \end{aligned}
  \right.
\end{equation}

\subsubsection{Stati legati}
Cerco gli staticon energia tale per cui risulti un confinamento nella buca (0$\langle $E$\langle V_0$).
Il procedimento è analogo a quello seguito per ricavare le funzioni d'onda nei casi precedenti:
si risolvono le equazioni differenziali agli autovalori nelle 3 aree in cui lla retta risulta divisa, e si raccordano
le soluzioni.

\begin{equation}
  \left\{
  \begin{aligned}
     & \psi''+k^2\psi=0 \quad , \quad |x|\langle L                                 \\
     & \psi''-\eta^2\psi=0 \quad , \quad |x|\rangle L                              \\
     & \eta= \frac{\sqrt{2m(V_0-E)}}{\hbar} \quad \kappa= \frac{\sqrt{2mE}}{\hbar}
  \end{aligned}
  \right.
\end{equation}

da cui, risolvendo ed applicando le condizioni a contorno, si ha:

\begin{equation}
  \left\{
  \begin{aligned}
     & \psi_1(x)= Ce^{\eta x}+De^{-\eta x}= Ce^{\eta x}  \\
     & \psi_2(x)= A\cos(kx)+B\sin(kx)                    \\
     & \psi_3(x)= Fe^{\eta x}+Ge^{-\eta x}= Ge^{-\eta x}
  \end{aligned}
  \right.
\end{equation}

Per il teorema di parità, le soluzioni devono essere pari o dispari. Analizzo il caso in cui siano pari.

\begin{equation}
  \left\{
  \begin{aligned}
     & \psi_1(x)= Ce^{\eta x} \rightarrow \psi'_1(x)= \eta Ce^{\eta x}    \\
     & \psi_2(x)= A\cos(kx) \rightarrow \psi'_2(x)= -Ak\sin(kx)           \\
     & \psi_3(x)= Ce^{-\eta x} \rightarrow \psi'_3(x)= -\eta Ce^{-\eta x}
  \end{aligned}
  \right.
\end{equation}

* inserire tutto il calcolo fatto con tavoletta *

\subsubsection{Stati del continuo}
*da terminare*

\subsection{Effetto tunnel}
Si consideri un potenziale del tipo:

\begin{equation}
  V(x)=
  \left\{
  \begin{aligned}
     & 0 \quad se \quad |x|\rangle L \\
     & V_0 \quad altrimenti
  \end{aligned}
  \right.
\end{equation}

Non essendovi chiaramente stati legati, si calcolano i coefficienti di riflessione e trasmissione.

Consideriamo solamente il caso in cui $0\langle E\langle V_0$.
L'equazione di Shrodinger agli autovalori diviene:

\begin{equation}
  \left\{
  \begin{aligned}
     & \psi''+k^2\psi=0  \quad , \quad |x|\rangle L                                \\
     & \psi''-\eta^2\psi=0  \quad , \quad |x|\langle L                             \\
     & \eta= \frac{\sqrt{2m(V_0-E)}}{\hbar} \quad \kappa= \frac{\sqrt{2mE}}{\hbar}
  \end{aligned}
  \right.
\end{equation}

Nella regione esterna, le soluzioni saranno funzioni oscillanti, mentre in quella interna vi saranno oscillazioni smorzate.

\begin{equation}
  \psi(x)=
  \left\{
  \begin{aligned}
     & Ae^{ikx}+Be^{-ikx} \quad , \quad x\langle -L        \\
     & Ce^{\eta x}+De^{-\eta x} \quad , \quad |x|\langle L
     & Fe^{ikx} \quad , \quad x\rangle L
  \end{aligned}
  \right.
\end{equation}

le derivate prime sono:

\begin{equation}
  \psi'(x)=
  \left\{
  \begin{aligned}
     & ik(Ae^{ikx}-Be^{-ikx}) \quad , \quad x\langle -L          \\
     & \eta(Ce^{\eta x}-De^{-\eta x}) \quad , \quad |x|\langle L
     & ikFe^{ikx} \quad , \quad x\rangle L
  \end{aligned}
  \right.
\end{equation}

imponendo la continuità delle funzioni e delle derivate prime, si ottengono le seguenti relazioni:

\begin{equation}
  \left\{
  \begin{aligned}
     & A= \frac{C}{2}\left(1+\frac{ik}{\eta}\right)e^{(ik-\eta)L}+ \frac{D}{2}\left(1-\frac{ik}{\eta}\right)e^{(ik+\eta)L}   \\
     & B= \frac{C}{2}\left(1-\frac{ik}{\eta}\right)e^{-(ik+\eta)L}+ \frac{D}{2}\left(1+\frac{ik}{\eta}\right)e^{(-ik+\eta)L} \\
     & C=\frac{F}{2}\left(1+\frac{ik}{\eta}\right)e^(ik-\eta)L                                                               \\
     & D=\frac{F}{2}\left(1-\frac{ik}{\eta}\right)e^(ik+\eta)L                                                               \\
  \end{aligned}
  \right.
\end{equation}

da cui:

\begin{equation}
  \left\{
  \begin{aligned}
     & A=\frac{F}{4ik\eta}e^{2ikL}\left[(\eta+ik)^2e^{-2\eta L}- (\eta-ik)^2e^{2\eta L}\right] \\
     & B= \frac{F}{2ik\eta}(\eta^2+k^2)\sinh(2\eta L)
  \end{aligned}
  \right.
\end{equation}

Da cui è possibile ricavare i coefficienti di riflessione e trasmissione:

\begin{equation}
  \left\{
  \begin{aligned}
     & R= \frac{|B|^2}{|A|^2} \\
     & T= \frac{1}{|A|^2}
  \end{aligned}
  \right.
\end{equation}

Questo risultato è noto come "Effetto Tunnel" e non ha un analogo in meccanica classica, ed indica che esiste una probabilità non nulla
che una particella attraversi una barriera di potenziale, anche se la sua energia è inferiore a quella della barriera stessa.

Tale effetto è alla base del funzionamento dei diodi.



\subsection{Oscillatore armonico}
\subsubsection{Metodo 1}
La trattazione matematica dell'oscillatore armonico quantistico conduce direttamente alla quantizzazione dell'energia per tale sistema.
Di seguito riporto gli step logici che possono essere seguiti:

\begin{itemize}
  \item Si scrive l'Hamiltoniano dell'oscillatore armonico, mediante il principio di corrispondenza, come:

        \begin{equation}
          \hat{H}=\frac{1}{2m}(\hat{p}^2+m^2\omega^2\hat{q}^2)
        \end{equation}
  \item Si scrivono le equazioni del moto nella rappresentazione di Heisenberg nelle coordinate canoniche:

        \begin{equation}
          \left\{
          \begin{aligned}
             & \dot{q}=\frac{1}{m}\hat{p} \\
             & \dot{p}=-m\omega^2\hat{q}
          \end{aligned}
          \right.
        \end{equation}

  \item Si introduce la variabile dinamica complessa $\eta$ tale che:

        \begin{equation}
          \eta= \frac{\hat{p}+im\omega \hat{q}}{\sqrt{2m\hbar \omega}}
        \end{equation}

        e se ne ricava l'equazione del moto:

        \begin{equation}
          \hat{\eta}=\hat{\eta_0} e^{i\omega t}
        \end{equation}

  \item Si ricavano i commutatori tra $\hat{\eta}$ e $H$:

        \begin{equation}
          \left\{
          \begin{aligned}
             & [\hat{\eta},\hat{\eta}^{\dagger}]=1    \\
             & [\hat{\eta},H]=\hbar \omega \hat{\eta}
          \end{aligned}
          \right.
        \end{equation}

  \item Considerando H' autovalore di $\hat{H}$, si scrive:

        \begin{equation}
          \hbar \omega\langle H' |\hat{\eta}\hat{\eta}^{\dagger}|H'\rangle =(H'-\frac{1}{2}\hbar \omega)\langle H' |H'\rangle
        \end{equation}

        che impone che un generico autovalore H' sia tale che:

        \begin{equation}
          H'\geq \frac{1}{2}\hbar \omega
        \end{equation}

  \item Dalle proprietà dei commutatori si ricava la serie di autovalori assumibili da H:

        \begin{equation}
          H_n=(n+\frac{1}{2})\hbar \omega \quad , \quad n \in N
        \end{equation}

  \item Si può procedere nel trovare la rappresentazione di Schrödinger dell'oscillatore armonico.
\end{itemize}

\subsubsection{Metodo 2, Algebrico}

Considero la rappresentazione di Heisenberg (diagonalizzazione dello spazio di Hilbert rispetto all'Hamiltoniana).
L'Hamiltoniana dell'oscillatore armonico monodimensionale è data da:

\begin{equation}
  \left\{
  \begin{aligned}
     & H=\hbar\omega\tilde{H}=\hbar\omega\frac{1}{2}(\tilde{p}^2+\tilde{x}^2) \\
     & \tilde{p}=\frac{p}{\sqrt{m\hbar \omega}}                               \\
     & \tilde{x}= \sqrt{\frac{m\omega}{\hbar}}x                               \\
     & \tilde{H}|\nu\rangle =\epsilon|\nu\rangle
  \end{aligned}
  \right.
\end{equation}

Introduco ora i seguenti operatori:

\begin{equation}
  \left\{
  \begin{aligned}
     & a= \frac{1}{\sqrt{2}}(\tilde{x}+i\tilde{p})           \\
     & a^{\dagger}= \frac{1}{\sqrt{2}}(\tilde{x}-i\tilde{p})
  \end{aligned}
  \right.
\end{equation}

tali per cui valgano:

\begin{equation}
  \left\{
  \begin{aligned}
     & [a,a^{\dagger}]=1                                    \\
     & \tilde{H}=\dagger aa+\frac{1}{2}\equiv N+\frac{1}{2} \\
     & [N,a]=-a \quad , \quad [N,a^{\dagger}]=a^{\dagger}
  \end{aligned}
  \right.
\end{equation}

vale la diagonalizzazione simultanea di N e H, per via della commutazione tra i due operatori.

\begin{equation}
  \epsilon_\nu=\nu+\frac{1}{2} \rightarrow E_\nu=\hbar \omega(\nu+\frac{1}{2})
\end{equation}

\begin{itemize}
  \item Lo spettro di N è tutto positivo, o al più nullo.
  \item Lo spettro è strettamente positivo.
  \item Lo stato $a|\nu\rangle $ appatiene all'autospazio di N relativo all'autovalore $\nu-1$.
        \begin{equation}
          Na|\nu\rangle =aN|\nu\rangle -a|\nu\rangle =(\nu-1)a|\nu\rangle
        \end{equation}
        Considerando autostati normalizzati, vale:
        \begin{equation}
          ||a|\nu\rangle ||= \sqrt{\langle \nu|\dagger a a|\nu\rangle }=\sqrt{\langle \nu|N|\nu\rangle }=\sqrt{\nu} \rightarrow a|\nu\rangle =\sqrt{\nu}|\nu-1\rangle
        \end{equation}
  \item Analogamente, si trova che:
        \begin{equation}
          a^{\dagger}|\nu\rangle =\sqrt{\nu+1}|\nu+1\rangle
        \end{equation}
  \item Applicando l'operatore a, si ottiene un autostato di N relativo all'autovalore $\nu-1$. Dunque,
        procedendo con la sua applicazione ad un generico autostato di N, si deve arrivare ad ottenere uno stato nullo e mai negativo.

        Al contrario, applicando l'operatore $a^{\dagger}$, si ottiene un autostato di N relativo all'autovalore $\nu+1$, senza vincoli di limitatezza.
        Dunque si è dimostrato che lo spettro di N è $\mathcal{N}$.
  \item A questo punto, abbiamo terminato, in quanto:

        \begin{equation}
          \begin{aligned}
             & N |n\rangle =n|n\rangle  \quad , \quad n \in \mathcal{N}                  \\
             & H= \hbar\omega(N+\frac{1}{2}) \rightarrow E_n= \hbar\omega(n+\frac{1}{2})
          \end{aligned}
        \end{equation}
  \item Lo stato ad energia minima (Energia del vuoto) sarà dunque:

        \begin{equation}
          E_0=\frac{1}{2}\hbar \omega
        \end{equation}
\end{itemize}

Questo approccio è fertile in termini di formalismo.

\subsubsection{Metodo Analitico}
*da terminare*


\section{Momento angolare}
\subsection{Momento angolare orbitale}
In termini operatoriali, tramite il principio di corrispondenza, il momento angolare orbitale è rappresentabile come:

\begin{equation}
  \hat{M}=\hat{x}\wedge \hat{p} \leftrightarrow \hat{L_i}=\epsilon_{ijk}\hat{x}_j\hat{p}_k
\end{equation}

E' possibile ricavare le seguenti fondamentali condizioni quantiche di commutazione:

\begin{itemize}
  \item Tra $L_i$ e $x_r$:

        \begin{equation}
          \begin{aligned}
             & [L_i, x_r]= \epsilon_{ijk}[x_jp_k, x_r]=\epsilon_{ijk}{x_j[p_k, x_r]+[x_j, x_r]p_k} = \\
             & = -\epsilon_{ijk}{x_ji\hbar\delta_{kr}}=i\hbar\epsilon_{ijr}x_j
          \end{aligned}
        \end{equation}
  \item Tra $L_i$ e $p_r$:

        \begin{equation}
          \begin{aligned}
             & [L_i, p_r]= \epsilon_{ijk}[x_jp_k, p_r]= \epsilon_{ijk}{x_j[p_k, p_r]+[x_j, p_r]p_k}= \\
             & = \epsilon_{ijk}{i\hbar \delta_{jr}p_k}=i\hbar\epsilon_{irj}p_j
          \end{aligned}
        \end{equation}


  \item Tra $L_i$ e $L_j$ (Calcoli su "Calcoli integrativi"):

        \begin{equation}
          \begin{aligned}
             & [L_i,L_j]=...=i\hbar\epsilon_{ijk}L_k
          \end{aligned}
        \end{equation}

  \item Tra $L^2$ e $L_i$:
        \begin{equation}
          \begin{aligned}
             & [L^2,L_i]=...=0
          \end{aligned}
        \end{equation}
\end{itemize}

Dunque è possibile diagonalizzare simultaneamente lo spazio di Hilbert rispetto a $L^2$ e $L_i$
Per convenzione si sceglie la base di autoket di $L^2$ e $L_z$.

In coordinate polari, gli operatori $L^2$ e $L_z$ assumono la forma:

\begin{equation}
  \left\{
  \begin{aligned}
     & L^2=-\hbar^2\left(\frac{\partial^2}{\partial \theta^2}+\cot{\theta}\frac{\partial}{\partial \phi}+\frac{1}{\sin^2(\theta)}\frac{\partial^2}{\partial \phi^2}\right)
     & L_z=-i\hbar\frac{\partial}{\partial \phi}
  \end{aligned}
  \right.
\end{equation}

Con tale sistema di coordinate, la quantizzazione del momento angolare viene effettuata mediante la risoluzione dell'equazione agli autovalori:

\begin{equation}
  \left\{
  \begin{aligned}
     & L^2Y_{lm}(\theta, \phi)=l(l+1)\hbar^2Y_{lm}(\theta, \phi) \\
     & L_zY_{lm}(\theta, \phi)=m\hbar Y_{lm}(\theta, \phi)
  \end{aligned}
  \right.
\end{equation}

tale per cui risulta che le autofunzioni di base siano le armoniche sferiche:

\begin{equation}
  Y_{lm}(\theta, \phi) = \frac{(-1)^{l+m}}{2^l l!} \sqrt{\frac{2l+1}{4\pi} \frac{(l-m)!}{(l+m)!}} \sin^m(\theta) \left( \frac{d}{d\cos{\theta}} \right)^{l+m} e^{im\phi}
\end{equation}

\subsection{Momento angolare di spin}
Il momento angolare di spin è una proprietà intrinseca di una qualsiasi particella elementare, indipendente dal suo stato di moto.

Ne segue che essa determini un grado di libertà aggiuntivo, e, pertanto, commuta con qualsiasi funzione delle osservabili canoniche.

L'operatore di spin gode delle medesime proprietà dell'operatore di momento angolare orbitale, e così una loro generica combinazione lineare.
Dato che il momento angolare commuta con qualsiasi scalare, e che l'Hamiltoniana è invariante per rotazione spaziale (e dunque uno scalare), il momento angolare è una costante del moto.

Perciò la funzione d'onda sarà fattorizzabile, in coordinate polari, come:

\begin{equation}
  \psi(r,\theta, \phi,\vec{z})=R(r)Y^m_l(\theta, \phi) \chi(\vec{z})
\end{equation}

Assumendo lo spin soltanto due valori, la funzione d'onda sarà data dalla sovrapposizione di due stati a spin positivo e negativo:

\begin{equation}
  \psi(\vec{r},\vec{z})=\psi_-(\vec{r})\chi_-(\vec{z}) + \psi_+(\vec{r})\chi_+(\vec{z})
\end{equation}

Lo spazio vettoriale degli stati di spin è bidimensionale, per cui vale la rappresentazione in ket della base (detti spinori):

\begin{equation}
  \chi_+=\begin{pmatrix}
    1 \\
    0
  \end{pmatrix}
  \quad
  \chi_-=\begin{pmatrix}
    0 \\
    1
  \end{pmatrix}
\end{equation}

Le osservabili di spin, operando su uno spazio vettoriale bidimensionale, sono rappresentabili come matrici 2x2, composizione lineare di matrici di Pauli:

\begin{equation}
  S=\frac{\hbar}{2}\sigma \leftrightarrow
  \sigma_x=\begin{pmatrix}
    0 & 1 \\
    1 & 0
  \end{pmatrix}
  \quad
  \sigma_y=\begin{pmatrix}
    0 & -i \\
    i & 0
  \end{pmatrix}
  \quad
  \sigma_z=\begin{pmatrix}
    1 & 0  \\
    0 & -1
  \end{pmatrix}
\end{equation}

\subsection{Composizione di momenti angolari}

Considero la somma operatoriale di due momenti angolari generici ( e un autostato di $L^2$ e $L_z$):

\begin{equation}
  \left\{
  \begin{aligned}
     & \vec{J}=\vec{J_1}+\vec{J_2}              \\
     & J^2|j,m\rangle =j(j+1)\hbar^2|j,m\rangle \\
     & J_z|j,m\rangle =m\hbar|j,m\rangle
  \end{aligned}
  \right.
\end{equation}

Fissagti $j_1$ e $j_2$, considero il sottospazio vettoriale costituito da ket del tipo $|m_1,m_2\rangle $, avente dunque dimensione pari a $(2j_1+1)(2j_2+1)$.

Dimostrando che il ket $|j,m\rangle $ costituisce una base di tale spazio, si costruisce un'agevole rappresentazione del momento angolare totale.

Si verifica che $j$ assume valori compresi tra $|j_1-j_2|$ e $j_1+j_2$, e che $m=m_1+m_2$.

Si può così valutare la dimensione dello spazio generato da $|j,m\rangle $:

\begin{equation}
  \sum_{j_1-j_2}^{j_1+j_2}=(2j_1+1)(2j_2+1)
\end{equation}

Perciò i due spazi, essendo isomorfi, ammettono entrambe le basi.

La trasformazione tra le due basi è fornita dai coefficienti di Clebsch-Gordan:

\begin{equation}
  |j,m\rangle =\sum_{m_1=-j_1}^{j_1} \sum_{m_2=-j_2}^{j_2}\langle m_1,m_2|j,m\rangle |m_1,m_2\rangle
\end{equation}

tenendo conto che se m non è somma di $m_1$ e $m_2$, il coefficiente di Clebsch-Gordan è nullo.

Si utilizza anche la notazione di produttoria diretta di spazi vettoriali:

\begin{equation}
  (j_1)+(j_2)= \oplus_{j=j_1-j_2}^{j_1+j_2} (j)
\end{equation}

\section{Sistemi stazionari a potenziale centrale}
L'Hamiltoniana di sistemi a campo centrale è del tipo:

\begin{equation}
  H=\frac{1}{2m}p^2+V(r)
\end{equation}

Trattandosi di sistemi stazionari, ci si può ricondurre alla risoluzione dell'equazione agli autovalori dell'Hamiltoniano,
con l'aggiuntiva prerogativa di fattorizzazione delle soluzioni in una parte spaziale e in una angolare (trascurando lo spin).

In particolare, ciò è dato dal fatto che l'Hamiltoniano commuta con $L_3$ e $L^2$, e dunque è possibile trovare delle autofunzioni simultanee dei tre operatori:

\begin{equation}
  \left\{
  \begin{aligned}
     & H|\psi\rangle =E|\psi\rangle \\
     & \psi= R(r)Y(\theta, \phi)
  \end{aligned}
  \right.
\end{equation}

da cui segue l'equazione degli autovalori dell'Hamiltoniano:

\begin{equation}
  \begin{aligned}
     & \frac{\hbar^2r^2}{R(r)}\left(\frac{\partial^2}{\partial r^2}+\frac{2}{r}\frac{\partial}{\partial r}  \right)R(r)+2mr^2[V(r)-E] =-\frac{1}{Y^l_m(\theta,\phi)}L^2Y^l_m(\theta,\phi) \\
  \end{aligned}
\end{equation}

Dunque entrambi i membri, dipendendo da variabili diverse, devono necessariamente essere uguali ad una costante,
ossia:

\begin{equation}
  \left\{
  \begin{aligned}
     & L^2Y^l_m(\theta,\phi)=l(l+1)\hbar^2Y^l_m(\theta,\phi)                                 \\
     & L_3Y^l_m(\theta,\phi)=m\hbar Y^l_m(\theta,\phi)                                       \\
     & \frac{d^2}{dr^2}R_l + \left(\frac{2m[E-V(r)]}{\hbar^2}-\frac{l(l+1)}{r^2}\right)R_l=0
  \end{aligned}
  \right.
\end{equation}

ridefinendo la funzione $R_l(r)=\frac{\chi_l(r)}{r}$, si ottiene:

\begin{equation}
  \left\{
  \begin{aligned}
     & -\frac{\hbar^2}{2m}\frac{d^2}{dr^2}\chi_l(r)+V_{eff}(r)\chi_l(r)=E\chi_l(r) \\
     & V_{eff}(r)=V(r)+\frac{\hbar^2}{2m}\frac{l(l+1)}{r^2}
  \end{aligned}
  \right.
\end{equation}

\subsection{Buca di potenziale sferica}
Il potenziale è del tipo:

\begin{equation}
  \left\{
  \begin{aligned}
     & -V_0 \quad se \quad 0\langle r\langle a \\
     & 0 \quad altrimenti
  \end{aligned}
  \right.
\end{equation}

Si studiano gli stati legati, ovvero tali per cui $-V_0\langle E\langle 0$.

Per r< a, l'equazione degli autovalori per la parte radiale diviene:

\begin{equation}
  \left\{
  \begin{aligned}
     & \frac{d^2}{dr^2}R_l(r)+\frac{2}{r}\frac{d}{dr}R_l(r)+\left(\alpha^2-\frac{l(l+1)}{r^2}\right)R_l(r)=0 \\
     & \alpha=\frac{\sqrt{2m(E+V_0)}}{\hbar}
  \end{aligned}
  \right.
\end{equation}

Per r> a, l'equazione diviene:

\begin{equation}
  \left\{
  \begin{aligned}
     & \frac{d^2}{dr^2}R_l(r)+\frac{2}{r}\frac{d}{dr}R_l(r)-\left(\beta^2+\frac{l(l+1)}{r^2}\right)R_l(r)=0 \\
     & \beta=\frac{\sqrt{-2mE}}{\hbar}
  \end{aligned}
  \right.
\end{equation}


La prima è risolvibile con funzioni di Bessel, mentre la seconda con funzioni di Hankel.
Per $r\langle a$ si ha, ponendo $x=\alpha r$:

\begin{equation}
  \frac{d^2}{dx^2}R_l(x)+\frac{2}{x}\frac{d}{dx}R_l(x)+\left(1-\frac{l(l+1)}{x^2}\right)R_l(x)=0
\end{equation}

con soluzione:

\begin{equation}
  R_l(x)=A_j j_l(x)=A_l \frac{\sqrt{\pi}}{2}\sum_{n=0}^{\infty}\frac{(-1)^nx^{2n+l}}{2{2n+l}n!\Gamma(n+l+\frac{3}{2})}
\end{equation}

Per $r\rangle a$ si ha, ponendo $x=i\beta r$:

\begin{equation}
  R_l(r)=Bh_l(i\beta r)
\end{equation}

Esprimendo le condizioni a contorno per le funzioni di Bessel e di Hankel e le loro derivate prime, si ottiene:

\begin{equation}
  \alpha \frac{j'_l(\alpha a)}{j_l(\alpha a)}=\beta \frac{h'_l(i\beta a)}{h_l(i\beta a)}
\end{equation}

La soluzione non è sempre ricavabile analiticamente, ma è possibile analizzare alcuni casi semplici, in funzione di l.

Nel caso di una "onda s" (l=0), si ha:

\begin{equation}
  \psi_0(r)=
  \left\{
  \begin{aligned}
     & A\frac{sin{\alpha r}}{r} \quad se \quad r\langle a \\
     & B\frac{e^{-\beta r}}{r} \quad se \quad r\rangle a
  \end{aligned}
  \right.
\end{equation}

con condizioni a contorno tali che risulti

\begin{equation}
  \left\{
  \begin{aligned}
     & \alpha \cot{\alpha a}=-\beta \leftrightarrow \eta=-\epsilon \cot{\epsilon}
     & \epsilon=\alpha a \quad , \quad \eta=\beta a
     & \epsilon^2+\eta^2=\frac{2mV_0^2}{\hbar^2}a^2
  \end{aligned}
  \right.
\end{equation}

\subsection{Oscillatore armonico isotropo}
Alla luce della trattazione di inizio sezione, si esprime il potenziale effiace come:

\begin{equation}
  V_{eff}(r)=\frac{1}{2}m\omega^2r^2+\frac{\hbar^2l(l+1)}{2mr^2}
\end{equation}

da cui l'equazione radiale di Shrodinger diviene:

\begin{equation}
  R''(r)+\frac{2}{r}R'(r)+\left[\frac{2m}{\hbar^2}\left(E-\frac{1}{2}m\omega^2r^2\right)-\frac{l(l+1)}{r^2}\right]R''(r)=0
\end{equation}

ovvero:

\begin{equation}
  \left\{
  \begin{aligned}
     & \chi''(\rho)+\left(\epsilon -\rho^2-\frac{l(l+1)}{\rho^2}\right)\xi(\rho)=0                                     \\
     & \rho=\alpha rs \quad , \quad \alpha=\sqrt{\frac{m\omega}{\hbar}} \quad , \quad \epsilon=\frac{2E}{\hbar \omega}
  \end{aligned}
  \right.
\end{equation}

da cui:
\begin{equation}
  \chi(\rho)=e^{-\rho^2/2} \rho^l L_n^l(\rho^2)
\end{equation}




\subsection{Moto in un campo centrale}
Si studa qui il moto di un elettrone attorno al nucleo atomico, sotto l'approssimazione di repulsione coulombiana trascurabile, a meno di un fattore medio, e non relativistica.
Lo studio viene facilitato dalle conoscenza apprese sul momento angolare.
L'Hamiltoniana del sistema è data da:

\begin{equation}
  H=\frac{1}{2m}p^2+V(r)
\end{equation}

Si introduce un sistema di coordinate polari, tale che:

\begin{equation}
  \left\{
  \begin{aligned}
     & x=r\sin{\theta}\cos{\phi} \\
     & y=r\sin{\theta}\sin{\phi} \\
     & z=r\cos{\theta}
  \end{aligned}
  \right.
\end{equation}

Ne seguono le regole di commutazione tra le variabili cartesiane ed r:

\begin{equation}
  \left\{
  \begin{aligned}
     & [r, p_x]=\frac{x}{r} \\
     & [r, p_y]=\frac{y}{r} \\
     & [r, p_z]=\frac{z}{r}
  \end{aligned}
  \right.
\end{equation}

Si definisce $p_r$:

\begin{equation}
  p_r=\frac{1}{r}(xp_x+yp_y+zp_z) \rightarrow [r, p_r]=1
\end{equation}

Sfruttando la relazione sul modulo del momento angolare in funzione di k:

\begin{equation}
  k(k+\hbar)=m_x^2+m_y^2+m_z^2
\end{equation}

si ottiene l'Hamiltoniana in polari:

\begin{equation}
  H=\frac{1}{2m}\left ( \frac{1}{r}p_r^2 r +\frac{k(k+\hbar)}{r^2}\right )+V(r)
\end{equation}

Ne segue l'equazione di Shrodinger:

\begin{equation}
  i\hbar \frac{\partial \psi}{\partial t}= \left \{ \frac{\hbar^2}{2m}\left ( -\frac{1}{r}\frac{\partial^2}{\partial r^2}r+\frac{k(k+\hbar)}{\hbar^2r^2} \right ) +V \right \} \psi
\end{equation}

Ne discende la forma dell'operatore lineare funzione di k:

\begin{equation}
  \frac{k(k+\hbar)}{\hbar^2}= -\frac{1}{\sin{\theta}}\frac{\partial}{\partial \theta}\left ( \sin{\theta}\frac{\partial}{\partial \theta} \right ) - \frac{1}{\sin^2{\theta}}\frac{\partial^2}{\partial \phi^2}
\end{equation}

i cui autostati sono le armoniche sferiche, con autovalori del tipo n(n+1). Per un'autofunzione di questo tipo, $\psi$ risulta:

\begin{equation}
  \psi_{k,m}(\xi)=\frac{1}{r}\chi_{k,l}(r)S_n(\theta, \phi)
\end{equation}

Inserendo tale funzione d'onda nell'equazione di Shrodinger, assumendo che lo stato sia stazionario con autovalore di H pari ad H':

\begin{equation}
  H' \chi_0=\left\{\frac{\hbar^2}{2m}\left( -\frac{d^2}{dr^2}+\frac{n(n+1)}{r^2}\right)+V \right\}\chi_0
\end{equation}

Quata equazione è fondamentale per la determinazione dei livelli energetici H' del sistema:

\begin{equation}
  \forall \quad n \quad in \quad N \quad \exists \quad 2n+1 \quad stati \quad indipendenti
\end{equation}


* discussione delle condizioni a contorno *

\subsection{Atomo di idrogeno}
La precedente derivazione è applicabile alla risoluzione del modello dell'atomo di idrogeno:
L'equazione (72) diviene così:

\begin{equation}
  \left\{\frac{d^2}{dr^2} -\frac{n(n+1)}{r^2} +\frac{2me^2}{\hbar^2}\frac{1}{r} \right\}\chi_0= -\frac{2mH'}{\hbar^2}\chi_0
\end{equation}

ponendo:
\begin{equation}
  \left\{
  \begin{aligned}
     &  & \chi_0(r)=f(r)e^{-\frac{r}{a}}     \\
     &  & a=\pm \sqrt{\frac{-\hbar^2}{2mH'}}
  \end{aligned}
  \right.
\end{equation}

l'equazione diviene:

\begin{equation}
  \left(\frac{d^2}{dr^2} -\frac{2}{a}\frac{d}{dr} -\frac{n(n+1)}{r^2} +\frac{2me^2}{\hbar^2}\frac{1}{r}\right)f(r)=0
\end{equation}

Sostituendo la f(r) sviluppata in serie di potenze, si ottiene:

\begin{equation}
  \sum_{s} c_s \{s(s-1)r^{s-2}-(2s/a)r^{s-1}-n(n+1)r^{s-2}+\frac{2me^2}{\hbar^2}r^{s-1} \}=0
\end{equation}

date le condizioni a contorno, si ottengono le energie possibili per l'elettrone nell'atomo di idrogeno:

* da terminare *

\begin{equation}
  H'=-\frac{me^4}{2s^2\hbar^2}
\end{equation}

\subsection{Effetto Zeeman}
L'effetto Zeeman è un effetto di separazione delle righe spettrali di un atomo in presenza di un campo magnetico esterno.
Esso è dovuto al fatto che l'Hamiltoniana del sistema ammette un termine aggiuntivo, dovuto all'interazione tra il momento magnetico dell'elettrone e il campo magnetico esterno,
che tuttavia non influisce sulla forma delle autofunzioni, ma solo sui livelli energetici, ovvero sugli autovalori di H,
che risulteranno aumentati, immutati o diminuiti di un fattore proporzionale al campo magnetico esterno.
Da qui si evincono le ragioni dell'Effetto Zeeman.
* da terminare *


\section{Teoria delle perturbazioni}
L'approccio di analisi di un sistema perturbato parte da una scissione dell'Hamiltoniana in due parti, una parte H$_0$ relativa al sistema non perturbato, e una parte H' relativa alla perturbazione, funzione di una variabile "piccola" $\epsilon$.
* da terminare successivamente *

\section{Particelle identiche}
L'Hamiltoniana per un sistema di particelle identiche deve essere invariante per permutazione degli osservabili corrispondenti.
E' utile dunque introdurre una notazione data dalla produttoria dialica dei ket relativi alle singole particelle, per rappresentare lo stato del sistema.

\begin{equation}
  |a_1\rangle |a_2\rangle ...|a_n\rangle =|a_1,a_2,...,a_n\rangle
\end{equation}

tale rappresentazione è detta "rappresentazione simmetrica".
Introducendo un operatore lineare volto a permutazioni indiciali pari o dispari, P.
Un ket si dice simmetrico se:

\begin{equation}
  P|a_1,a_2,...,a_n\rangle =|a_1,a_2,...,a_n\rangle
\end{equation}

e antisimmetrico se:

\begin{equation}
  P|a_1,a_2,...,a_n\rangle =\pm|a_1,a_2,...,a_n\rangle
\end{equation}
(+ se la permutazione è pari, - altrimenti)

Ne segue che uno stato antisimmetrico è esprimibile mediante una funzione antisimmetrica, e viceversa.
Ciò deve valere ad ogni t.
Un ket antisimmetrico è ottenibile dalla sommatoria estesa sulle permutazioni sugli indici del ket simmetrico, a cui viene applicato l'operatore P, con fattori moltiplicativi $\pm 1$, a
seconda della parità della permutazione.
Conseguentemente, tale ket èesprimibile come il determinante di una matrice ("di Slater") avente per colonne i tensori dello stato simmetrico.
Quindi, dal fatto che colonne uguali determinerebbero un determinante nullo, si deduce che due particelle non possono occupare lo stesso stato quantico.
Tutti gli stati per particelle antisimmetriche devono quindi essere indipendenti.
Inoltre, si deduce che la funzione d'onda di uno stato antisimmetrico è nulla se due particelle si trovano nello stesso stato quantico.
Particelle di questo tipo sono dette "fermioni".
Al contrario, particelle simmetriche sono dette "bosoni".
La funzione d'onda dei fermioni cambia di segno se si scambiano due particelle, mentre quella dei bosoni rimane invariata.
Inoltre, si deduce che la funzione d'onda di uno stato antiimmetrico è nulla se due particelle si trovano nello stesso stato quantico.
Particelle con spin semi-intero sono fermioni, mentre quelle con spin intero sono bosoni.

Quindi, per esempio, due elettroni, protoni o neutroni non possono occupare lo stesso stato quantico, mentre due fotoni possono farlo.

\subsection*{Applicazione agli elettroni}
L'effetto dello spin non è determinante nella variazione dell'Hamiltoniana (l'interazione spin-orbita determina una correzione trascurabile se il campo
magnetico non è intenso), ma è determinante nella variazione della funzione d'onda, in quanto provoca una duplicazione degli stati quantici,
essendo la funzione d'onda fattorizzabile in una parte spazi-temporale e in una parte di spin.
Si introducono due osservabili di permutazione $P^x$ e $P^\sigma$, operanti rispettivamente sulla parte orbitale e su quella di spin di $\psi$.
Formando le permutazioni un gruppo, da una composizione di $P^x$ e $P^\sigma$ si ottiene una permutazione totale $P_a$.

Per ogni stato esistente in natura, per un sistema di particelle identiche, gli autovalori di $P_a$ sono $\pm 1$.

\section{Meccanica quantistica relativistica}
Questa sezione esula dagli argomenti del corso.

Si sviluppa qui una teoria quantistica per gli elettroni invariante per trasformazioni di Lorentz.
Si utilizzerà conseguentemente la notazione indiciale, tale per cui l'impulso, per il principio di covarianza, sarò espresso come:

\begin{equation}
  p_\mu=i\hbar\frac{\partial}{\partial r^{\mu}}
\end{equation}


Per semplicità, si considera c=1 e si sottointende spesso la natura operatoriale delle quantità in questione, in quanto perlopiù deducibile
dal contesto.

\subsection{Equazione di Dirac}


Si consideri il moto di una particella libera, tale per cui l'Hamiltoniana e l'equazione agli autovalori siano date da:

\begin{equation}
  H=\sqrt{p^2c^2+m^2c^4} \rightarrow p_0-\sqrt{m^2c^4+p_1^2+p_2^2+p_3^2}\psi=0
\end{equation}

Tale equazione tiene conto della relazione energia-impulso, ma non è affatto simmetrica per le componenti spaziali e temporali.

Si può facilmente giungere alla seguente forma:

\begin{equation}
  {p_0^2-p_1^2-p_2^2-p_3^2-m^2c^2}\psi=0
\end{equation}

che è relativisticamente invariante, ma non lineare nell'energia ($p_0$), come invece la teoria quantistica richiede, per la linearità dell'equazione di Schrodinger.

Dunque ricerco un'equazione del tipo:

\begin{equation}
  \left\{p_o-\alpha_1p_1-\alpha_2p_2-\alpha_3p_3-\beta\right\}\psi=0
\end{equation}

da cui si ottiene:

\begin{equation}
  \left\{p_0^2-\sum_{123}[\alpha_1^2p_1^2+(\alpha_1\alpha_2+\alpha_2\alpha_1)p_1p_2+(\alpha_1\beta+\beta\alpha_1)p_1]-\beta^2\right\}\psi=0
\end{equation}

da cui risultano le seguenti condizioni per i coefficienti:

\begin{equation}
  \alpha_a\alpha_b+\alpha_b\alpha_a=2\delta_{ab} \quad (a,b=1,2,3 o m)
\end{equation}

Dunque le $\alpha$ anticommutano fra di loro e sono di modulo unitario.

E' possibile esprimere le alpha in termini di matrici di Pauli e matrici $rho$ a 4 componenti:

\begin{equation}
  \alpha_i=\begin{pmatrix}
    0        & \sigma_i \\
    \sigma_i & 0
  \end{pmatrix}
  \quad
  \beta=\begin{pmatrix}
    I & 0  \\
    0 & -I
  \end{pmatrix}
\end{equation}

Naturalmente ora avremo a che fare con una funzione d'onda a 4 componenti.
Si perviene così all'equazione di Dirac:



\section{Bibliografia}
\begin{thebibliography}{9}
  \bibitem{dirac} P.A.M. Dirac, \emph{The Principles of Quantum Mechanics}, Oxford University Press, 1930.
\end{thebibliography}

\end{document}