\documentclass{article}

\usepackage[utf8]{inputenc}
\usepackage{amsmath}
\usepackage{amsfonts}
\usepackage{amssymb}

\title{Meccanica Quantistica}
\author{Cesare Sabattini Manginella}
\date{\today}

\begin{document}

\maketitle
\section{Introduzione}
ATTENZIONE: formulario ancora in fase di sviluppo. Alcune informazioni potrebbero essere incomplete o errate.

\section{Sviluppo del formalismo}

\subsection{Principio di Sovrapposizione}
\subsection{vettori bra e ket}
Data una varietà infinito-dimensionale $V$ su un campo $\mathbb{C}$, un \emph{ket} $<A|$ è un elemento di $V$, mentre un \emph{bra} $|A>$ è un funzionale lineare su $V$.
Tale terminologia è intuitivamente derivante dal fatto che il prodotto scalare è il più generale funzionale lineare definito su uno spazio di Hilbert.
Tale affermazione rappresenta una semplificazione dell'enunciato del Teorema di Rappresentazione di Fisher-Reisz, tale per cui, dato uno spazio di Hilbert H su $\mathcal{C}$:

\begin{equation}
    \forall \phi \in H^* \exists x \in H : \forall y \in H \phi(x)=<x,y>
\end{equation}

In effetti la biunivocità di tale corrispondenza permette la rappresentazione covariante-controvariante dello stesso vettore.
Nell'ambito della meccanica quantistica, si attribuisce la rappresentazione della varietà generata da un ket ad uno stato, che, per il Principio di Sovrapposizione,
risulta combinazione lineare di vettori della base ortonormale che origina H, spazio delle configurazioni (stati) del sistema quantistico.
\footnote[2]{
    Il ket nullo non è associabile a nessuno stato (? o a tutti).
}
Non vi è motivo di discutere la numerabilità dell'insieme di indici non nulli della combinazione, per il teorema di ...

La contrazione degli indici covariante-controvariante del prodotto di bra e l'immaginario coniugato del ket (semplicemente un prodotto interno tra due elementi A e B di H),
è tale da originare un numero complesso.
Si considerano nel seguito prodotti scalari hermitiani, cioè tali che:

\begin{equation}
    <A|B> = \overline{<B|A>}
\end{equation}

\subsection{Spazi Operatoriali}

Si definisce "operatore" un endomorfismo da uno spazio lineare in sè stesso.
Estendendo un'algebra associativa non commutativa su insiemi di operatori, sono definibili spazi operatoriali.

Ragionando in termini di indici covarianti e controvarianti, appare logico che il prodotto di un ket per un bra origini un operatore lineare, operante su bra e su ket.
Secondo tale formalismo, ad un operaotre lineare è fisicamente associata una variabile dinamica, istantaneamente.
Ecco dunque palesato un' altra differenza tra fisica classica e quantistica: la non commutatività degli operatori, ossia delle misure delle variabili dinamiche.


\subsubsection{Operatore aggiunto}
Operatore tale per cui:

\begin{equation}
    <A|\overline{\alpha}|B>= \overline{<B|\alpha|A>}
\end{equation}

Dai corsi precedenti sono note la definizione rigorsosa di tale operatore e le sue proprietà.
$\alpha$ si dice autoaggiunto o reale se
\begin{equation}
    \alpha=\overline{\alpha}
\end{equation}

TEOREMA:


\begin{equation}
    \forall \psi \in \mathcal{L}(H) : \psi^m <P|=0 , \psi <P|=0
\end{equation}

Intuitivamente, ciò può essere desunto dalla convergenza in norma della serie operatoriale:
la convergenza in norma implica la convergenza forte all'operatore nullo. Avendo a che fare con un m $\in N$, e trattando
operatori lineari su un dominio ovunque denso e semplicemente connesso, la norma del ket esito dell'applicazione è minorabile
con il prodotto della norma dell'operatore per quella del ket. Non avendo a che fare con serie infinite, non è plausibile che la produttoria
di operatori di modulo diverso da zero m volte sia pari a 0. Dunque $\psi$ deve essere necessariamente l'operatore nullo.

\subsubsection{Autovalori e autovettori}

$|\xi>$ è autoket di $\xi$ relativo all'autovalore $\xi'$ \footnote{
    ma che è sta robaccia
} se:

\begin{equation}
    \alpha |P>= a |P>
\end{equation}

Analogamente si definiscono autobra.

Segue che, riferendosi ad operatori reali:
\begin{itemize}
    \item gli autovalori sono reali.
    \item gli autovalori associati agli autoket sono gli stessi associati agli autobra.
    \item l'immaginario coniugato di un autoket è un autobra riferito allo stesso autovalore, e viceversa.
\end{itemize}

Nell'ambito del formalismo della meccanica quantistica un autoket è detto autostato del sistema rispetto alla variabile dinamica $\alpha$, operatore.
Non riporterò di seguito ulteriori proprietà relative ad autovalori ed autovettori, in quanto già note dai corsi precedenti.

\section{Osservabili}
Se il sistema dinamico si trova in un autostato rispetto alla variabile dinamica $\xi$, allora una misura di tale variabile darà come esito
$\xi'$. Viceversa se ogni misura riporta come esito $\xi'$, allora il sistema si trova in un autostato rispetto a $\xi$.
Importante è osservare che $\xi'$, essendo reale, risulta compatibile con la necessità di effettuare un'unica misura per rilevarlo:
se fossero multiple, ognuna perturberebbe le successive.
Ovviamente $\xi'$ si otterrà per ogni sovrapposizione di autostati ad esso relativi.
Per entrare nell'ottica della teoria, possiamo aspettarci cosa possa accadere nel caso di uno stato esito di sovrapposizione di autostati relativi ad autovalori differenti,
dunque ortogonali (da qui la quantizzazione dell'esito della misura (terminologia discutibile?)).
Dunque l'esito di una successiva misura sarà uguale a quello della precedente, essendo il sistema collassato nell'autostato relativo allo scalare misurato nella misura 1.
Dunque l'insieme degli autovalori di un operatore lineare non è che l'insieme dei risultati ottenibili attraverso una misura.
Dato che uno stato appartiene ad un insieme completo generato dalla base di autostati, si definisce osservabile una variabile dinamica i cui autostati formano un sistema completo.
Nella pratica un osservabile è una quantità misurabile, in quanto solo i sistemi completi rendono misure.
Dunque l'insieme delle misure possibili di una variabile dinamica è l'insieme degli autovalori associati al corrispettivo operatore lineare.

La completezza di uno spazo generato da un sistema di autoket è tale che ogni $|P>$ possa essere espresso come

\begin{equation}
    |P>= \int{|\xi'c>d\xi'}+ \sum_{r}|\xi'>
\end{equation}
Per via del fatto che la cardinalità di base è non necessariamente numerabile.
Ogni operatore soddisfacente un'equazione algebrica è un osservabile.

\subsection{Funzioni di Osservabili}

Si definisce funzione di osservabili:
\begin{equation}
    f(\xi): L(H) \rightarrow L(H) : f(\xi)|\xi'>= f(\xi')|\xi'>, \forall |\xi'> autostato \quad di \quad  \xi
\end{equation}

In questo modo si definisce una funzione di osservabile f, sotto la sola ipotesi di inclusione degli autovalori nel dominio della
funzione reale, tale che gli autovalori di $f(\xi)$ siano uguali all'immagine degli autovalori di $\xi$ tramite f, con f iniettiva.

\subsubsection{Inverso di un osservabile}
L'inverso di un osservabile $\xi$ esiste se e solo se 0 non è autovalore di $\xi$, ed è definito da:

\begin{equation}
    \xi^{-1} |\xi'>= \xi'^{-1}|\xi'>
\end{equation}

ovvero $\xi$ tale che:

\begin{equation}
    \xi\xi^{-1}=\mathcal{I}
\end{equation}

\subsubsection{Radice di un osservabile}

Si definisce radice di un osservabile un operatore tale che:

\begin{equation}
    \sqrt{\xi} |\xi'>= \pm \sqrt{\xi'} |\xi'>
\end{equation}

ovvero tale che

\begin{equation}
    \sqrt{\xi}\sqrt{\xi}=\xi
\end{equation}


\subsection{Interpretazione Fisica}
Si ipotizza che il valor medio delle misure di un osservabile $\xi$ su un sistema dinamico in uno stato $/x>$
assuma un valore medio pari a $<x|\xi|x>$.
Tale interpretazione fisica è formalizzata nell'ambito della rappresentazione degli stati di un sistema quantistico,
su basi ortonormali di cardinalità finita o infinita, numerabili o meno, tale per cui risulta, come analizzato nel seguito:

\begin{equation}
    \left\{
    \begin{aligned}
    & |x>= \sum_{0}^{v} <x|\xi_i>|\xi_i>+ \int_{0}^{v} <x|\xi'>|\xi'>d\xi' \\
    & <\xi' | \xi''>= \delta(\xi'-\xi'')
    \end{aligned}
    \right.
\end{equation}

ove $\delta$ è la delta di Dirac, e v è la cardinalità della base ortonormale numerabile. $\xi_i$ sono gli autovalori dell'osservabile $\xi$, rispetto a quali si costruisce la base degli autostati.
Si sottointende la complementarietà dello spettro continuo rispetto al numerabile nell'ambito della rappresentazione dello spazio degli stati.


\subsection{Commutazione di osservabili}
Gli operatori non sono in generale commutativi.
Si definisce pertanto il commutatore di due osservabili $\xi$ e $\eta$ l'operatore:

\begin{equation}
    [\xi,\eta]=\xi\eta-\eta\xi
\end{equation}

fisicamente, il commutatore ritorna la possibilità di misurare contemporaneamente due variabili dinamiche.
Matematicamente, ciò equivale ad affermare che i due osservabili formano lo stesso sistema completo di autostati.
Se due osservabili commutano, allora esiste una base di autostati comuni ad entrambi.
Vale anche l'inverso del teorema.
Dunque, data una funzione di osservabili che commutano, la densità di misurare un set di autovalori è data dalla produttoria:

\begin{equation}
  P_{a,b,...,n}=<x|\delta_{\xi,a}\delta_{\eta,b}...\delta_{\nu,n}|x>
\end{equation}

Solo se le variabili commutano talee probabilità ha un senso fisico.


\section{Rappresentazioni}
Rappresentare una grandezza equivale alla ricerca delle coordinate su una data base dello spazio di Hilbert in cui essa esiste.
La più semplice delle basi su cui operare, nell'ambito di un osservabile $\xi$, è data dai sui autostati, normalizzati.
La condizione di normalizzazione, nel più generale dei casi, equivalente alla composizione di uno spettro continuo con uno discreto, è data da:

\begin{equation}
<\xi'|\xi''> = \delta(\xi'-\xi'') + \delta_{\xi',\xi''}
\end{equation}

ove $\delta$ è da intendere come un funzionale astratto che restituisce 1 se i due argomenti sono uguali, 0 altrimenti.
Nel caso di uno spettro continuo, la delta di Dirac è da intendere come un funzionale tale che:

\begin{equation}
\int_{0}^{v} \delta(\xi'-\xi'')d\xi' = 1
\end{equation}

dunque, il simbolo $\delta$ nel caso di uno spettro continuo, rappresenta, con un abuso di notazione, una generalizzazione della delta di Kronecker per uno spettro discreto.
Data tale assunzione, si possono definire le seguenti rappresentazioni.
Generalizzando tale concetto a varietà di dimensionalità superiore, corrispondenti cioè a spazi di Hilbert generati da prodotto dialico
di autovettori di autovettori corrispondenti ad un set di n osservabili commutativi, si può scrivere la condizione di ortogonalità come:

\begin{equation}
    <\xi_1,\xi_2,...,\xi_n|\xi'_1,\xi'_2,...,\xi'_n> = \delta(\xi_1-\xi'_1)\delta(\xi_2-\xi'_2)...\delta(\xi_n-\xi'_n)
\end{equation}

\subsection{Rappresentazione di Bra e Ket}
Bra e Ket rappresentano degli stati di un sistema quantistico, dunque, per il principio di sovrapposizione,
rispetto ad una base ortonormale di H, di autovettori relativi ad un osservabile $\xi$, sono rappresentabili come
una generalizzazione a dimensione potenzialmente infinita dei componenti della base.
Nel caso di un ket, si ha:

\begin{equation}
    |P>= \sum_{0}^{v} <P|\xi_i>|\xi_i>+ \int <P|\xi'>|\xi'>d\xi'
\end{equation}

dove si sottointende il dominio di integrazione sulla componente continua dello spettro.
Data la linearità dell'espressione e del prodotto scalare, si può scrivere:

\begin{equation}
    <Q|P>= \sum_{0}^{v} <Q|\xi_i><\xi_i|P>+ \int <Q|\xi'><\xi'|P>d\xi'
\end{equation}


\subsection{Rappresentazione di un osservabile}
Il caso di rappresentazione di un osservabile introduce un indice di variabilità tensoriale pari a 2, in quanto si ha a che fare con un operatore lineare, operante su un ket e su un bra.
Alla luce di ciò, si può scrivere:

\begin{equation}
    \xi = \sum_{0}^{v} \xi_i|\xi_i><\xi_i|+ \int \xi'|\xi'><\xi'|d\xi'
\end{equation}

o, analogamente:

\begin{equation}
    <\xi' | \xi | \xi''>= \xi' \delta(\xi'-\xi'')
\end{equation}

definendo così la generalizzazione di una matrice diagonale nel caso di uno spettro continuo.

\subsection{Rappresentazione generica}

Nel caso di uno spazio di Hilbert generato dagli autovalori di n osservabili commutative, definisco la rappresentazione per 
un tensore di rango n $\omega$ come:

\begin{equation}
    <\xi_1,\xi_2,...,\xi_n|\omega|\xi'_1,\xi'_2,...,\xi'_n> = \omega' \delta(\xi_1-\xi'_1)\delta(\xi_2-\xi'_2)...\delta(\xi_n-\xi'_n)
\end{equation}


\subsection{Applicazione delle rappresentazioni alla probabilità di una misura}
Nel caso di una misura di n set di n osservabili commutative, la probabilità di ottenere un set di autovalori $\xi_1,\xi_2,...,\xi_n$ è data da:

\begin{equation}
    P_{\xi_1,\xi_2,...,\xi_n} = <x|\delta_{\xi_1,\xi_1}\delta_{\xi_2,\xi_2}...\delta_{\xi_n,\xi_n}|x>
\end{equation}

si trova immediatamente che la probabilità di una misura non è altro che il modulo quadro dello stato a cui si applica l'operatore set di autovalori in questione.
Tale concetto è facilmente estendibile ad un operatore con uno spettro continuo definendo un'ampiezza di probabilità, per esempio per una misura di posizione.

\section{Funzione d'onda come rappresentazione di ket}
Una generica funzione di osservabili può rappresentarre un ket in H, ovvero uno stato variabile dinamicamente nel contesto di una sistema quantistico.
Dunque il modulo quadro di una funzione d'onda, a cui si applica un osservabile, rappresenta la probabilità di misurare un set di autovalori di tale osservabile.
Nella notazione, si sottointende da ora:

\begin{equation}
    \psi = |\psi>
\end{equation}

dunque, sottointendendo ora la presenza del ket standard >, si desume che la differenza tra una generica funzione di osservabili f e $\psi$ è che
quest'ultima può essere applicata ad un operatore solo a sinistra, mentre f da entrambi i lati.
Fisicamente, f non rappresenta lo stato del sistema quantistico, mentre $\psi$ sì.
La normalizzazione di $\psi$ consente di interpretare il modulo quadro come probabilità di misurare il sistema in dato stato.


\section{Condizioni quantiche}
Si definisce condizione quantica una relazione tra due osservabili, volte a descrivere il sistema quantistico.

\subsection{Parentesi di Poisson quantistiche}
Date le variabili canoniche, si definiscono le parentesi di Poisson classiche come:

\begin{equation}
    [u,v]_P = \sum_{i=1}^{n} \frac{\partial u}{\partial q_i}\frac{\partial v}{\partial p_i} - \frac{\partial u}{\partial p_i}\frac{\partial v}{\partial q_i}
\end{equation}

Servendosi del principio di corrispondenza, si instaura la corrispondenza tra esse e l'operazione di commutazione operatoriale:

\begin{equation}
    [u,v]_P = \frac{1}{i\hbar}[\hat{u},\hat{v}]
\end{equation}

\subsection{Rappresentazione di Schrödinger}
La rappresentazione di Shrodinger consiste nel formalizzare gli operatori posizione e quantità di moto come operatori differenziali sullo spazio degli stati variabili,
ovvero delle funzioni d'onda su uno spazio di Hilbert.
In tal modo, si definisce l'operatore posizione come:

\begin{equation}
    \hat{q} = q
\end{equation}

e l'operatore quantità di moto come:

\begin{equation}
    \hat{p} = -i\hbar \nabla
\end{equation}

In tale rappresentazione, come affermato nel paragrafo precedente ... , la densità di probabilità di misurare una variabile dinamica è data dal modulo quadro della funzione d'onda, ma solo
nel caso in cui la funzione d'onda normalizzata sia autostato dell'operatore.
Alternativamente, essendo essa ket di H, è esprimibile come combinazione lineare generalizzata degli autostati dell'operatore,
per cui è possibile calcolare la probabilità di misurare una variabile dinamica come il modulo quadro della proiezione della funzione d'onda su un autostato dell'operatore.

\subsection{Principio di Indeterminazione di Heisenberg}
Il principio di indeterminazione di Heisenberg afferma che non è possibile misurare contemporaneamente con precisione arbitraria due variabili dinamiche coniugate:

\begin{equation}
    \Delta q \Delta p \geq \frac{\hbar}{2}
\end{equation}

La sua espressione generalizzata afferma che, date due osservabili non commutative, si può affermare che:

\begin{equation}
    \Delta \xi \Delta \eta \geq \frac{1}{2i}\int <\psi|[\hat{\xi},\hat{\eta}]|\psi>d\xi
\end{equation}


\section{Equazioni del moto}
Dato un generico ket P$>$, si definisce l'operatore di traslazione temporale T:

\begin{equation}
    \begin{aligned}
    & T |P>= |P(t)>
    \end{aligned}
\end{equation}

sotto l'ipotesi di invarianza del modulo di P per traslazione temporale (che equivale ad un'equiprobabilità di uno stato di moto imperturbato),
si deduce che T sia un operatore unitario.
Calcolo:

\begin{equation}
     \frac{dP(t_0)}{dt}=\lim_{ t \rightarrow t_0} \frac{|T(t)P>-|P(t_0)>}{t-t_0}= \lim{\frac{T-Id}{t-t_0}P(t_0)}|P(t_0)> \\
\end{equation}

Pongo tale operatore immaginario puro pari ad H diviso per i$\hbar$, ottenendo:

\begin{equation}
   i\hbar \frac{dP(t_0)}{dt}=H(t)|P(t_0)>
\end{equation}

dove H è l'operatore Hamiltoniano, che rappresenta l'energia del sistema.
In termini operatoriali è equivalentemente esprimibile come:

\begin{equation}
    i\hbar \frac{d}{dt}T>=H(t)T
\end{equation}

\subsection{Equazione di Schrödinger}
Si giunge all'equazione di Shroedinger ponendo il generico ket P$>$ come funzione d'onda $\psi(\xi, t)$, funzione di osservabili operante sul ket standard.
Dunque, l'equazione di Schrödinger è:

\begin{equation}
    i\hbar \frac{\partial \psi(\xi, t)}{\partial t} = H(t) \psi(\xi,t)
\end{equation}

dove l'operatore $\xi$ è un generico osservabile, dipendente dallo spazio in cui si opera (spazio delle configurazioni, spazio delle impulsi, spazio delle fasi, etc.).

\subsection{Forma di Heisenberg delle equazioni del moto}
Se, a partire dalla rappresentazione di Schrödinger delle equazioni del moto, si effettua una trasformazione inversa $T^{-1}$ implicita sullo spazio dei ket
e sugli operatori in esso agenti, si ottiene la forma di Heisenberg delle equazioni del moto, che se ne differenzia per staticità degli stati e variazione temporale degli operatori.
Dunque, nello schema di Heisenberg si otterranno delle equazioni del moto per gli operatori, e non per gli stati.
Le equazioni del moto di Heisenberg sono:

\begin{equation}
   \frac{d\xi}{dt}=i\hbar[\xi,H(t)]
\end{equation}

in perfetta analogia con le equazioni del moto classiche, secondo il principio di corrispondenza tra parentesi di Poisson e commutatori operatoriali.
Si deduce immediatamente che $\xi$ è una costante del moto se e solo se commuta con l'Hamiltoniano. In tal caso, si dice che $\xi$ è un integrale primo del moto.
In meccanica quantistica, nello schema di Heisenberg, se un operatore è il trasformato di una costante del moto classica, allora esso commuta con l'Hamiltoniano(?).

\subsection{Stati stazionari}
A partire dall'equazione differenziale operatoriale 35, si può esprimere l'operatore T come:

\begin{equation}
    T=e^{-\frac{i}{\hbar}\int_{t_0}^{t}H(t')dt'}=e^{-\frac{i}{\hbar}Ht}
\end{equation}
dove la seconda uguaglianza vale naturalmente per stati ad energia costante.
Così, nelle due rappresentazioni, si può scrivere:

\begin{equation}
    \left\{
    \begin{aligned}
    & \psi(\xi,t)=e^{-\frac{i}{\hbar}Ht}\psi(\xi,0) \\
    & \xi(t)=e^{\frac{i}{\hbar}Ht}\xi(0)e^{-\frac{i}{\hbar}Ht}
    \end{aligned}
    \right.
\end{equation}

Tali equazioni, per quanto esplicative e semplici, non sono utili nella pratica, per via della difficoltà nel calcolare l'esponenziale di un operatore.
Tuttavia, nella rappresentazione di Shrodinger, l'operazione di calcolo delle equazioni del moto se H è costante risulta immediata, nel caso in cui
lo stato sia autostato dell'Hamiltoniano, con autovalore E. In tal caso, si ha:

\begin{equation}
    \begin{aligned}
    \psi(\xi,t)=e^{-\frac{i}{\hbar}Et}\psi(\xi,0)
    \end{aligned}
\end{equation}

Dunque ad H costante, un ket differisce solamente per un fattore di fase.
Fisicamente, ciò implica un'equiprobabilità dello stato al variare del tempo, e dunque una stazionarietà dello stato, in quanto il fattore di fase non varia il modulo quadro della funzione d'onda.

\subsection{Equazioni del moto per una particella libera}
Nel caso di velocità relativistiche, l'Hamiltoniana classica corrispondente a tale stato stazionario è data da:

\begin{equation}
    H= \sqrt{p^2c^2+m^2c^4}
\end{equation}

dato che p$_x$, operatore quantità di moto, commuta con qualsiasi funzione di osservabili p$_x$, p$_y$, p$_z$, esso è  una costante del moto
anche nel sistema quantistico. Analogamente per p$_y$ e p$_z$.
Nella rappresentazione di Heisenberg, si ottiene:

\begin{equation}
    \dot{x}=\frac{c^2 p_x}{H}, \dot{y}=\frac{c^2 p_y}{H}, \dot{z}=\frac{c^2 p_z}{H}
\end{equation}

Nella rappresentazione di Shrodinger, cerco l'equazione di evoluzione di un autostato dell'Hamiltoniano, ossia di uno stato stazionario.
Si ottiene l'equazione di un'onda piana:

\begin{equation}
    \psi(x,t)=e^{\frac{i}{\hbar}(<p\cdot x>-Et)}
\end{equation}

\section{Applicazioni elementari}
\subsection{Oscillatore armonico}
La trattazione matematica dell'oscillatore armonico quantistico conduce direttamente alla quantizzazione dell'energia per tale sistema.
Di seguito riporto gli step logici che possono essere seguiti:

\begin{itemize}
    \item Si scrive l'Hamiltoniano dell'oscillatore armonico, mediante il principio di corrispondenza, come:
    
    \begin{equation}
        \hat{H}=\frac{1}{2m}(\hat{p}^2+m^2\omega^2\hat{q}^2)
    \end{equation}
    \item Si scrivono le equazioni del moto nella rappresentazione di Heisenberg nelle coordinate canoniche:
    
    \begin{equation}
        \left\{
        \begin{aligned}
            & \dot{q}=\frac{1}{m}\hat{p} \\
            & \dot{p}=-m\omega^2\hat{q}
        \end{aligned}
        \right.
    \end{equation}

    \item Si introduce la variabile dinamica complessa $\eta$ tale che:
    
    \begin{equation}
    \eta= \frac{\hat{p}+im\omega \hat{q}}{\sqrt{2m\hbar \omega}}
    \end{equation}

    e se ne ricava l'equazione del moto:

    \begin{equation}
     \hat{\eta}=\hat{\eta_0} e^{i\omega t}
    \end{equation}

    \item Si ricavano i commutatori tra $\hat{\eta}$ e $H$:
    
    \begin{equation}
        \left\{
        \begin{aligned}
            & [\hat{\eta},\hat{\eta}^{\dagger}]=1 \\
            & [\hat{\eta},H]=\hbar \omega \hat{\eta}
        \end{aligned}
        \right.
    \end{equation}

    \item Considerando H' autovalore di $\hat{H}$, si scrive:
    
    \begin{equation}
        \hbar \omega<H' |\hat{\eta}\hat{\eta}^{\dagger}|H'>=(H'-\frac{1}{2}\hbar \omega)<H' |H'>
    \end{equation}

    che impone che un generico autovalore H' sia tale che:

    \begin{equation}
        H'\geq \frac{1}{2}\hbar \omega
    \end{equation}

    \item Dalle proprietà dei commutatori si ricava la serie di autovalori assumibili da H:
    
    \begin{equation}
        H_n=(n+\frac{1}{2})\hbar \omega \quad , \quad n \in N
    \end{equation}

    \item Si può procedere nel trovare la rappresentazione di Schrödinger dell'oscillatore armonico.
\end{itemize}

\subsection{Momento angolare}
\subsubsection{Momento angolare orbitale}
In termini operatoriali, tramite il principio di corrispondenza, il momento angolare orbitale è rappresentabile come:

\begin{equation}
    \hat{M}=\hat{x}\times \hat{p}
\end{equation}

dove r e p sono rispettivamente gli operatori posizione e quantità di moto.
Definendo un operatore di rotazione intorno all'asse z $r_z$, M si può scrivere come:

\begin{equation}
    \hat{M}= i\hbar r_z
\end{equation}

\subsubsection{Momento angolare di spin}

\subsubsection{Quantizzazione del momento angolare}
Si può procedere in maniera analoga alla quantizzazione dell'oscillatore armonico, considerando l'Hamiltoniano del sistema come:




\section{Bibliografia}
\begin{thebibliography}{9}
    \bibitem{dirac} P.A.M. Dirac, \emph{The Principles of Quantum Mechanics}, Oxford University Press, 1930.
\end{thebibliography}

\end{document}