\documentclass{article}

\usepackage[utf8]{inputenc}
\usepackage{amsmath}
\usepackage{amsfonts}
\usepackage{amssymb}

\title{Meccanica Quantistica}
\author{Cesare Sabattini Manginella}
\date{\today}

\begin{document}

\maketitle
\section{Introduzione}
NOTA IMPORTANTE: formulario in via di composizione. Rigore e sintesi sono ancora da ottenere.

\section{Sviluppo del formalismo}

\subsection{Principio di Sovrapposizione}
\subsection{vettori bra e ket}
Data una varietà infinito-dimensionale $V$ su un campo $\mathbb{C}$, un \emph{ket} $<A|$ è un elemento di $V$, mentre un \emph{bra} $|A>$ è un funzionale lineare su $V$.
Tale terminologia è intuitivamente derivante dal fatto che il prodotto scalare è il più generale funzionale lineare definito su uno spazio di Hilbert.
Tale affermazione rappresenta una semplificazione dell'enunciato del Teorema di Rappresentazione di Fisher-Reisz, tale per cui, dato uno spazio di Hilbert H su $\mathcal{C}$:

\begin{equation}
    \forall \phi \in H^* \exists x \in H : \forall y \in H \phi(x)=<x,y>
\end{equation}

In effetti la biunivocità di tale corrispondenza permette la rappresentazione covariante-controvariante dello stesso vettore.
Nell'ambito della meccanica quantistica, si attribuisce la rappresentazione della varietà generata da un ket ad uno stato, che, per il Principio di Sovrapposizione,
risulta combinazione lineare di vettori della base ortonormale che origina H, spazio delle configurazioni (stati) del sistema quantistico.
\footnote[2]{
    Il ket nullo non è associabile a nessuno stato (? o a tutti).
}
Non vi è motivo di discutere la numerabilità dell'insieme di indici non nulli della combinazione, per il teorema di ...

La contrazione degli indici covariante-controvariante del prodotto di bra e l'immaginario coniugato del ket (semplicemente un prodotto interno tra due elementi A e B di H),
è tale da originare un numero complesso.
Si considerano nel seguito prodotti scalari hermitiani, cioè tali che:

\begin{equation}
    <A|B> = \overline{<B|A>}
\end{equation}

\subsection{Spazi Operatoriali}

Si definisce "operatore" un endomorfismo da uno spazio lineare in sè stesso.
Estendendo un'algebra associativa non commutativa su insiemi di operatori, sono definibili spazi operatoriali.

Ragionando in termini di indici covarianti e controvarianti, appare logico che il prodotto di un ket per un bra origini un operatore lineare, operante su bra e su ket.
Secondo tale formalismo, ad un operaotre lineare è fisicamente associata una variabile dinamica, istantaneamente.
Ecco dunque palesato un' altra differenza tra fisica classica e quantistica: la non commutatività degli operatori, ossia dell'azione delle variabili dinamiche(?).


\subsubsection{Operatore aggiunto}
Operatore tale per cui:

\begin{equation}
    <A|\overline{\alpha}|B>= \overline{<B|\alpha|A>}
\end{equation}

Dai corsi precedenti sono note la definizione rigorsosa di tale operatore e le sue proprietà.
$\alpha$ si dice autoaggiunto o reale se
\begin{equation}
    \alpha=\overline{\alpha}
\end{equation}

TEOREMA:


\begin{equation}
    \forall \psi \in \mathcal{L}(H) : \psi^m <P|=0 , \psi <P|=0
\end{equation}

Intuitivamente, ciò può essere desunto dalla convergenza in norma della serie operatoriale:
la convergenza in norma implica la convergenza forte all'operatore nullo. Avendo a che fare con un m $\in N$, e trattando
operatori lineari su un dominio ovunque denso e semplicemente connesso, la norma del ket esito dell'applicazione è minorabile
con il prodotto della norma dell'operatore per quella del ket. Non avendo a che fare con serie infinite, non è plausibile che la produttoria
di operatori di modulo diverso da zero m volte sia pari a 0. Dunque $\psi$ deve essere necessariamente l'operatore nullo.

\subsubsection{Autovalori e autovettori}

$|\xi>$ è autoket di $\xi$ relativo all'autovalore $\xi'$ \footnote{
    ma che è sta robaccia
} se:

\begin{equation}
    \alpha |P>= a |P>
\end{equation}

Analogamente si definiscono autobra.

Segue che, riferendosi ad operatori reali:
\begin{itemize}
    \item gli autovalori sono reali.
    \item gli autovalori associati agli autoket sono gli stessi associati agli autobra.
    \item l'immaginario coniugato di un autoket è un autobra riferito allo stesso autovalore, e viceversa.
\end{itemize}

Nell'ambito del formalismo della meccanica quantistica un autoket è detto autostato del sistema rispetto alla variabile dinamica $\alpha$, operatore.
Non riporterò di seguito ulteriori proprietà relative ad autovalori ed autovettori, in quanto già note dai corsi precedenti.

\section{Osservabili}
Se il sistema dinamico si trova in un autostato rispetto alla variabile dinamica $\xi$, allora una misura di tale variabile darà come esito
$\xi'$. Viceversa se ogni misura riporta come esito $\xi'$, allora il sistema si trova in un autostato rispetto a $\xi$.
Importante è osservare che $\xi'$, essendo reale, risulta compatibile con la necessità di effettuare un'unica misura per rilevarlo:
se fossero multiple, ognuna perturberebbe le successive.
Ovviamente $\xi'$ si otterrà per ogni sovrapposizione di autostati ad esso relativi.
Per entrare nell'ottica della teoria, possiamo aspettarci cosa possa accadere nel caso di uno stato esito di sovrapposizione di autostati relativi ad autovalori differenti,
dunque ortogonali (da qui la quantizzazione dell'esito della misura (terminologia discutibile?)).
Dunque l'esito di una successiva misura sarà uguale a quello della precedente, essendo il sistema collassato nell'autostato relativo allo scalare misurato nella misura 1.
Dunque l'insieme degli autovalori di un operatore lineare non è che l'insieme dei risultati ottenibili attraverso una misura.
Dato che uno stato appartiene ad un insieme completo generato dalla base di autostati, si definisce osservabile una variabile dinamica i cui autostati formano un sistema completo.
Nella pratica un osservabile è una quantità misurabile, in quanto solo i sistemi completi rendono misure.
Dunque l'insieme delle misure possibili di una variabile dinamica è l'insieme degli autovalori associati al corrispettivo operatore lineare.

La completezza di uno spazo generato da un sistema di autoket è tale che ogni $|P>$ possa essere espresso come

\begin{equation}
    |P>= \int{|\xi'c>d\xi'}+ \sum_{r}|\xi'>
\end{equation}
Per via del fatto che la cardinalità di base è non necessariamente numerabile.
Ogni operatore soddisfacente un'equazione algebrica è un osservabile.

\subsection{Funzioni di Osservabili}

Si definisce funzione di osservabili:
\begin{equation}
    f(\xi): L(H) \rightarrow L(H) : f(\xi)|\xi'>= f(\xi')|\xi'>, \forall |\xi'> autostato \quad di \quad  \xi
\end{equation}

In questo modo si definisce una funzione di osservabile f, sotto la sola ipotesi di inclusione degli autovalori nel dominio della
funzione reale, tale che gli autovalori di $f(\xi)$ siano uguali all'immagine degli autovalori di $\xi$ tramite f, con f iniettiva.

\subsubsection{Inverso di un osservabile}
L'inverso di un osservabile $\xi$ esiste se e solo se 0 non è autovalore di $\xi$, ed è definito da:

\begin{equation}
    \xi^{-1} |\xi'>= \xi'^{-1}|\xi'>
\end{equation}

ovvero $\xi$ tale che:

\begin{equation}
    \xi\xi^{-1}=\mathcal{I}
\end{equation}

\subsubsection{Radice di un osservabile}

Si definisce radice di un osservabile un operatore tale che:

\begin{equation}
    \sqrt{\xi} |\xi'>= \pm \sqrt{\xi'} |\xi'>
\end{equation}

ovvero tale che

\begin{equation}
    \sqrt{\xi}\sqrt{\xi}=\xi
\end{equation}


\subsection{Interpretazione Fisica}
Si ipotizza che il valor medio delle misure di un osservabile $\xi$ su un sistema dinamico in uno stato $/x>$
assuma un valore medio pari a $<x|\xi|x>$.








\section{Principi Fondamentali}
\section{Bibliografia}
\begin{thebibliography}{9}
    \bibitem{dirac} P.A.M. Dirac, \emph{The Principles of Quantum Mechanics}, Oxford University Press, 1930.
\end{thebibliography}

\end{document}