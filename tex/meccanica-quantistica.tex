\documentclass{article}

\usepackage[utf8]{inputenc}
\usepackage{amsmath}
\usepackage{amsfonts}
\usepackage{amssymb}
\usepackage{graphicx}
\usepackage{hyperref}
\usepackage{tcolorbox}

\hypersetup{
    colorlinks=true,      
    linkcolor=black,        
    citecolor=blue,        
    urlcolor=blue,         
    pdfborder={0 0 0},     
}

\title{Meccanica Quantistica}
\author{Cesare Sabattini Manginella}
\date{\today}

\begin{document}

\maketitle

\tableofcontents

\section{Introduzione}
ATTENZIONE: formulario ancora in fase di sviluppo. Alcune informazioni potrebbero essere incomplete o errate.
Inoltre, è in corso la riscrittura in inglese.

\section{Sviluppo del formalismo}

\subsection{Principio di Sovrapposizione}
\subsection{Bra an Ket}

Given an infinite dimensional curvature-less and tortion-less manifold, one defines a ket $<A|$ as an element of $V$, while a bra $|A>$ is a linear functional on $V$.
The notation embodies the theorem:

\begin{tcolorbox}[colframe=gray!50, colback=gray!10, coltitle=black, title=Fisher Reisz Representation Theorem]
    Being  $H$ over $\mathcal{C}$,

    \begin{equation}
        \forall \phi \in H^* \exists x \in H : \forall y \in H \phi(x)=<x,y>
    \end{equation}
\end{tcolorbox}

meaning that the biunivocal correspondence between a ket and a bra allows for the covariant-contravariant representation of the same vector.
In quantum mechanics, as it'll be shown in the following, the ket is associated to a state, while the bra is associated to the dual state, or the complex conjugate of the state:
this implies, due to the linearity of the Hilbert space, that the most general state can be expressed as a linear combination of the basis vectors of the Hilbert space, meaning
the ket that diagonalize a certain operator.

\subsection{Operatorial spaces}

Given a linear space, an operator is defined as an endomorphism on the space itself.
Extending a non-commutative associative algebra on sets of operators, operatorial spaces can be defined.

Operators in quantum mechanics represent dynamical variables: if they are self-adjoint, they represent observables.

Thinking in terms of covariant and contravariant indices, it appears logical that the product of a ket by a bra originates an observable, acting on bras and kets:
this is fundamental to understand the possibility to diagonalize an Hilbert space in terms of the eigenstates of a set of operators.

Since, in general, operators do not commute, physical measurements can permanently alter the state of the system, as the system collapses in the eigenstate of the measured operator.

\section{Observables}

\begin{tcolorbox}[colframe=gray!50, colback=gray!10, coltitle=black, title=Definition]
    An observable $\xi$ is a self-adjoint linear operator on a Hilbert space, whose eigenstates form a complete system, meaning that:

    \begin{equation}
        \text{$|P\rangle$ = $\sum_{0}^{v} <P|\xi_i>|\xi_i> + \int_{0}^{v} <P|\xi'>|\xi'>d\xi'$  $\forall |P\rangle \in H$}
    \end{equation}
\end{tcolorbox}

The request of self-adjointness is necessary to ensure that the eigenvalues of the operator are real, meaning they do represent a unique physical quantity.
Indeed, their complexity would imply the possibility of multiple measurements, each potentially perturbing the following one.

Therefore, if the system is in an eigenstate of an observable, the outcome of a measurement of that observable will be the eigenvalue associated to that eigenstate.

Moreover, if an operator is an observable, the Hilbert space can be diagonalized in terms of the eigenstates of the operator, and a general state,
for the principle of superposition, can be expressed as a linear combination of the eigenstates of the operator.

\subsection{Functions of observables}

\begin{tcolorbox}[colframe=gray!90, colback=blue!5, coltitle=white, title= \textbf{Definition}: Function of Observables, fonttitle=\large]
    A function of an observable $\xi$ is defined as:
    \begin{equation}
        \text{$(f(\xi): L(H) \rightarrow L(H))$ : $f(\xi)|\xi'>= f(\xi')|\xi'>$ , $\forall |\xi'>$}
    \end{equation}
\end{tcolorbox}

\subsection{Physical interpretation of observables}

The average value of an observable $\xi$ on a system in a state $|x>$ is expected to be equal to $<x|\xi|x>$.
Since an observable form a complete system, and thestate can be expressed as a superposition of the eigenstates of the observable,
this a

Si ipotizza che il valor medio delle misure di un osservabile $\xi$ su un sistema dinamico in uno stato $/x>$
assuma un valore medio pari a $<x|\xi|x>$.
Tale interpretazione fisica è formalizzata nell'ambito della rappresentazione degli stati di un sistema quantistico,
su basi ortonormali di cardinalità finita o infinita, numerabili o meno, tale per cui risulta, come analizzato nel seguito:

\begin{equation}
    \left\{
    \begin{aligned}
         & |x>= \sum_{0}^{v} <x|\xi_i>|\xi_i>+ \int_{0}^{v} <x|\xi'>|\xi'>d\xi' \\
         & <\xi' | \xi''>= \delta(\xi'-\xi'')
    \end{aligned}
    \right.
\end{equation}

ove $\delta$ è la delta di Dirac, e v è la cardinalità della base ortonormale numerabile. $\xi_i$ sono gli autovalori dell'osservabile $\xi$, rispetto a quali si costruisce la base degli autostati.
Si sottointende la complementarietà dello spettro continuo rispetto al numerabile nell'ambito della rappresentazione dello spazio degli stati.


\subsection{Commutazione di osservabili}
Gli operatori non sono in generale commutativi.
Si definisce pertanto il commutatore di due osservabili $\xi$ e $\eta$ l'operatore:

\begin{equation}
    [\xi,\eta]=\xi\eta-\eta\xi
\end{equation}

fisicamente, il commutatore ritorna la possibilità di misurare contemporaneamente due variabili dinamiche.
Matematicamente, ciò equivale ad affermare che i due osservabili formano lo stesso sistema completo di autostati.
Se due osservabili commutano, allora esiste una base di autostati comuni ad entrambi.
Vale anche l'inverso del teorema.
Dunque, data una funzione di osservabili che commutano, la densità di misurare un set di autovalori è data dalla produttoria:

\begin{equation}
    P_{a,b,...,n}=<x|\delta_{\xi,a}\delta_{\eta,b}...\delta_{\nu,n}|x>
\end{equation}

Solo se le variabili commutano talee probabilità ha un senso fisico.


\section{Rappresentazioni}
Rappresentare una grandezza equivale alla ricerca delle coordinate su una data base dello spazio di Hilbert in cui essa esiste.
La più semplice delle basi su cui operare, nell'ambito di un osservabile $\xi$, è data dai sui autostati, normalizzati.
La condizione di normalizzazione, nel più generale dei casi, equivalente alla composizione di uno spettro continuo con uno discreto, è data da:

\begin{equation}
    <\xi'|\xi''> = \delta(\xi'-\xi'') + \delta_{\xi',\xi''}
\end{equation}

ove $\delta$ è da intendere come un funzionale astratto che restituisce 1 se i due argomenti sono uguali, 0 altrimenti.
Nel caso di uno spettro continuo, la delta di Dirac è da intendere come un funzionale tale che:

\begin{equation}
    \int_{0}^{v} \delta(\xi'-\xi'')d\xi' = 1
\end{equation}

dunque, il simbolo $\delta$ nel caso di uno spettro continuo, rappresenta una generalizzazione della delta di Kronecker per uno spettro discreto.
Data tale assunzione, si possono definire le seguenti rappresentazioni.
Generalizzando tale concetto a varietà di dimensionalità superiore, corrispondenti cioè a spazi di Hilbert generati da prodotto dialico
di autovettori di autovettori corrispondenti ad un set di n osservabili commutativi, si può scrivere la condizione di ortogonalità come:

\begin{equation}
    <\xi_1,\xi_2,...,\xi_n|\xi'_1,\xi'_2,...,\xi'_n> = \delta(\xi_1-\xi'_1)\delta(\xi_2-\xi'_2)...\delta(\xi_n-\xi'_n)
\end{equation}

\subsection{Rappresentazione di Bra e Ket}
Bra e Ket rappresentano degli stati di un sistema quantistico, dunque, per il principio di sovrapposizione,
rispetto ad una base ortonormale di H, di autovettori relativi ad un osservabile $\xi$, sono rappresentabili come
una generalizzazione a dimensione potenzialmente infinita dei componenti della base.
Nel caso di un ket, si ha:

\begin{equation}
    |P>= \sum_{0}^{v} <P|\xi_i>|\xi_i>+ \int <P|\xi'>|\xi'>d\xi'
\end{equation}

dove si sottointende il dominio di integrazione sulla componente continua dello spettro.
Data la linearità dell'espressione e del prodotto scalare, si può scrivere:

\begin{equation}
    <Q|P>= \sum_{0}^{v} <Q|\xi_i><\xi_i|P>+ \int <Q|\xi'><\xi'|P>d\xi'
\end{equation}


\subsection{Rappresentazione di un osservabile}
Il caso di rappresentazione di un osservabile introduce un indice di variabilità tensoriale pari a 2, in quanto si ha a che fare con un operatore lineare, operante su un ket e su un bra.
Alla luce di ciò, si può scrivere:

\begin{equation}
    \xi = \sum_{0}^{v} \xi_i|\xi_i><\xi_i|+ \int \xi'|\xi'><\xi'|d\xi'
\end{equation}

Il fatto che l'integrale sia esteso solamente agli elementi diagonali di matrice è naturalmente determinato dalla diagonalizzazione
dello spazio rispetto all'osservabile $\xi$, ossia l'introduzione degli autostati di $\xi$ come base ortonormale di H, ossia tali che:

\begin{equation}
    <\xi' | \xi | \xi''>= \xi' \delta(\xi'-\xi'')
\end{equation}

definendo così la generalizzazione di una matrice diagonale nel caso di uno spettro continuo.

\subsection{Rappresentazione generica}

Nel caso di uno spazio di Hilbert generato dagli autostati di n osservabili commutative, definisco la rappresentazione per
un tensore di rango n $\omega$, dato dalla produttoria dialica degli osservabili rappresentati diagonalmente, come:

\begin{equation}
    <\xi_1,\xi_2,...,\xi_n|\omega|\xi'_1,\xi'_2,...,\xi'_n> = \omega' \delta(\xi_1-\xi'_1)\delta(\xi_2-\xi'_2)...\delta(\xi_n-\xi'_n)
\end{equation}


\subsection{Applicazione delle rappresentazioni alla probabilità di una misura}
Nel caso di una misura di n set di n osservabili commutative, la probabilità di ottenere un set di autovalori $\xi_1,\xi_2,...,\xi_n$ è data da:

\begin{equation}
    P_{\xi_1,\xi_2,...,\xi_n} = <x|\delta_{\xi_1,\xi_1}\delta_{\xi_2,\xi_2}...\delta_{\xi_n,\xi_n}|x>
\end{equation}

si trova immediatamente che la probabilità di una misura di un osservabile non è altro che il modulo quadro dello stato a cui si applica l'operatore set di autovalori in questione.
Dunque la media di una misura di un osservabile è data da:

\begin{equation}
    <\xi> = \int \xi' P_{\xi'}d\xi'
\end{equation}

dove l'integrale è esteso allo spettro dell'osservabile $\xi$, nel caso in cui sia continuo. Alternativamente l'integrale si riduce ad una sommatoria.


\section{Funzione d'onda come rappresentazione di ket}
Una generica funzione di osservabili può rappresentarre un ket in H, ovvero uno stato variabile dinamicamente nel contesto di una sistema quantistico.
Si definisce ora la funzione d'onda come rappresentazione di un ket, ovvero come funzione di osservabili operante sul ket standard $>$:

\begin{equation}
    \psi = |\psi>
\end{equation}

dunque, sottointendendo ora la presenza del ket standard $>$, si desume che la differenza tra una generica funzione di osservabili f e $\psi$ è che
quest'ultima può essere applicata ad un operatore solo a sinistra, mentre f da entrambi i lati.
Le proprietà che tale funzione deve soddisfare, affinhé possa essere rappresentativa dell'evoluzione di un
sistema quantistico nel tempo, sono:

\begin{itemize}
    \item $\psi$ $\in$ $L^2(\mathbb{R}^3)$, a meno di normalizzazione "in scatola".
    \item $P$(x,t)=|$\psi(x,t)$|$^2$ è la densità di probabilità di misurare il sistema in x al tempo t, nel caso in cui $\psi$ sia normalizzata.
    \item La probabilità totale estesa su tutto lo spazio è pari a 1.
\end{itemize}

Fisicamente, dunque, una generica f non rappresenta lo stato del sistema quantistico, mentre la $\psi$ così definita sì.
In generale, è corretto affermare che per ogni funzione d'onda esiste una forma di equazione di Shrodinger tale per cui essa risulta soluzione. Questo non è in generale vero
per una generica funzione di osservabili. (affermazione da verificare...costituirebbe un sunto delle proprietà di $\psi$).


\subsection{Esempio di funzione d'onda}
Sullo spazio di Hilbert degli stati, diagonalizzato rispetto all'osservabile impulso, le armoniche piane relative ad uno spettro continuo di impulsi, costituiscono una base ortonormale.
Pertanto, considerando un modello di pacchetto d'onda, ovverosia un'onda risultante dalla sovrapposizione di armoniche di base tali che l'impulso sia incluso in
un certo range, si può scrivere la funzione d'onda, secondo la rappresentazione di ket dell'equazione ... come:

\begin{equation}
    \psi(x,t)=\int_{p-\Delta p}^{p+\Delta p} C(p)e^{i(px-Et)/\hbar}dp
\end{equation}

Introducendo un'approssimazione tale per cui risulti:

\begin{itemize}
    \item C quasi costante nell'intervallo.
    \item $\epsilon \simeq \Delta k$, con k vettore d'onda, dipendente da p secondo la relazione di De Broglie.
\end{itemize}

si ottiene:

\begin{equation}
    \psi(x,t)= \frac{1}{\sqrt{\pi \Delta k}}\frac{\sin[(x-\omega_1 t)\Delta k]}{x-\omega_1 t}e^i(k_0x-\omega_0t)
\end{equation}

in cui gli indici 0 e 1 sono riferiti all'indice di espansione in serie di Taylor della pulsazione.

\section{Condizioni quantiche}
Si definisce condizione quantica una relazione tra due osservabili, volte a descrivere il sistema quantistico.

\subsection{Parentesi di Poisson quantistiche}
Date le variabili canoniche, si definiscono le parentesi di Poisson classiche come:

\begin{equation}
    [u,v]_P = \sum_{i=1}^{n} \frac{\partial u}{\partial q_i}\frac{\partial v}{\partial p_i} - \frac{\partial u}{\partial p_i}\frac{\partial v}{\partial q_i}
\end{equation}

Servendosi del principio di corrispondenza, si instaura la corrispondenza tra esse e l'operazione di commutazione operatoriale:

\begin{equation}
    [\hat{u},\hat{v}]= \frac{1}{i\hbar}[\hat{u},\hat{v}]_P
\end{equation}

Di grande utilità sono le relazioni di commutazione delle coordinate canoniche, che si traspongono naturalmente a partire
dalle parentesi di Poisson classiche:

\begin{equation}
    \left\{
    \begin{aligned}
         & [q_i, q_j]=0                  \\
         & [p_i, p_j]=0                  \\
         & [q_i, p_j]=i\hbar \delta_{ij}
    \end{aligned}
    \right.
\end{equation}

\subsection{Rappresentazione di Schrödinger}
La rappresentazione di Shrodinger consiste nella formalizzazione delle coordinate canoniche come osservabili operanti sullo spazio delle funzioni d'onda.
Se diagonalizziamo la rappresentazione rispetto alle q, otteniamo la rappresentazione di Schrodinger.
Analizziamo le proprietà che ne derivano.
Qualunque ket rappresentante uno stato del sistema può essere espresso secondo la formalizzazione del ket standard come:

\begin{equation}
    |P>= \psi>\equiv \psi(q)
\end{equation}

Introduco ora un operatore lineare di derivazione sulle q, e ne calcolo le proprietà, da cui emergerà un'importante conclusione in termini di rappresentazione.
Nel caso monodimesionale:

\begin{equation}
    \frac{d}{dq} : \frac{d}{dq} \psi = \frac{d \psi}{dq}
\end{equation}

ne segue che l'operatore di derivazione applicato al ket standard sia uguale al ket nullo.
Inoltre:

\begin{equation}
    \begin{aligned}
         & <\psi \frac{d}{dq} | \psi>= <\psi |\frac{d}{dq} \psi> \quad \forall \psi                                                  \\
         & \int <\psi \frac{d}{dq} | q'> \psi(q')dq'= \int \psi(q') dq' \frac{d\psi(q')}{dq'}=-\int \frac{\psi(q')}{dq'}dq' \psi(q') \\
         & <\psi \frac{d}{dq}|q'>=-\frac{d\psi(q')}{dq'}                                                                             \\
         & <\psi \frac{d}{dq} = -<\frac{d\psi}{dq}
    \end{aligned}
\end{equation}

inoltre vale che il complesso coniugato di un operatore di derivazione è l'operatore di derivazione cambiato di segno.
Dunque si deduce che l'operatore di derivazione è immaginario puro.
Si può anhe ricavare l'insieme rappresentativo dell'operatore di derivazione, rispetto alla base diagonalizzata rispetto alle q:

\begin{equation}
    \begin{aligned}
         & |q''>= \delta(q-q'')>                                                                     \\
         & \frac{d}{dq}|q''>= \frac{d}{dq}\delta'(q-q'')>                                            \\
         & <q'|\frac{d}{dq}|q''>= \frac{d}{dq}\delta(q-q'')\delta(q-q')= \frac{d}{dq} \delta(q'-q'') \\
    \end{aligned}
\end{equation}

Calcolo ora le condizioni quantiche tra l'operatore di posizione e quello di derivazione:

\begin{equation}
    \begin{aligned}
         & \frac{d}{dq} q\psi>= q\frac{d}{dq}\psi> + \psi> \\
         & q\frac{d}{dq}-\frac{d}{dq}q=1                   \\
         & [q,\frac{d}{dq}]=-i\hbar
    \end{aligned}
\end{equation}

quindi un operatore del tipo $-i\hbar\frac{d}{dq}$ è tale che $[q,-i\hbar\frac{d}{dq}]=1$.
Tale operatore è l'operatore quantità di moto unidimensionale nella rappresentazione di Shrodinger.
L'operatore quantità di moto relativo alla direzione $q_r$ è esprimibile come:

\begin{equation}
    \hat{p}_r = -i\hbar \frac{\partial}{\partial q_r}
\end{equation}

Per il principio di sovrapposizione, l'operatore quantità di moto p generico è esprimibile come sommatoria
delle produttorie delle proiettorie dell'operatore quantità di moto direzionale per il relativo ket di base.
la rappresentazione di Shrodinger è univocamente determinata a meno di un fattore di fase.


\subsection{Rappresentazione dell'impulso}
Simmetricamente alla rappresentazione di Shrodinger, si può diagonalizzare lo spazio di Hilbert rispetto agli autostati dell'osservabile impulso, per
esempio le armoniche piane, a meno di modificare il criterio di convergenza degli integrali di misura.
* non mi torna il meno (non dovrebbe essere applicato al bra???)
\begin{equation}
    \begin{aligned}
         & p'<q'|p'>==<q'|p|p'|=-i\hbar\frac{d}{dq'}<q'|p'>                                \\
         & <q'|p'>=c'(p')e^{ip'q'/\hbar}                                                   \\
         & <p' | p''> = \int_{-\inf}^{\inf} <p' |q'><p''|q'>dq'=                           \\
         & \int_{-\inf}^{\inf} c'c'' e^{i(p'-p'')q'/\hbar}dq'= (c')^2 \hbar \delta(p'-p'')
    \end{aligned}
\end{equation}

In questa rappresentazione la coppia di operatori canonici è, per simmetria, data da:

\begin{equation}
    \left\{
    \begin{aligned}
         & \hat{p}=p                   \\
         & \hat{q}=i\hbar \frac{d}{dp}
    \end{aligned}
    \right.
\end{equation}

Per passare dalla rappresentazione di Shrodinger a quella dell'impulso, vale la seguente regola di trasformazione:

\begin{equation}
    <q'|p'>= \frac{1}{\sqrt{h}}e^{ip'q'/\hbar}
\end{equation}

\subsection{Statistiche di osservabili}
Il valor medio di un osservabile applicato ad uno stato $|\psi>$ è dato, sotto le ipotesi di funzione d'onda normalizzata su uno spazio di Hilbert diagonalizzato su autostato dell'osservabile, da:

\begin{equation}
    \begin{aligned}
         & <\xi>= <\psi|\xi|\psi>= \sum \xi_i |\psi_i|^2 \quad discreto     \\
         & <\xi>= <\psi|\xi|\psi>= \int \psi^*\xi'\psi d\xi' \quad continuo
    \end{aligned}
\end{equation}

lo scarto quadratico medio è invece pari a:

\begin{equation}
    (\Delta \xi)^2=<(\xi-<\xi>)^2>= <\psi|(\xi-<\xi>)^2|\psi>
\end{equation}

Si vede con evidenza che l'errore sulla misura di un osservabile è nulla se e solo se lo stato è autostato dell'osservabile, cioé se:

\begin{equation}
    \xi |\psi>= \xi |\xi'>= \xi' |\xi'>= <\xi>|\xi'>
\end{equation}




\subsection{Principio di Indeterminazione di Heisenberg}
Dati due ket $|a>$ e $|b>$:
\begin{equation}
    \left\{
    \begin{aligned}
         & |a>=(A-<A>)|\psi> \\
         & |b>=(B-<B>)|\psi>
    \end{aligned}
    \right.
\end{equation}

vale:

\begin{equation}
    \begin{aligned}
         & (\Delta A)^2=<a|a> \quad , \quad (\Delta B)^2=<b|b> \rightarrow                         \\
         & (\Delta A)^2(\Delta B)^2=<a|a><b|b> \geq |<a|b>|^2 \geq \frac{(<f|g>-<g|f>)^2}{(2i)^2}= \\
         & =\frac{(<FG>-<F><G>-<GF>+<G><F>)^2}{(2i)^2}=\frac{<[A,B]>^2}{(2i)^2} \rightarrow        \\
         & \rightarrow \Delta A \Delta B \geq |\frac{<[A,B]>}{2i}|
    \end{aligned}
\end{equation}

Nel caso degli operatori canonici, vale in particolare, discendendo dalle relazioni di commutazione fondamentali:

\begin{equation}
    \Delta q_i \Delta p_j\geq \frac{\hbar}{2}\delta_{ij}
\end{equation}

E' possibile anche ricavare la relazione che intercorre tra $\Delta t$ e $\Delta E$.
Per un generico stato $\psi$, non necessariamente autostato di H e per una generica variabile F, vale:

\begin{equation}
    \begin{aligned}
         & <H>\equiv E_0= <\psi|H|\psi>                                               \\
         & \Delta E=\sqrt{<H^2>-<H>^2}                                                \\
         & \frac{d}{dt}<F>=\frac{i}{\hbar}<[H,F]>                                     \\
         & \Delta F\Delta H \geq \frac{|<[H,F]>|}{2}=\frac{\hbar}{2}|\frac{d}{dt}<F>| \\
         & dt \simeq Dt \rightarrow \Delta H\Delta t \geq \frac{\hbar}{2}
    \end{aligned}
\end{equation}




\subsection{Operatori di traslazione}
Si definiscono operatori di traslazione, tali per cui risulti valida la seguente relazione:

\begin{equation}
    |R>=c_1|A>+c_2|B> \rightarrow |Rd>=c_1|Ad>+c_2|Bd>
\end{equation}

in tal modo la sola arbitrarietà dei ket traslati risulta essere data da un fattore di fase.
Si vede immediatamente che la traslazione di un ket è lineare, per cui si può definire l'operatore D:

\begin{equation}
    D|A>=|Ad>
\end{equation}

D è determinato a meno di un fattore di fase.
Le leggi fisiche devono essere invarianti per traslazione..
Da tale asserto si possono ricavare le principali propietà dell'operatore D:

\begin{equation}
    \left\{
    \begin{aligned}
         & <Q |P>=c=<Qd|Pd>                                       \\
         & <Q |\bar{D}= <Qd| \rightarrow <Q |D \bar{D}|P>=<Qd|Pd>
    \end{aligned}
    \right.
\end{equation}

dunque, essendo l'immagine dell'operatore coincidente con $L(H)$, ossia l'insieme completo degli operatori
lineari su H, D risulta unitario.
Inoltre:

\begin{equation}
    \begin{aligned}
         & \xi |P>=|R> \rightarrow \xi_d |Pd>= |Rd>         \\
         & D^{-1}\xi D |P>= |R> \quad \forall  \xi \in L(H)
    \end{aligned}
\end{equation}

quindi gli osservabili si trasformano come:

\begin{equation}
    \xi_d = D \xi D^{-1}
\end{equation}
Considerando ora una traslazione infinitesima, mi aspetto che esista, per continuità fisica,

\begin{equation}
    \lim_{\delta x \rightarrow 0} \frac{|Pd>-|P>}{\delta x}= \frac{D-1}{\delta x}|P>\equiv d_x |P>
\end{equation}
dove si è definito l'operatore di traslazione infinitesimo lungo x.
Inoltre, considerando la variabilità dell'operatore in funzione della fase, essa deve tendere ad 1 per $\delta x \rightarrow 0$:

\begin{equation}
    \lim_{\delta x \rightarrow 0} \frac{De^{i\gamma}-1}{\delta x}|P>= \frac{D-1+i\gamma}{\delta x}\equiv d_x+ia_x
\end{equation}

è infatti intuitivo routare un ket mediante un immaginario puro.
Si dimostra inoltre che l'operatore di traslazione infinitesimo è immaginario puro.
Per spostamenti infinitesimi vale:

\begin{equation}
    D= 1+d_x\delta x \rightarrow \delta x (\bar{d_x}+d_x)=0
\end{equation}

Si trova ora la relazione fondamentale tra operatore di traslazione e impulso:

\begin{equation}
    \begin{aligned}
         & \lim_{\delta x \rightarrow 0} \frac{v_d-v}{\delta x}= d_xv-vd_x               \\
         & d_x x- xd_x= \lim_{\delta x \rightarrow 0} \frac{x-\delta x -x}{\delta x}= -1 \\
         & p_x= i\hbar d_x
    \end{aligned}
\end{equation}

Dove si sono trascurati i termini $\delta x^2$, e si è dimostrata la validità della commutazione di $d_x$ con la sola variabile x.
Si trova con ragionamenti analoghi che le condizioni quantiche di $i\hbar d_x$ e $p_x$, rispetto alle altre variabili canoniche, sono le stesse.
Perciò è lecita tale rappresentazione.
La traslazione è un operatore unitario, e dunque è un'isometria. Non è in generale vero che un'isometria sia un operatore unitario, per via
della condizione di suriettività.


\section{Equazioni del moto}
Si vuole ora studiare l'evoluzione di un sistema quantistico in funzione del tempo.
In linea con il principio di causlità, possiamo supporre l'evoluzione temporale come un operatore di traslazione, in accordo con quanto sviluppato
nel paragrafo precedente.
Dato un  ket $|P>$, si può definire l'operatore di traslazione temporale T:

\begin{equation}
    \begin{aligned}
         & T |P>= |P(t)>
    \end{aligned}
\end{equation}

che gode delle proprietà precedentemente dimostrate per un generico operatore di traslazione.
Come precedentemente dimostrato, vale la relazione:

\begin{equation}
    d_t= \lim_{\delta t \rightarrow 0} \frac{T-1}{\delta t}
\end{equation}

In analogia con quanto fatto precedentemente per l'operatore di tralazione di coordinate e l'impulso, posso porre:

\begin{equation}
    \begin{aligned}
         & d_t= i\hbar H
    \end{aligned}
\end{equation}

in questo modo ho:

\begin{equation}
    \begin{aligned}
         & \lim_{\delta t \rightarrow 0} \frac{T-1}{\delta t}|P>= \frac{d |P>}{d to_0}= d_t |P> \\
         & i\hbar \frac{d |P>}{dt}= H|P>
    \end{aligned}
\end{equation}

o, operatorialmente:

\begin{equation}
    i\hbar \frac{d T}{dt}= HT \rightarrow T=e^{-\frac{i}{\hbar}Ht}
\end{equation}

Si dimostra che H è l'hamiltoniana del sistema per un principio di simmetria:
l'invarianza del sistema fisico per traslazioni di coordinate determina la conservazione della quantità di moto,
mentre l'invarianza per traslazioni temporali determina la conservazione dell'energia \footnote{
    Ragionamento non di certo formale, ma logicamente esaustivo. Magari in futuro proverò a fare una dimostrazione
    più rigorosa.
}

Introducendo ora il formalismo del ket standard, su uno spazio di Hilbert tale per cui un osservabile $\xi$ sia diagonale,
si giunge all'equazione di Schrödinger, che descrive l'evoluzione temporale dello stato di un sistema quantistico:

\begin{equation}
    i\hbar \frac{\partial \psi(\epsilon,t)>}{\partial t}= H\psi(\epsilon,t)>
\end{equation}

\subsection{Forma di Heisenberg delle equazioni del moto}
Simmetricamente alla rappresentazione di Schrödinger, è possibile costruire un formalismo tale per cui gli stati rimangano invariati nel tempo,
mentre le osservabili mutino.
Matematicamente, ciò equivale a ripercorrere il processo di ricavazione della trasformazione di una generica variabile dinamica $v$, fatta nella sezione precedente,
nell'ambito di una traslazione temporale.
Si ricavano di seguito le equazioni del moto di Heisenberg per un generico osservabile $\xi$:

\begin{equation}
    \begin{aligned}
         & v_t=T^{-1}vT \rightarrow Tv_t= vT                 \\
         & \frac{dT}{dt}v_t+T\frac{dv_t}{dt}= v\frac{dT}{dt} \\
         & HTv_t+i\hbar T \frac{d v_t}{dt}=v\frac{dT}{dt}    \\
         & i\hbar \frac{dv_t}{dt}=v_tH_t-H_tv_t              \\
         & \frac{dv_t}{dt}=[v_t, H_t]
    \end{aligned}
\end{equation}

In accordo con il principio di corrispondenza, si è ricavata una perfetta analogia con le equazioni del moto classiche.
In particolare, se $\xi$ è una costante del moto, allora commuta con l'Hamiltoniano.
Si può dimostrare che in meccanica quantistica valgono le stesse leggi di conservazione della meccanica classica.

\subsection{Stati stazionari}
Per sistemi stazionari, cioè ad energia costante, vale l'equazione 57.
Diagonalizando lo spazio di Hilbert rispetto all'Hamiltoniana, si può scrivere, per un generico autostato $|H'>$:

\begin{equation}
    T |H'>= e^{-\frac{i}{\hbar}Ht}|H'>= e^{-\frac{i}{\hbar}H't}|H'>
\end{equation}

semplificando notevolmente l'equazione (per via della difficoltà di calcolare l'esponenziale di un operatore).
In questo modo, si è ricavato il seguente risultato: Per un sistema stazionario, lo stato evolve nel tempo solo per un fattore di fase.
Un sistema stazionario è così rappresentato da un autostato dell'Hamiltoniana.
La rappresentazione di ket su uno spazio di Hilbert diagonalizzato rispetto ad H prende il nome di rappresentazione di Heisenberg.
In questa base, è ricavabile l'espressione matriciale dell'evoluzione nel tempo di una generica osservabile $\xi$:

\begin{equation}
    \begin{aligned}
        <H'|\xi(t)|H''>=<H'|e^{\frac{i}{\hbar}Ht}\epsilon e^{-\frac{i}{\hbar}Ht}|H''>=e^{\frac{i}{\hbar}(H'-H'')t}<H'|\xi|H''>
    \end{aligned}
\end{equation}
Nel caso degli elementi di matrice dell'osservabile $\xi$, risulta una periodicità con frequenza pari a:

\begin{equation}
    \nu= \frac{E'-E''}{h}
\end{equation}
in accordo con il modello di Bohr.

\subsection{Equazioni del moto per una particella libera}
L'evoluzione temporale dello stato di unsistema quantistico in uno spazio di Hilbert diagonalizzato rispetto all'impulso,
è descritta dall'equazione di Schrödinger, come già affrontato.
Si calcolano ora le equazioni delle osservabili canoniche nel tempo,secondo la rappresentazione di Heisenberg.
L'Hamiltoniana di un sistema quantistico di particella libera, senza approssimazioni relativistiche, è pari a:

\begin{equation}
    H=c\sqrt{p^2+m^2c^4}
\end{equation}

le equazioni del moto di Heisenberg per la coordinata x sono date da:

\begin{equation}
    \begin{aligned}
         & \frac{dx(t)}{dt}= [x(t),H(t)]          \\
         & i\hbar \frac{dx}{dt}= \frac{c^2p_x}{H}
    \end{aligned}
\end{equation}

Analoghe equazioni si ottengono per le altre coordinate.
nella rappresentazione di Schrödinger, con diagonalizzazione rispetto all'impulso, si ottiene la ben nota equazione di De Broglie.
E' utile notare che le onde piane sono anche autostati di H. Dunque, l'impulso e l'Hamiltoniano commutano. Pertanto l'impulso è una costante del moto di una particella libera.

\section{1D Systems}


Considerando uno stato stazionario, esso evolve nel tempo a meno di un fattore di fase, essendo l'operatore di traslazione temporale dato da:

\begin{equation}
    T=e^{-\frac{i}{\hbar}H't}
\end{equation}
con H' autovalore di H nella rappresentazione di Heisenberg.
Per cui, il problema si riduce semplicemente a calcolare le soluzioni dell'equazione agli autovalori di H per t=0, per poi applicarvi l'operatore di traslazione temporale T.
E' importante notare come la soluzione $\psi(x,t)$ risulti essere ancora un autostato dell'Hamiltoniano, ovviamente, essendo a questo punto la funzione d'onda fattorizzata.

Sebbene i sistemi unidimensionali rappresentino una sorta di paradosso a livello quantistico, essi si rivelano utili nell'estensione a sistemi multidimensionali.
Si pensi, per esempio, al fatto che ogni sistema naturale periodico ammette una decomposizione in oscillazioni proprie.
Si tratta ora di studiare casi specifici, in funzione del potenziale V(x).

\subsection{Buca di potenziale infinita}

Il potenziale V(x) è definito da:

\begin{equation}
    \left\{
    \begin{aligned}
         & V(x)=0 \quad se \quad 0\leq x\leq L \\
         & V(x)=\infty \quad altrimenti
    \end{aligned}
    \right.
\end{equation}

seguono le condizioni al contorno:

\begin{equation}
    \psi(0)=\psi(L)=0
\end{equation}

L'equazione di Shroedinger nella regione interna è quella di una particella libera, nell'ambito di un'approssimazione
non relativistica. Per t=0:

\begin{equation}
    -\frac{\hbar^2}{2m}\frac{d^2}{dx^2}\psi_0= E\psi_0
\end{equation}

la cui soluzione è del tipo:

\begin{equation}
    \left\{
    \begin{aligned}
         & \psi_0=A\sin(kx)+B\cos(kx) \\
         & k=\frac{\sqrt{2mE}}{\hbar}
    \end{aligned}
    \right.
\end{equation}

che, a sistema con le condizioni a contorno diviene:

\begin{equation}
    \begin{aligned}
        \psi_0= A\sin(kx) \quad , \quad k=\frac{n\pi}{L}
    \end{aligned}
\end{equation}

Da cui emerge una quantizzazione dell'energia:

\begin{equation}
    E_n=\frac{\hbar^2n^2\pi^2}{2mL^2}
\end{equation}

Per calcolare la distribuzione di probabilità dello stato n-esimo, calcolo la costante di normalizzazione A:
\begin{equation}
    \begin{aligned}
         & <\psi_n | \psi_n>=1                                              \\
         & \int_{0}^{L} A^2\sin^2(kx)dx=1  \rightarrow A=\sqrt{\frac{2}{L}} \\
         & \psi_n=\sqrt{\frac{2}{L}}\sin(\frac{n\pi x}{L})
    \end{aligned}
\end{equation}

essendo lo spazio di Hilbert dato dalla produttoria dialica R per R, per via della fattorizzazione eseguita, e sapendo che le armoniche piane costituiscono un sistema completo (in un intervallo simmetrico rispetto all'origine, dunque a meno di una traslazione),
è sufficiente dimostrare che le onde stazionarie così trovate formano una base ortonormale per R.
La dimostrazione è semplice e sfrutta la disparità della funzione seno.
Dunque la funzione d'onda traslata temporalmente a t è data da, per un generico stato, come sovrapposizione delle armoniche di base traslate:

\begin{equation}
    T\psi_{0_n}>= e^{-\frac{i}{\hbar}E_nt}\psi_{0_n}= \sum_{1}^{\infty} B_n \sin(\frac{n\pi x}{L})e^{-\frac{i}{\hbar}E_nt}
\end{equation}

Il valor medio della posizione è dato da \footnote{il calcolo si svolge eseguendo un'integrazione per parti}

\begin{equation}
    \begin{aligned}
         & <x>=<\psi_n |x|\psi_>=...=L/2
    \end{aligned}
\end{equation}

Si evince che il valor medio non dipende dal tempo, secondo le aspettazioni.

\subsection{Potenziale deltiforme}

Considero un potenziale attrattivo del tipo:

\begin{equation}
    V(x)= -\alpha \delta(x) \quad , \quad \alpha>0
\end{equation}

da cui segue un'equazione di Shroedinger agli autovalori pari a (omettendo il pedice 0)

\begin{equation}
    -\frac{\hbar^2}{2m}\frac{d^2}{dx^2}\psi-\alpha \delta(x)\psi=E\psi
\end{equation}

\subsubsection[short]{Stati legati}
Mi chiedo se possano esistere stati legati, ossia stati tali per cui la probabilità all'infinito sia nulla.
Per x $\neq$ 0, l'equazione di Shroedinger è:

\begin{equation}
    \left\{
    \begin{aligned}
         & \psi(x)=Be^{k|x|}
         & k=\frac{\sqrt{2mE}}{\hbar}
        \\ B= \sqrt{k}
    \end{aligned}
    \right.
\end{equation}

Dunque esiste solo uno stato legato per E<0.
Tale funzione risulta a quadrato sommabile, ma con derivata discontinua per x=0. Esaminiamone le conseguenze, integrando
l'equazione agli autovalori:

\begin{equation}
    \begin{aligned}
         & \int_{-\epsilon}^{\epsilon} \frac{d^2}{dx^2} \psi dx = \int_{-\epsilon}^{\epsilon}\frac{2m\alpha}{\hbar^2}\delta(x)\psi dx= \int_{-\epsilon}^{\epsilon} \frac{2mE}{\hbar^2}\psi dx \\
         & \psi'(\epsilon)-\psi'(-\epsilon)+\frac{2m\alpha}{\hbar^2}\psi(0)=k^2[\psi(\epsilon)- \psi(-\epsilon)] \rightarrow                                                                  \\
         & \rightarrow \Delta \psi'= -\frac{2m\alpha}{\hbar^2}\sqrt{k} \rightarrow                                                                                                            \\
         & k= \frac{m\alpha}{\hbar^2} \rightarrow E=-\frac{m\alpha^2}{2\hbar^2}
    \end{aligned}
\end{equation}

dunque il salto è finito, per via del fatto che il potenziale è infinito.

Si calcolano ora i valori medi delle osservabili canoniche: * da terminare *

\subsection{Stati del continuo}
Considero E>0, per cui l'equazione agli autovalori assume la forma:

\begin{equation}
    \frac{d^2}{dx^2} \psi+k^2\psi=0
\end{equation}

la cui soluzione più generale sull'intero asse reale è:
\begin{equation}
    \left\{
    \begin{aligned}
         & \psi(x)=Ae^{ikx}+Be^{-ikx} \quad , x<0 \\
         & \psi(x)=Fe^{ikx}+Ge^{-ikx} \quad , x>0 \\
         & k=\frac{\sqrt{2mE}}{\hbar}>0
    \end{aligned}
    \right.
\end{equation}

per le condizioni di riflessione e trasmissione si ha:

\begin{equation}
    \left\{
    \begin{aligned}
         & A+B=F+G                                                                        \\
         & ik(A-B)=ik(F-G) \rightarrow \Delta= ik(F-G-A+B)=-\frac{2m\alpha}{\hbar^2}(A+B)
    \end{aligned}
    \right.
\end{equation}

ne segue, ponendo $\beta=\frac{m\alpha}{\hbar^2k}$:

\begin{equation}
    F-G=A(1+2i\beta)-B(1-2i\beta)
\end{equation}

se G=0 (onda proveniente da $-infty$), si ha:

\begin{equation}
    \left\{
    \begin{aligned}
         & F=A+B
         & F= A(1+2i\beta)-B(1-2i\beta)
    \end{aligned}
    \right.
\end{equation}

da cui:

\begin{equation}
    \left\{
    \begin{aligned}
         & B= \frac{i\beta}{1-i\beta}A \\
         & F= \frac{1}{1-i\beta}A
    \end{aligned}
    \right.
\end{equation}

considerando che il modulo quadro dei coefficienti rappresenta la densità di probabilità dello stato nella rispettiva regione,
è possibile definire le correnti di probabilità come:

\begin{equation}
    \left\{
    \begin{aligned}
         & J_{inc}= \frac{\hbar k}{m}|A|^2 \\
         & J_{tr}= \frac{\hbar k}{m}|F|^2  \\
         & J_{rif}= \frac{\hbar k}{m}|B|^2
    \end{aligned}
    \right.
\end{equation}

da cui si ricavano i coefficienti di riflessione e trasmissione:

\begin{equation}
    \left\{
    \begin{aligned}
         & R= \frac{|B|^2}{|A|^2}= \frac{\beta^2}{1+\beta^2} \\
         & T= \frac{|F|^2}{|A|^2}= \frac{1}{1+\beta^2}       \\
         & \beta= \frac{m\alpha}{\hbar^2k}
    \end{aligned}
    \right.
\end{equation}

\subsection{Buca di potenziale rettangolare simmetrica}

Il potenziale è dato da:

\begin{equation}
    V(x)=
    \left\{
    \begin{aligned}
         & 0 \quad se \quad |x|<L \\
         & V_0 \quad altrimenti
    \end{aligned}
    \right.
\end{equation}

\subsubsection{Stati legati}
Cerco gli staticon energia tale per cui risulti un confinamento nella buca (0$<$E$<V_0$).
Il procedimento è analogo a quello seguito per ricavare le funzioni d'onda nei casi precedenti:
si risolvono le equazioni differenziali agli autovalori nelle 3 aree in cui lla retta risulta divisa, e si raccordano
le soluzioni.

\begin{equation}
    \left\{
    \begin{aligned}
         & \psi''+k^2\psi=0 \quad , \quad |x|<L                                        \\
         & \psi''-\eta^2\psi=0 \quad , \quad |x|>L                                     \\
         & \eta= \frac{\sqrt{2m(V_0-E)}}{\hbar} \quad \kappa= \frac{\sqrt{2mE}}{\hbar}
    \end{aligned}
    \right.
\end{equation}

da cui, risolvendo ed applicando le condizioni a contorno, si ha:

\begin{equation}
    \left\{
    \begin{aligned}
         & \psi_1(x)= Ce^{\eta x}+De^{-\eta x}= Ce^{\eta x}  \\
         & \psi_2(x)= A\cos(kx)+B\sin(kx)                    \\
         & \psi_3(x)= Fe^{\eta x}+Ge^{-\eta x}= Ge^{-\eta x}
    \end{aligned}
    \right.
\end{equation}

Per il teorema di parità, le soluzioni devono essere pari o dispari. Analizzo il caso in cui siano pari.

\begin{equation}
    \left\{
    \begin{aligned}
         & \psi_1(x)= Ce^{\eta x} \rightarrow \psi'_1(x)= \eta Ce^{\eta x}    \\
         & \psi_2(x)= A\cos(kx) \rightarrow \psi'_2(x)= -Ak\sin(kx)           \\
         & \psi_3(x)= Ce^{-\eta x} \rightarrow \psi'_3(x)= -\eta Ce^{-\eta x}
    \end{aligned}
    \right.
\end{equation}

* inserire tutto il calcolo fatto con tavoletta *

\subsubsection{Stati del continuo}
*da terminare*

\subsection{Effetto tunnel}
Si consideri un potenziale del tipo:

\begin{equation}
    V(x)=
    \left\{
    \begin{aligned}
         & 0 \quad se \quad |x|>L \\
         & V_0 \quad altrimenti
    \end{aligned}
    \right.
\end{equation}

Non essendovi chiaramente stati legati, si calcolano i coefficienti di riflessione e trasmissione.

Consideriamo solamente il caso in cui $0<E<V_0$.
L'equazione di Shrodinger agli autovalori diviene:

\begin{equation}
    \left\{
    \begin{aligned}
         & \psi''+k^2\psi=0  \quad , \quad |x|>L                                       \\
         & \psi''-\eta^2\psi=0  \quad , \quad |x|<L                                    \\
         & \eta= \frac{\sqrt{2m(V_0-E)}}{\hbar} \quad \kappa= \frac{\sqrt{2mE}}{\hbar}
    \end{aligned}
    \right.
\end{equation}

Nella regione esterna, le soluzioni saranno funzioni oscillanti, mentre in quella interna vi saranno oscillazioni smorzate.

\begin{equation}
    \psi(x)=
    \left\{
    \begin{aligned}
         & Ae^{ikx}+Be^{-ikx} \quad , \quad x<-L        \\
         & Ce^{\eta x}+De^{-\eta x} \quad , \quad |x|<L
         & Fe^{ikx} \quad , \quad x>L
    \end{aligned}
    \right.
\end{equation}

le derivate prime sono:

\begin{equation}
    \psi'(x)=
    \left\{
    \begin{aligned}
         & ik(Ae^{ikx}-Be^{-ikx}) \quad , \quad x<-L          \\
         & \eta(Ce^{\eta x}-De^{-\eta x}) \quad , \quad |x|<L
         & ikFe^{ikx} \quad , \quad x>L
    \end{aligned}
    \right.
\end{equation}

imponendo la continuità delle funzioni e delle derivate prime, si ottengono le seguenti relazioni:

\begin{equation}
    \left\{
    \begin{aligned}
         & A= \frac{C}{2}\left(1+\frac{ik}{\eta}\right)e^{(ik-\eta)L}+ \frac{D}{2}\left(1-\frac{ik}{\eta}\right)e^{(ik+\eta)L}   \\
         & B= \frac{C}{2}\left(1-\frac{ik}{\eta}\right)e^{-(ik+\eta)L}+ \frac{D}{2}\left(1+\frac{ik}{\eta}\right)e^{(-ik+\eta)L} \\
         & C=\frac{F}{2}\left(1+\frac{ik}{\eta}\right)e^(ik-\eta)L                                                               \\
         & D=\frac{F}{2}\left(1-\frac{ik}{\eta}\right)e^(ik+\eta)L                                                               \\
    \end{aligned}
    \right.
\end{equation}

da cui:

\begin{equation}
    \left\{
    \begin{aligned}
         & A=\frac{F}{4ik\eta}e^{2ikL}\left[(\eta+ik)^2e^{-2\eta L}- (\eta-ik)^2e^{2\eta L}\right] \\
         & B= \frac{F}{2ik\eta}(\eta^2+k^2)\sinh(2\eta L)
    \end{aligned}
    \right.
\end{equation}

Da cui è possibile ricavare i coefficienti di riflessione e trasmissione:

\begin{equation}
    \left\{
    \begin{aligned}
         & R= \frac{|B|^2}{|A|^2} \\
         & T= \frac{1}{|A|^2}
    \end{aligned}
    \right.
\end{equation}

Questo risultato è noto come "Effetto Tunnel" e non ha un analogo in meccanica classica, ed indica che esiste una probabilità non nulla
che una particella attraversi una barriera di potenziale, anche se la sua energia è inferiore a quella della barriera stessa.

Tale effetto è alla base del funzionamento dei diodi.



\subsection{Oscillatore armonico}
\subsubsection{Metodo 1}
La trattazione matematica dell'oscillatore armonico quantistico conduce direttamente alla quantizzazione dell'energia per tale sistema.
Di seguito riporto gli step logici che possono essere seguiti:

\begin{itemize}
    \item Si scrive l'Hamiltoniano dell'oscillatore armonico, mediante il principio di corrispondenza, come:

          \begin{equation}
              \hat{H}=\frac{1}{2m}(\hat{p}^2+m^2\omega^2\hat{q}^2)
          \end{equation}
    \item Si scrivono le equazioni del moto nella rappresentazione di Heisenberg nelle coordinate canoniche:

          \begin{equation}
              \left\{
              \begin{aligned}
                   & \dot{q}=\frac{1}{m}\hat{p} \\
                   & \dot{p}=-m\omega^2\hat{q}
              \end{aligned}
              \right.
          \end{equation}

    \item Si introduce la variabile dinamica complessa $\eta$ tale che:

          \begin{equation}
              \eta= \frac{\hat{p}+im\omega \hat{q}}{\sqrt{2m\hbar \omega}}
          \end{equation}

          e se ne ricava l'equazione del moto:

          \begin{equation}
              \hat{\eta}=\hat{\eta_0} e^{i\omega t}
          \end{equation}

    \item Si ricavano i commutatori tra $\hat{\eta}$ e $H$:

          \begin{equation}
              \left\{
              \begin{aligned}
                   & [\hat{\eta},\hat{\eta}^{\dagger}]=1    \\
                   & [\hat{\eta},H]=\hbar \omega \hat{\eta}
              \end{aligned}
              \right.
          \end{equation}

    \item Considerando H' autovalore di $\hat{H}$, si scrive:

          \begin{equation}
              \hbar \omega<H' |\hat{\eta}\hat{\eta}^{\dagger}|H'>=(H'-\frac{1}{2}\hbar \omega)<H' |H'>
          \end{equation}

          che impone che un generico autovalore H' sia tale che:

          \begin{equation}
              H'\geq \frac{1}{2}\hbar \omega
          \end{equation}

    \item Dalle proprietà dei commutatori si ricava la serie di autovalori assumibili da H:

          \begin{equation}
              H_n=(n+\frac{1}{2})\hbar \omega \quad , \quad n \in N
          \end{equation}

    \item Si può procedere nel trovare la rappresentazione di Schrödinger dell'oscillatore armonico.
\end{itemize}

\subsubsection{Metodo 2, Algebrico}

Considero la rappresentazione di Heisenberg (diagonalizzazione dello spazio di Hilbert rispetto all'Hamiltoniana).
L'Hamiltoniana dell'oscillatore armonico monodimensionale è data da:

\begin{equation}
    \left\{
    \begin{aligned}
         & H=\hbar\omega\tilde{H}=\hbar\omega\frac{1}{2}(\tilde{p}^2+\tilde{x}^2) \\
         & \tilde{p}=\frac{p}{\sqrt{m\hbar \omega}}                               \\
         & \tilde{x}= \sqrt{\frac{m\omega}{\hbar}}x                               \\
         & \tilde{H}|\nu>=\epsilon|\nu>
    \end{aligned}
    \right.
\end{equation}

Introduco ora i seguenti operatori:

\begin{equation}
    \left\{
    \begin{aligned}
         & a= \frac{1}{\sqrt{2}}(\tilde{x}+i\tilde{p})           \\
         & a^{\dagger}= \frac{1}{\sqrt{2}}(\tilde{x}-i\tilde{p})
    \end{aligned}
    \right.
\end{equation}

tali per cui valgano:

\begin{equation}
    \left\{
    \begin{aligned}
         & [a,a^{\dagger}]=1                                    \\
         & \tilde{H}=\dagger aa+\frac{1}{2}\equiv N+\frac{1}{2} \\
         & [N,a]=-a \quad , \quad [N,a^{\dagger}]=a^{\dagger}
    \end{aligned}
    \right.
\end{equation}

vale la diagonalizzazione simultanea di N e H, per via della commutazione tra i due operatori.

\begin{equation}
    \epsilon_\nu=\nu+\frac{1}{2} \rightarrow E_\nu=\hbar \omega(\nu+\frac{1}{2})
\end{equation}

\begin{itemize}
    \item Lo spettro di N è tutto positivo, o al più nullo.
    \item Lo spettro è strettamente positivo.
    \item Lo stato $a|\nu>$ appatiene all'autospazio di N relativo all'autovalore $\nu-1$.
          \begin{equation}
              Na|\nu>=aN|\nu>-a|\nu>=(\nu-1)a|\nu>
          \end{equation}
          Considerando autostati normalizzati, vale:
          \begin{equation}
              ||a|\nu>||= \sqrt{<\nu|\dagger a a|\nu>}=\sqrt{<\nu|N|\nu>}=\sqrt{\nu} \rightarrow a|\nu>=\sqrt{\nu}|\nu-1>
          \end{equation}
    \item Analogamente, si trova che:
          \begin{equation}
              a^{\dagger}|\nu>=\sqrt{\nu+1}|\nu+1>
          \end{equation}
    \item Applicando l'operatore a, si ottiene un autostato di N relativo all'autovalore $\nu-1$. Dunque,
          procedendo con la sua applicazione ad un generico autostato di N, si deve arrivare ad ottenere uno stato nullo e mai negativo.

          Al contrario, applicando l'operatore $a^{\dagger}$, si ottiene un autostato di N relativo all'autovalore $\nu+1$, senza vincoli di limitatezza.
          Dunque si è dimostrato che lo spettro di N è $\mathcal{N}$.
    \item A questo punto, abbiamo terminato, in quanto:

          \begin{equation}
              \begin{aligned}
                   & N |n>=n|n> \quad , \quad n \in \mathcal{N}                                \\
                   & H= \hbar\omega(N+\frac{1}{2}) \rightarrow E_n= \hbar\omega(n+\frac{1}{2})
              \end{aligned}
          \end{equation}
    \item Lo stato ad energia minima (Energia del vuoto) sarà dunque:

          \begin{equation}
              E_0=\frac{1}{2}\hbar \omega
          \end{equation}
\end{itemize}

Questo approccio è fertile in termini di formalismo.

\subsubsection{Metodo Analitico}
*da terminare*


\section{Momento angolare}
\subsection{Momento angolare orbitale}
In termini operatoriali, tramite il principio di corrispondenza, il momento angolare orbitale è rappresentabile come:

\begin{equation}
    \hat{M}=\hat{x}\wedge \hat{p} \leftrightarrow \hat{L_i}=\epsilon_{ijk}\hat{x}_j\hat{p}_k
\end{equation}

E' possibile ricavare le seguenti fondamentali condizioni quantiche di commutazione:

\begin{itemize}
    \item Tra $L_i$ e $x_r$:

          \begin{equation}
              \begin{aligned}
                   & [L_i, x_r]= \epsilon_{ijk}[x_jp_k, x_r]=\epsilon_{ijk}{x_j[p_k, x_r]+[x_j, x_r]p_k} = \\
                   & = -\epsilon_{ijk}{x_ji\hbar\delta_{kr}}=i\hbar\epsilon_{ijr}x_j
              \end{aligned}
          \end{equation}
    \item Tra $L_i$ e $p_r$:

          \begin{equation}
              \begin{aligned}
                   & [L_i, p_r]= \epsilon_{ijk}[x_jp_k, p_r]= \epsilon_{ijk}{x_j[p_k, p_r]+[x_j, p_r]p_k}= \\
                   & = \epsilon_{ijk}{i\hbar \delta_{jr}p_k}=i\hbar\epsilon_{irj}p_j
              \end{aligned}
          \end{equation}


    \item Tra $L_i$ e $L_j$ (Calcoli su "Calcoli integrativi"):

          \begin{equation}
              \begin{aligned}
                   & [L_i,L_j]=...=i\hbar\epsilon_{ijk}L_k
              \end{aligned}
          \end{equation}

    \item Tra $L^2$ e $L_i$:
          \begin{equation}
              \begin{aligned}
                   & [L^2,L_i]=...=0
              \end{aligned}
          \end{equation}
\end{itemize}

Dunque è possibile diagonalizzare simultaneamente lo spazio di Hilbert rispetto a $L^2$ e $L_i$
Per convenzione si sceglie la base di autoket di $L^2$ e $L_z$.

In coordinate polari, gli operatori $L^2$ e $L_z$ assumono la forma:

\begin{equation}
    \left\{
    \begin{aligned}
         & L^2=-\hbar^2\left(\frac{\partial^2}{\partial \theta^2}+\cot{\theta}\frac{\partial}{\partial \phi}+\frac{1}{\sin^2(\theta)}\frac{\partial^2}{\partial \phi^2}\right)
         & L_z=-i\hbar\frac{\partial}{\partial \phi}
    \end{aligned}
    \right.
\end{equation}

Con tale sistema di coordinate, la quantizzazione del momento angolare viene effettuata mediante la risoluzione dell'equazione agli autovalori:

\begin{equation}
    \left\{
    \begin{aligned}
         & L^2Y_{lm}(\theta, \phi)=l(l+1)\hbar^2Y_{lm}(\theta, \phi) \\
         & L_zY_{lm}(\theta, \phi)=m\hbar Y_{lm}(\theta, \phi)
    \end{aligned}
    \right.
\end{equation}

tale per cui risulta che le autofunzioni di base siano le armoniche sferiche:

\begin{equation}
    Y_{lm}(\theta, \phi) = \frac{(-1)^{l+m}}{2^l l!} \sqrt{\frac{2l+1}{4\pi} \frac{(l-m)!}{(l+m)!}} \sin^m(\theta) \left( \frac{d}{d\cos{\theta}} \right)^{l+m} e^{im\phi}
\end{equation}

\subsection{Momento angolare di spin}
Il momento angolare di spin è una proprietà intrinseca di una qualsiasi particella elementare, indipendente dal suo stato di moto.

Ne segue che essa determini un grado di libertà aggiuntivo, e, pertanto, commuta con qualsiasi funzione delle osservabili canoniche.

L'operatore di spin gode delle medesime proprietà dell'operatore di momento angolare orbitale, e così una loro generica combinazione lineare.
Dato che il momento angolare commuta con qualsiasi scalare, e che l'Hamiltoniana è invariante per rotazione spaziale (e dunque uno scalare), il momento angolare è una costante del moto.

Perciò la funzione d'onda sarà fattorizzabile, in coordinate polari, come:

\begin{equation}
    \psi(r,\theta, \phi,\vec{z})=R(r)Y^m_l(\theta, \phi) \chi(\vec{z})
\end{equation}

Assumendo lo spin soltanto due valori, la funzione d'onda sarà data dalla sovrapposizione di due stati a spin positivo e negativo:

\begin{equation}
    \psi(\vec{r},\vec{z})=\psi_-(\vec{r})\chi_-(\vec{z}) + \psi_+(\vec{r})\chi_+(\vec{z})
\end{equation}

Lo spazio vettoriale degli stati di spin è bidimensionale, per cui vale la rappresentazione in ket della base (detti spinori):

\begin{equation}
    \chi_+=\begin{pmatrix}
        1 \\
        0
    \end{pmatrix}
    \quad
    \chi_-=\begin{pmatrix}
        0 \\
        1
    \end{pmatrix}
\end{equation}

Le osservabili di spin, operando su uno spazio vettoriale bidimensionale, sono rappresentabili come matrici 2x2, composizione lineare di matrici di Pauli:

\begin{equation}
    S=\frac{\hbar}{2}\sigma \leftrightarrow
    \sigma_x=\begin{pmatrix}
        0 & 1 \\
        1 & 0
    \end{pmatrix}
    \quad
    \sigma_y=\begin{pmatrix}
        0 & -i \\
        i & 0
    \end{pmatrix}
    \quad
    \sigma_z=\begin{pmatrix}
        1 & 0  \\
        0 & -1
    \end{pmatrix}
\end{equation}

\subsection{Composizione di momenti angolari}

Considero la somma operatoriale di due momenti angolari generici ( e un autostato di $L^2$ e $L_z$):

\begin{equation}
    \left\{
    \begin{aligned}
         & \vec{J}=\vec{J_1}+\vec{J_2} \\
         & J^2|j,m>=j(j+1)\hbar^2|j,m> \\
         & J_z|j,m>=m\hbar|j,m>
    \end{aligned}
    \right.
\end{equation}

Fissagti $j_1$ e $j_2$, considero il sottospazio vettoriale costituito da ket del tipo $|m_1,m_2>$, avente dunque dimensione pari a $(2j_1+1)(2j_2+1)$.

Dimostrando che il ket $|j,m>$ costituisce una base di tale spazio, si costruisce un'agevole rappresentazione del momento angolare totale.

Si verifica che $j$ assume valori compresi tra $|j_1-j_2|$ e $j_1+j_2$, e che $m=m_1+m_2$.

Si può così valutare la dimensione dello spazio generato da $|j,m>$:

\begin{equation}
    \sum_{j_1-j_2}^{j_1+j_2}=(2j_1+1)(2j_2+1)
\end{equation}

Perciò i due spazi, essendo isomorfi, ammettono entrambe le basi.

La trasformazione tra le due basi è fornita dai coefficienti di Clebsch-Gordan:

\begin{equation}
    |j,m>=\sum_{m_1=-j_1}^{j_1} \sum_{m_2=-j_2}^{j_2}<m_1,m_2|j,m>|m_1,m_2>
\end{equation}

tenendo conto che se m non è somma di $m_1$ e $m_2$, il coefficiente di Clebsch-Gordan è nullo.

Si utilizza anche la notazione di produttoria diretta di spazi vettoriali:

\begin{equation}
    (j_1)+(j_2)= \oplus_{j=j_1-j_2}^{j_1+j_2} (j)
\end{equation}

\section{Sistemi stazionari a potenziale centrale}
L'Hamiltoniana di sistemi a campo centrale è del tipo:

\begin{equation}
    H=\frac{1}{2m}p^2+V(r)
\end{equation}

Trattandosi di sistemi stazionari, ci si può ricondurre alla risoluzione dell'equazione agli autovalori dell'Hamiltoniano,
con l'aggiuntiva prerogativa di fattorizzazione delle soluzioni in una parte spaziale e in una angolare (trascurando lo spin).

In particolare, ciò è dato dal fatto che l'Hamiltoniano commuta con $L_3$ e $L^2$, e dunque è possibile trovare delle autofunzioni simultanee dei tre operatori:

\begin{equation}
    \left\{
    \begin{aligned}
         & H|\psi>=E|\psi>           \\
         & \psi= R(r)Y(\theta, \phi)
    \end{aligned}
    \right.
\end{equation}

da cui segue l'equazione degli autovalori dell'Hamiltoniano:

\begin{equation}
    \begin{aligned}
         & \frac{\hbar^2r^2}{R(r)}\left(\frac{\partial^2}{\partial r^2}+\frac{2}{r}\frac{\partial}{\partial r}  \right)R(r)+2mr^2[V(r)-E] =-\frac{1}{Y^l_m(\theta,\phi)}L^2Y^l_m(\theta,\phi) \\
    \end{aligned}
\end{equation}

Dunque entrambi i membri, dipendendo da variabili diverse, devono necessariamente essere uguali ad una costante,
ossia:

\begin{equation}
    \left\{
    \begin{aligned}
         & L^2Y^l_m(\theta,\phi)=l(l+1)\hbar^2Y^l_m(\theta,\phi)                                 \\
         & L_3Y^l_m(\theta,\phi)=m\hbar Y^l_m(\theta,\phi)                                       \\
         & \frac{d^2}{dr^2}R_l + \left(\frac{2m[E-V(r)]}{\hbar^2}-\frac{l(l+1)}{r^2}\right)R_l=0
    \end{aligned}
    \right.
\end{equation}

ridefinendo la funzione $R_l(r)=\frac{\chi_l(r)}{r}$, si ottiene:

\begin{equation}
    \left\{
    \begin{aligned}
         & -\frac{\hbar^2}{2m}\frac{d^2}{dr^2}\chi_l(r)+V_{eff}(r)\chi_l(r)=E\chi_l(r) \\
         & V_{eff}(r)=V(r)+\frac{\hbar^2}{2m}\frac{l(l+1)}{r^2}
    \end{aligned}
    \right.
\end{equation}

\subsection{Buca di potenziale sferica}
Il potenziale è del tipo:

\begin{equation}
    \left\{
    \begin{aligned}
         & -V_0 \quad se \quad 0<r<a \\
         & 0 \quad altrimenti
    \end{aligned}
    \right.
\end{equation}

Si studiano gli stati legati, ovvero tali per cui $-V_0<E<0$.

Per r<a, l'equazione degli autovalori per la parte radiale diviene:

\begin{equation}
    \left\{
    \begin{aligned}
         & \frac{d^2}{dr^2}R_l(r)+\frac{2}{r}\frac{d}{dr}R_l(r)+\left(\alpha^2-\frac{l(l+1)}{r^2}\right)R_l(r)=0 \\
         & \alpha=\frac{\sqrt{2m(E+V_0)}}{\hbar}
    \end{aligned}
    \right.
\end{equation}

Per r>a, l'equazione diviene:

\begin{equation}
    \left\{
    \begin{aligned}
         & \frac{d^2}{dr^2}R_l(r)+\frac{2}{r}\frac{d}{dr}R_l(r)-\left(\beta^2+\frac{l(l+1)}{r^2}\right)R_l(r)=0 \\
         & \beta=\frac{\sqrt{-2mE}}{\hbar}
    \end{aligned}
    \right.
\end{equation}


La prima è risolvibile con funzioni di Bessel, mentre la seconda con funzioni di Hankel.
Per $r<a$ si ha, ponendo $x=\alpha r$:

\begin{equation}
    \frac{d^2}{dx^2}R_l(x)+\frac{2}{x}\frac{d}{dx}R_l(x)+\left(1-\frac{l(l+1)}{x^2}\right)R_l(x)=0
\end{equation}

con soluzione:

\begin{equation}
    R_l(x)=A_j j_l(x)=A_l \frac{\sqrt{\pi}}{2}\sum_{n=0}^{\infty}\frac{(-1)^nx^{2n+l}}{2{2n+l}n!\Gamma(n+l+\frac{3}{2})}
\end{equation}

Per $r>a$ si ha, ponendo $x=i\beta r$:

\begin{equation}
    R_l(r)=Bh_l(i\beta r)
\end{equation}

Esprimendo le condizioni a contorno per le funzioni di Bessel e di Hankel e le loro derivate prime, si ottiene:

\begin{equation}
    \alpha \frac{j'_l(\alpha a)}{j_l(\alpha a)}=\beta \frac{h'_l(i\beta a)}{h_l(i\beta a)}
\end{equation}

La soluzione non è sempre ricavabile analiticamente, ma è possibile analizzare alcuni casi semplici, in funzione di l.

Nel caso di una "onda s" (l=0), si ha:

\begin{equation}
    \psi_0(r)=
    \left\{
    \begin{aligned}
         & A\frac{sin{\alpha r}}{r} \quad se \quad r<a \\
         & B\frac{e^{-\beta r}}{r} \quad se \quad r>a
    \end{aligned}
    \right.
\end{equation}

con condizioni a contorno tali che risulti

\begin{equation}
    \left\{
    \begin{aligned}
         & \alpha \cot{\alpha a}=-\beta \leftrightarrow \eta=-\epsilon \cot{\epsilon}
         & \epsilon=\alpha a \quad , \quad \eta=\beta a
         & \epsilon^2+\eta^2=\frac{2mV_0^2}{\hbar^2}a^2
    \end{aligned}
    \right.
\end{equation}

\subsection{Oscillatore armonico isotropo}
Alla luce della trattazione di inizio sezione, si esprime il potenziale effiace come:

\begin{equation}
    V_{eff}(r)=\frac{1}{2}m\omega^2r^2+\frac{\hbar^2l(l+1)}{2mr^2}
\end{equation}

da cui l'equazione radiale di Shrodinger diviene:

\begin{equation}
    R''(r)+\frac{2}{r}R'(r)+\left[\frac{2m}{\hbar^2}\left(E-\frac{1}{2}m\omega^2r^2\right)-\frac{l(l+1)}{r^2}\right]R''(r)=0
\end{equation}

ovvero:

\begin{equation}
    \left\{
    \begin{aligned}
         & \chi''(\rho)+\left(\epsilon -\rho^2-\frac{l(l+1)}{\rho^2}\right)\xi(\rho)=0                                     \\
         & \rho=\alpha rs \quad , \quad \alpha=\sqrt{\frac{m\omega}{\hbar}} \quad , \quad \epsilon=\frac{2E}{\hbar \omega}
    \end{aligned}
    \right.
\end{equation}

da cui:
\begin{equation}
    \chi(\rho)=e^{-\rho^2/2} \rho^l L_n^l(\rho^2)
\end{equation}




\subsection{Moto in un campo centrale}
Si studa qui il moto di un elettrone attorno al nucleo atomico, sotto l'approssimazione di repulsione coulombiana trascurabile, a meno di un fattore medio, e non relativistica.
Lo studio viene facilitato dalle conoscenza apprese sul momento angolare.
L'Hamiltoniana del sistema è data da:

\begin{equation}
    H=\frac{1}{2m}p^2+V(r)
\end{equation}

Si introduce un sistema di coordinate polari, tale che:

\begin{equation}
    \left\{
    \begin{aligned}
         & x=r\sin{\theta}\cos{\phi} \\
         & y=r\sin{\theta}\sin{\phi} \\
         & z=r\cos{\theta}
    \end{aligned}
    \right.
\end{equation}

Ne seguono le regole di commutazione tra le variabili cartesiane ed r:

\begin{equation}
    \left\{
    \begin{aligned}
         & [r, p_x]=\frac{x}{r} \\
         & [r, p_y]=\frac{y}{r} \\
         & [r, p_z]=\frac{z}{r}
    \end{aligned}
    \right.
\end{equation}

Si definisce $p_r$:

\begin{equation}
    p_r=\frac{1}{r}(xp_x+yp_y+zp_z) \rightarrow [r, p_r]=1
\end{equation}

Sfruttando la relazione sul modulo del momento angolare in funzione di k:

\begin{equation}
    k(k+\hbar)=m_x^2+m_y^2+m_z^2
\end{equation}

si ottiene l'Hamiltoniana in polari:

\begin{equation}
    H=\frac{1}{2m}\left ( \frac{1}{r}p_r^2 r +\frac{k(k+\hbar)}{r^2}\right )+V(r)
\end{equation}

Ne segue l'equazione di Shrodinger:

\begin{equation}
    i\hbar \frac{\partial \psi}{\partial t}= \left \{ \frac{\hbar^2}{2m}\left ( -\frac{1}{r}\frac{\partial^2}{\partial r^2}r+\frac{k(k+\hbar)}{\hbar^2r^2} \right ) +V \right \} \psi
\end{equation}

Ne discende la forma dell'operatore lineare funzione di k:

\begin{equation}
    \frac{k(k+\hbar)}{\hbar^2}= -\frac{1}{\sin{\theta}}\frac{\partial}{\partial \theta}\left ( \sin{\theta}\frac{\partial}{\partial \theta} \right ) - \frac{1}{\sin^2{\theta}}\frac{\partial^2}{\partial \phi^2}
\end{equation}

i cui autostati sono le armoniche sferiche, con autovalori del tipo n(n+1). Per un'autofunzione di questo tipo, $\psi$ risulta:

\begin{equation}
    \psi_{k,m}(\xi)=\frac{1}{r}\chi_{k,l}(r)S_n(\theta, \phi)
\end{equation}

Inserendo tale funzione d'onda nell'equazione di Shrodinger, assumendo che lo stato sia stazionario con autovalore di H pari ad H':

\begin{equation}
    H' \chi_0=\left\{\frac{\hbar^2}{2m}\left( -\frac{d^2}{dr^2}+\frac{n(n+1)}{r^2}\right)+V \right\}\chi_0
\end{equation}

Quata equazione è fondamentale per la determinazione dei livelli energetici H' del sistema:

\begin{equation}
    \forall \quad n \quad in \quad N \quad \exists \quad 2n+1 \quad stati \quad indipendenti
\end{equation}


* discussione delle condizioni a contorno *

\subsection{Atomo di idrogeno}
La precedente derivazione è applicabile alla risoluzione del modello dell'atomo di idrogeno:
L'equazione (72) diviene così:

\begin{equation}
    \left\{\frac{d^2}{dr^2} -\frac{n(n+1)}{r^2} +\frac{2me^2}{\hbar^2}\frac{1}{r} \right\}\chi_0= -\frac{2mH'}{\hbar^2}\chi_0
\end{equation}

ponendo:
\begin{equation}
    \left\{
    \begin{aligned}
         &  & \chi_0(r)=f(r)e^{-\frac{r}{a}}     \\
         &  & a=\pm \sqrt{\frac{-\hbar^2}{2mH'}}
    \end{aligned}
    \right.
\end{equation}

l'equazione diviene:

\begin{equation}
    \left(\frac{d^2}{dr^2} -\frac{2}{a}\frac{d}{dr} -\frac{n(n+1)}{r^2} +\frac{2me^2}{\hbar^2}\frac{1}{r}\right)f(r)=0
\end{equation}

Sostituendo la f(r) sviluppata in serie di potenze, si ottiene:

\begin{equation}
    \sum_{s} c_s \{s(s-1)r^{s-2}-(2s/a)r^{s-1}-n(n+1)r^{s-2}+\frac{2me^2}{\hbar^2}r^{s-1} \}=0
\end{equation}

date le condizioni a contorno, si ottengono le energie possibili per l'elettrone nell'atomo di idrogeno:

* da terminare *

\begin{equation}
    H'=-\frac{me^4}{2s^2\hbar^2}
\end{equation}

\subsection{Effetto Zeeman}
L'effetto Zeeman è un effetto di separazione delle righe spettrali di un atomo in presenza di un campo magnetico esterno.
Esso è dovuto al fatto che l'Hamiltoniana del sistema ammette un termine aggiuntivo, dovuto all'interazione tra il momento magnetico dell'elettrone e il campo magnetico esterno,
che tuttavia non influisce sulla forma delle autofunzioni, ma solo sui livelli energetici, ovvero sugli autovalori di H,
che risulteranno aumentati, immutati o diminuiti di un fattore proporzionale al campo magnetico esterno.
Da qui si evincono le ragioni dell'Effetto Zeeman.
* da terminare *


\section{Teoria delle perturbazioni}
L'approccio di analisi di un sistema perturbato parte da una scissione dell'Hamiltoniana in due parti, una parte H$_0$ relativa al sistema non perturbato, e una parte H' relativa alla perturbazione, funzione di una variabile "piccola" $\epsilon$.
* da terminare successivamente *

\section{Particelle identiche}
L'Hamiltoniana per un sistema di particelle identiche deve essere invariante per permutazione degli osservabili corrispondenti.
E' utile dunque introdurre una notazione data dalla produttoria dialica dei ket relativi alle singole particelle, per rappresentare lo stato del sistema.

\begin{equation}
    |a_1>|a_2>...|a_n>=|a_1,a_2,...,a_n>
\end{equation}

tale rappresentazione è detta "rappresentazione simmetrica".
Introducendo un operatore lineare volto a permutazioni indiciali pari o dispari, P.
Un ket si dice simmetrico se:

\begin{equation}
    P|a_1,a_2,...,a_n>=|a_1,a_2,...,a_n>
\end{equation}

e antisimmetrico se:

\begin{equation}
    P|a_1,a_2,...,a_n>=\pm|a_1,a_2,...,a_n>
\end{equation}
(+ se la permutazione è pari, - altrimenti)

Ne segue che uno stato antisimmetrico è esprimibile mediante una funzione antisimmetrica, e viceversa.
Ciò deve valere ad ogni t.
Un ket antisimmetrico è ottenibile dalla sommatoria estesa sulle permutazioni sugli indici del ket simmetrico, a cui viene applicato l'operatore P, con fattori moltiplicativi $\pm 1$, a
seconda della parità della permutazione.
Conseguentemente, tale ket èesprimibile come il determinante di una matrice ("di Slater") avente per colonne i tensori dello stato simmetrico.
Quindi, dal fatto che colonne uguali determinerebbero un determinante nullo, si deduce che due particelle non possono occupare lo stesso stato quantico.
Tutti gli stati per particelle antisimmetriche devono quindi essere indipendenti.
Inoltre, si deduce che la funzione d'onda di uno stato antisimmetrico è nulla se due particelle si trovano nello stesso stato quantico.
Particelle di questo tipo sono dette "fermioni".
Al contrario, particelle simmetriche sono dette "bosoni".
La funzione d'onda dei fermioni cambia di segno se si scambiano due particelle, mentre quella dei bosoni rimane invariata.
Inoltre, si deduce che la funzione d'onda di uno stato antiimmetrico è nulla se due particelle si trovano nello stesso stato quantico.
Particelle con spin semi-intero sono fermioni, mentre quelle con spin intero sono bosoni.

Quindi, per esempio, due elettroni, protoni o neutroni non possono occupare lo stesso stato quantico, mentre due fotoni possono farlo.

\subsection*{Applicazione agli elettroni}
L'effetto dello spin non è determinante nella variazione dell'Hamiltoniana (l'interazione spin-orbita determina una correzione trascurabile se il campo
magnetico non è intenso), ma è determinante nella variazione della funzione d'onda, in quanto provoca una duplicazione degli stati quantici,
essendo la funzione d'onda fattorizzabile in una parte spazi-temporale e in una parte di spin.
Si introducono due osservabili di permutazione $P^x$ e $P^\sigma$, operanti rispettivamente sulla parte orbitale e su quella di spin di $\psi$.
Formando le permutazioni un gruppo, da una composizione di $P^x$ e $P^\sigma$ si ottiene una permutazione totale $P_a$.

Per ogni stato esistente in natura, per un sistema di particelle identiche, gli autovalori di $P_a$ sono $\pm 1$.

\section{Meccanica quantistica relativistica}
Questa sezione esula dagli argomenti del corso.

Si sviluppa qui una teoria quantistica per gli elettroni invariante per trasformazioni di Lorentz.
Si utilizzerà conseguentemente la notazione indiciale, tale per cui l'impulso, per il principio di covarianza, sarò espresso come:

\begin{equation}
    p_\mu=i\hbar\frac{\partial}{\partial r^{\mu}}
\end{equation}


Per semplicità, si considera c=1 e si sottointende spesso la natura operatoriale delle quantità in questione, in quanto perlopiù deducibile
dal contesto.

\subsection{Equazione di Dirac}


Si consideri il moto di una particella libera, tale per cui l'Hamiltoniana e l'equazione agli autovalori siano date da:

\begin{equation}
    H=\sqrt{p^2c^2+m^2c^4} \rightarrow p_0-\sqrt{m^2c^4+p_1^2+p_2^2+p_3^2}\psi=0
\end{equation}

Tale equazione tiene conto della relazione energia-impulso, ma non è affatto simmetrica per le componenti spaziali e temporali.

Si può facilmente giungere alla seguente forma:

\begin{equation}
    {p_0^2-p_1^2-p_2^2-p_3^2-m^2c^2}\psi=0
\end{equation}

che è relativisticamente invariante, ma non lineare nell'energia ($p_0$), come invece la teoria quantistica richiede, per la linearità dell'equazione di Schrodinger.

Dunque ricerco un'equazione del tipo:

\begin{equation}
    \left\{p_o-\alpha_1p_1-\alpha_2p_2-\alpha_3p_3-\beta\right\}\psi=0
\end{equation}

da cui si ottiene:

\begin{equation}
    \left\{p_0^2-\sum_{123}[\alpha_1^2p_1^2+(\alpha_1\alpha_2+\alpha_2\alpha_1)p_1p_2+(\alpha_1\beta+\beta\alpha_1)p_1]-\beta^2\right\}\psi=0
\end{equation}

da cui risultano le seguenti condizioni per i coefficienti:

\begin{equation}
    \alpha_a\alpha_b+\alpha_b\alpha_a=2\delta_{ab} \quad (a,b=1,2,3 o m)
\end{equation}

Dunque le $\alpha$ anticommutano fra di loro e sono di modulo unitario.

E' possibile esprimere le alpha in termini di matrici di Pauli e matrici $rho$ a 4 componenti:

\begin{equation}
    \alpha_i=\begin{pmatrix}
        0        & \sigma_i \\
        \sigma_i & 0
    \end{pmatrix}
    \quad
    \beta=\begin{pmatrix}
        I & 0  \\
        0 & -I
    \end{pmatrix}
\end{equation}

Naturalmente ora avremo a che fare con una funzione d'onda a 4 componenti.
Si perviene così all'equazione di Dirac:



\section{Bibliografia}
\begin{thebibliography}{9}
    \bibitem{dirac} P.A.M. Dirac, \emph{The Principles of Quantum Mechanics}, Oxford University Press, 1930.
\end{thebibliography}

\end{document}