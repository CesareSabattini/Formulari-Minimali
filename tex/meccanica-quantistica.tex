\documentclass{article}

\usepackage[utf8]{inputenc}
\usepackage{amsmath}
\usepackage{amsfonts}
\usepackage{amssymb}

\title{Meccanica Quantistica}
\author{Cesare Sabattini Manginella}
\date{\today}

\begin{document}

\maketitle

\section{Introduzione e sviluppo del formalismo}

\subsection{Principio di Sovrapposizione}
\subsection{vettori bra e ket}
Data una varietà infinito-dimensionale $V$ su un campo $\mathbb{C}$, un \emph{ket} $<A|$ è un elemento di $V$, mentre un \emph{bra} $|A>$ è un funzionale lineare su $V$.
Tale terminologia è intuitivamente derivante dal fatto che il prodotto scalare è il più generale funzionale lineare definito su uno spazio di Hilbert.
Tale affermazione rappresenta una semplificazione dell'enunciato del Teorema di Rappresentazione di Fisher-Reisz, tale per cui, dato uno spazio di Hilbert H su $\mathcal{C}$:

\begin{equation}
    \forall \phi \in H^* \exists x \in H : \forall y \in H \phi(x)=<x,y>
\end{equation}

In effetti la biunivocità di tale corrispondenza permette la rappresentazione covariante-controvariante dello stesso vettore.
Nell'ambito della meccanica quantistica, si attribuisce la rappresentazione della varietà generata da un ket ad uno stato, che, per il Principio di Sovrapposizione,
risulta combinazione lineare di vettori della base ortonormale che origina H, spazio delle configurazioni (stati) del sistema quantistico.
\footnote[2]{
    Il ket nullo non è associabile a nessuno stato (? o a tutti).
}
Non vi è motivo di discutere la numerabilità dell'insieme di indici non nulli della combinazione, per il teorema di ...

La contrazione degli indici covariante-controvariante del prodotto di bra e l'immaginario coniugato del ket (semplicemente un prodotto interno tra due elementi A e B di H),
è tale da originare un numero complesso.
Si considerano nel seguito prodotti scalari hermitiani, cioè tali che:

\begin{equation}
    <A|B> = \overline{<B|A>}
\end{equation}

\subsection{Spazi Operatoriali}

Si definisce "operatore" un endomorfismo da uno spazio lineare in sè stesso.
Estendendo un'algebra associativa non commutativa su insiemi di operatori, sono definibili spazi operatoriali.

Ragionando in termini di indici covarianti e controvarianti, appare logico che il prodotto di un ket per un bra origini un operatore lineare, operante su bra e su ket.
Secondo tale formalismo, ad un operaotre lineare è fisicamente associata una variabile dinamica, istantaneamente.
Ecco dunque palesato un altra differenza tra fisica classica e quantistica: la non commutatività degli operatori, ossia dell'azione delle variabili dinamiche(?).



\section{Principi Fondamentali}
\section{Bibliografia}
\begin{thebibliography}{9}
    \bibitem{dirac} P.A.M. Dirac, \emph{The Principles of Quantum Mechanics}, Oxford University Press, 1930.
\end{thebibliography}

\end{document}