\documentclass{article}
\usepackage{graphicx} % Required for inserting images
\usepackage{amsmath}

\title{Formulario-Meccanica-Analitica}
\author{Cesare Sabattini}
\date{June 2024}

\begin{document}

\maketitle

\section{Introduction}

\section{Sistemi ad un grado di libertà}

\begin{equation}
    \Ddot{x}=f(x)=-\frac{dU}{dx}
\end{equation}
Integrazione per quadratura:

\begin{equation}
    \int dt = \int \frac{1}{\sqrt{2E-2U}}
\end{equation}

Globalità delle equazioni del moto:

\begin{equation}
    \exists inf(U) \rightarrow |\dot x|=\sqrt{2E} \rightarrow |x-x_0|=\sqrt{2E}t<\infty \forall t \in R
\end{equation}

\subsection{Moti in campi centrali}
E' possibile ricondursi ad un grado di libertà.
Potenziale efficace:
\begin{equation}
    V_{eff}=U+\frac{L^2}{2\rho^2}
\end{equation}
La dimostrazione è semplice: basta partire dalla conservazione di L in polari.

\section{Meccanica Lagrangiana}

\textit{Definizione: Funzionale}

$\psi$ è un funzionale $\leftrightarrow$ $\psi$ : \L $\rightarrow$ R
con L spazio delle curve.

\textit{Estremale di un funzionale}

\begin{equation}
    L \in \mathcal{L} \text{ è estremale per } \psi \leftrightarrow \frac{d}{dt} \frac{\partial L}{\partial \dot{x}} - \frac{\partial L}{\partial x} = 0
\end{equation}

\subsection{Principio di Minima Azione di Hamilton}

Le leggi orarie dei sistemi dinamici coincidono con gli estremali del funzionale 

\begin{equation}
    \Phi = \int_{t_0}^{t} L dt , L=T-U
\end{equation}
$\forall$ sistema di coordinate ($q, \dot q$), in quanto la curva esiste indipendentemente dalla sua descrizione analitica.
Segue il sistema di equazioni di Eulero-Lagrange nelle coordinate generalizzate:
\begin{equation}
    \frac{d}{dt}\frac{\partial L}{\partial \dot q}= \frac{\partial L}{\partial q}
\end{equation}

\section{Elementi di Geometria Differenziale}

\subsection{Carta}
Date U, V varietà,  si definisce carta un'applicazione $\phi$: $\phi$: U $\xrightarrow[su]{1-1}$ V , $\phi^{-1}$: V $\xrightarrow[su]{1-1}$ U, con $\phi$, $\phi^{-1}$ $\in$$ \mathcal{C}^0$.

\subsection{Atlante}
Una collezione di carte $V_\alpha$ si dice atlante per la varietà M se:
\begin{itemize}
    \item $U V_\alpha$= M.
    \item se $V_i \cap V_j != \emptyset \rightarrow$ $V_i$ è compatibile con $V_j$.
\end{itemize}
Al fine della parametrizzazione di una varietà, per non perdere la biettività della carta, è necessario un atlante.

\section{Sistemi vincolati}
Lo spazio delle configurazioni di un sistema con k vincoli, descritto da n coordinate definisce una varietà n-k dimensionale. Si può dunque ragionare intuitivamente per gradi di libertà, ovvero di coordinate necessarie a descrivere la configurazione del sistema.
Se un vincolo è bilatero e dipende unicamente da q e t, si dice $olonomo$.

Dunque, ai fini della risoluzione di un sistema soggetto unicamente a vincoli olonomi:

\begin{itemize}
    \item Individuare lo spazio delle configurazioni M.
    \item Scegliere un sistema di coordinate.
    \item Esprimere l'energia cinetica come forma quadratia nelle coordinate generalizzate:
    \begin{equation}
        T= \sum \mathcal{A}_{ij} \dot q_i \dot q_j
    \end{equation}
    \item Ricavare U(q) su M.
    \item Costruire L= T- U e trovare le equazioni di Eulero-Lagrange.
\end{itemize}

\subsubsection{Esercizio esemplificativo:}
Ricavare il teorema di Clairaut dallo studio del moto libero di un punto materiale su una superficie di rotazione in $\mathcal{R}^3$.

\subsubsection{Esercizio esemplificativo:}
Ricavare equazioni del moto di un punto materiale su una guida circolare in moto rotatorio uniforme intorno ad un asse fisso.

\subsection{Principio di d'Alembert}
Nello studio delle equazioni del moto, si vuole ridurre il problema sostituendo all'inclusione delle reazioni vincolari, la presenza di vincoli. Il principio di d'Alembert formalizza tale possibilità.

\textit{Il lavoro delle reazioni vincolari su spostamenti virtuali è nullo, ovvero:}

\begin{equation}
    \sum_i R \delta \vec{r}=0
\end{equation}

\textit{Dato un vincolo scleronomo, che individua una varietà differenziale M, una legge oraria su M è un estremale condizionato su M del funzionale $\Phi$ se e solo se per esso vale il principio di d'Alembert.}




Dunque il principio di d'Alembert implica l'estremalità confdizionata alla varietà M definita dal vincolo stesso.

\subsection{Teorema di Noether}
Dato un vincolo olonomo, che induce una varietà n-1 dimesionale in $\mathcal{E}^n$, con Lagrangiana L ivi definita, se esiste una famiglia di diffeomorfismi $h^s: M\rightarrow M$, che conservino L, allora il corrispondente sistema ammette una costante del moto, che in coordinate locali vale:

\begin{equation}
    p_q= \frac{\partial L}{\partial \dot q}\cdot \frac{d h^s}{ds}
\end{equation}

\subsection{Trasformata di Legendre}

\[
f: A \rightarrow \mathcal{R}, \quad A \subset \mathcal{R}^n, \quad \text{con} \quad \frac{\partial ^2 f}{\partial x_i^2}>0 \quad \text{continua},
\]
\[
\mathcal{L}f = g: B \rightarrow \mathcal{R},
\]
\[
g(\vec{p}) \equiv \max_{\vec{x}} (\vec{p} \cdot \vec{x} - f(\vec{x}))
\]

f è l'inviluppo di g.
$\mathcal{L}$ è involutiva.


\section{Meccanica Hamiltoniana}

\subsection{Funzione Hamiltoniana}

\begin{equation}
    \mathcal{H}(q,p, t)\equiv \mathcal{L}L(q,\dot q, t)= p\dot q- L(q,\dot q, t)
\end{equation}

\begin{equation}
    L=T-U \rightarrow H= T+U
\end{equation}

\subsection{Equazioni di Hamilton}
\begin{equation}
\left\{
\begin{aligned}
    & \dot p=-\frac{\partial H}{\partial q} \\
    & \dot q= \frac{\partial H}{\partial p}
\end{aligned}
\right.
\end{equation}

Il sistema di 2n equazioni di Hamilton è equivalente al sistema di n equazioni del secondo ordine di Eulero-Lagrange.

\subsection{Conservazione di H}

\begin{equation}
    \frac{d H}{dt}= \frac{\partial H}{\partial t}
\end{equation}


\section{Oscillazioni}
\subsection{Criterio di linearizzazione del sistema lagrangiano}
Dalla linearizzazione di $U$, si ottiene l'equazione di Lagrange linearizzata, che intorno ad un punto di equilibrio è data dalla somma di due forme quadratiche (che qui i ipotizzano definite positive).

\begin{equation}
    L_l=T-U= \frac{1}{2}QAQ- \frac{1}{2}Q\mathcal{H}_UQ \quad Q=q-q_{eq}
\end{equation}

(La linearizzazione è tale in quanto $\nabla$$U$=0 in q=q\_{eq}, in quanto q\_e è estremo del potenziale).

da cui si ricavano le equazioni di Eulero Lagrange del sistema linearizzato:

\begin{equation}
    A\ddot{Q} + BQ = 0, \quad B = \mathcal{H}_U
\end{equation}

I moti descritti dall'equazione sono oscillatori, espressi da (se A, B definite positive):

\begin{equation}
    Q(t)=\sum_{i=1}^n \vec{m_i}[c_1 \cos{\sqrt{\lambda}t}+c_2 \sin{\sqrt{\lambda}t}]
\end{equation}

dove $\lambda_i$ sono gli autovalori relativi a $\Vec{m_i}$, autovettori di B rispetto ad A, cioè

\begin{equation}
    B \vec{m_i}=\lambda_iA\vec{m_i}
\end{equation}

I $\lambda_i$ sono pertanto ottenibili dall'equazione:

\begin{equation}
    det(B-A\lambda)=0
\end{equation}

Segue che gli autovettori sono ortogonali secondo la metrica indotta da A.

\section{Moto relativo}
\subsection{Cinematica relativa}
Dato K s.d.r. mobile, k s.d.r. fisso, q $\in$ k e Q $\in$ K, r $\in$ k coordinata di O\_K vale

\begin{equation}
    q(t)=r(t)+Q(t)= r(t)+ BQ(t)= DQ(t)
\end{equation}
con B matrice di rotazione che "allinea" K con k. D è la composizione di una traslazione con una rotazione, tale che

\begin{equation}
    D: K \rightarrow k
\end{equation}

Ne segue che la velocità del punto in k sia

\begin{equation}
    \dot{q}= \dot(r)+\dot{B}Q+ B\dot{Q}
\end{equation}

Si dimostra che:

\begin{equation}
    \exists \omega \in k : w \times q= \dot{B}Q 
\end{equation}

dunque

\begin{equation}
    \dot{q}= \dot{r}+\omega \times q+ B\dot{Q}
\end{equation}

dunque il moto è elicoidale intorno all'asse di istantanea rotazione.

\subsection{Dinamica relativa}
Vale






\end{document}
